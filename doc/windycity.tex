% Last modified: Tue 15 Jan 2019 04:28:50 PM CST
\documentclass[11pt,letterpaper,oneside]{article}
\usepackage{windycity}

\begin{document}
\title{Windy City}
\subtitle{A Chicago Style for \biblatex}
\author{Brian Michael Chase}
\email{brianmichaelchase@gmail.com}
\website{https://github.com/brianchase/windycity}
\version{2019.01.07}
\maketitle
\tableofcontents\markboth{Contents}{Contents}

\section{Introduction}

\nfootnote{Copyright \textcopyright\ 2019 Brian Michael Chase. Under
the terms of the \LaTeX\ Project Public License, version 1.3,
permission is granted to copy, distribute, or modify this software.
See \url{http://www.ctan.org/tex-archive/macros/latex/base/lppl.txt}.}

Windy City is a style for \biblatex that formats notes,
bibliographies, parenthetical citations, and reference lists according
to the \textit{The Chicago Manual of Style}
(\textit{CMS}).\footnote{\cite{chicago2017}} It accurately handles a
wide range of citations in different formats and includes a set of
options and commands to support special circumstances. It also has
extensive support for citing and arranging different kinds of editors,
translators, and compilers within a single citation. These features
make Windy City especially suitable for academic work.

The following sections assume familiarity with \textit{CMS} and
\biblatex. Section \ref{overview} gives a brief overview of the
style's features. Section \ref{edtrans} discusses the assignment of
editors, translators, and compilers. Section \ref{multivolume}
discusses several issues with multivolume works. Sections \ref{notes}
and \ref{paren} reproduce examples from various parts of \textit{CMS}
with a focus on notes and bibliographies. It will help to compare the
results throughout these sections with the corresponding information
in this document's bibliography database and \LaTeX\ file,
\file{windycity.bib} and \file{windycity.tex}, respectively.

Aside from \biber, which is necessary for a few options, Windy City
has no requirements beyond those of \biblatex version 3.8 or later.

\section{Overview}
\label{overview}

This section covers basic information about Windy City. If you're
completely new to \biblatex, you should probably glance at its
documentation and try one of the styles that come with it, if only to
get a sense of the basic commands. For the impatient, examples in
Sections \ref{default}, \ref{short}, \ref{notes}, and \ref{paren}
might be of more immediate interest.

\subsection{Getting Started}

If you already know how to use \biblatex, getting started with Windy
City is easy. Locate \biblatex on your system, and copy Windy City's
files into their respective directories:

\begin{itemize}[label=]
\item \ldots\path{/biblatex/windycity.dbx}
\item \ldots\path{/biblatex/bbx/windycity.bbx}
\item \ldots\path{/biblatex/cbx/windycity.cbx}
\item \ldots\path{/biblatex/lbx/american-windycity.lbx}
\end{itemize}

\noindent Next, tell \biblatex to load Windy City with the load-time
option \opt{style}:

\begin{verbatim}
   \usepackage[style=windycity]{biblatex}
\end{verbatim}

For some entries in your bibliography database, you may need to add
fields or make other adjustments to get the right output. However,
since Windy City relies as much as possible on standard \BibTeX
fields, and secondarily on \biblatex fields, you may not need to make
major changes. The examples in this document and its accompanying
bibliography database, \file{windycity.bib}, should serve as a guide
for how to manage your input for nearly every circumstance that the
style is meant to handle.

\subsection{The Default Format}
\label{default}

For a first set of examples, consider this passage from \textit{CMS}
\ref{14.30}:

\begin{citeonly}
\item \cite[24--25]{morley1995}
\item \cite{schwartz1992}
\item \cite{kaiser1964}
\item \cite[43]{morley1995}
\item \cite[138]{schwartz1992}
\item \cite[189--90]{kaiser1964}
\end{citeonly}

The output shows Windy City's default format. The first citation of a
work is similar to its entry in the bibliography. It includes all or
most of its bibliographic information. Subsequent citations are
shorter, usually consisting of a short form of the author's name and a
short form of the work's title.

Windy City supports variations on this format. For information on
short forms of citation, including the use of \textit{ibid.}, see
Section \ref{short}. For options to skip parts of citations, change
the order of editors and translators, and more, see Sections
\ref{preops} and \ref{entryops}. For parenthetical citations, see
examples in Section \ref{paren}.

The block below shows Windy City's default bibliography for the
previously cited works:

\begin{bibonly}
\nocite{kaiser1964,morley1995,schwartz1992}
\end{bibonly}

\noindent You may also print a bibliography in an author-date format,
what \textit{CMS} calls a reference list:

\begin{refonly}
\nocite{kaiser1964,morley1995,schwartz1992}
\end{refonly}

To make every instance of \cmd{printbibliography} use the author-date
format, load \biblatex with the preamble option \opt{reflist}:

\begin{verbatim}
   \usepackage[reflist,style=windycity]{biblatex}
\end{verbatim}

\noindent Note that \opt{reflist=true} has the same effect:

\begin{verbatim}
   \usepackage[reflist=true,style=windycity]{biblatex}
\end{verbatim}

To use the author-date format on a case-by-case basis, run
\cmd{printbibliography} with an appropriate \opt{env} option. In order
to use Windy City's author-date format, a so-called ``bib
environment'' must set the style's internal \opt{reflist} toggle to
\opt{true}. Windy City's own such environment is called \opt{reflist},
which you may use as follows:

\begin{verbatim}
   \printbibliography[env=reflist]
\end{verbatim}

As you proceed through the text, note that all examples of citations
and bibliographies are outputs of the style from commands that you can
inspect in the document's source, \file{windycity.tex}, and in its
style file, \file{windycity.sty}. Almost all citations are from
\cmd{cite} or \cmd{parencite}. A few are from more specialized
commands, such as \cmd{cite*} or \cmd{cites}. All example
bibliographies are outputs of the style from \cmd{printbibliography}.
All bibliographic data resides in \file{windycity.bib}.

\subsection{Short Citations}
\label{short}

Aside from parenthetical citations, which \textit{CMS} covers in
Chapter 15, alternative formats receive scant documentation.
Nevertheless, \textit{CMS} does give options. Consider this example
from \textit{CMS} \ref{14.34}:\footnote{Switching formats within a
document isn't a feature of the style and isn't at all convenient.}

\begin{citeonly}
\AtNextCitekey{\toggletrue{short}\toggletrue{firstshort}}
\item \cite[3]{morrison2004a}
\AtNextCitekey{\toggletrue{short}}
\item \cite[18]{morrison2004a}
\AtNextCitekey{\toggletrue{short}}
\item \cite[18]{morrison2004a}
\AtNextCitekey{\toggletrue{short}}
\item \cite[24--26]{morrison2004a}
\AtNextCitekey{\toggletrue{short}\toggletrue{firstshort}}
\item \cite[401-2]{morrison2004b}
\AtNextCitekey{\toggletrue{short}}
\item \cite[433]{morrison2004b}
\AtNextCitekey{\toggletrue{short}\toggletrue{firstshort}}
\item \cite[37--38]{diaz2008}
\AtNextCitekey{\toggletrue{short}}
\item \cite[403]{morrison2004b}
\AtNextCitekey{\toggletrue{short}}
\item \cite[152]{diaz2008}
\AtNextCitekey{\toggletrue{short}}
\item \cite[201-2]{diaz2008}
\AtNextMultiCite{\toggletrue{short}}
\item \cites[240]{morrison2004b}[32]{morrison2004a}
\AtNextCitekey{\toggletrue{short}}
\item \cite[33]{morrison2004a}
\end{citeonly}

Compare that with the style's default output:

\begin{citeonly}
\item \cite[3]{morrison2004a}
\item \cite[18]{morrison2004a}
\item \cite[18]{morrison2004a}
\item \cite[24--26]{morrison2004a}
\item \cite[401-2]{morrison2004b}
\item \cite[433]{morrison2004b}
\item \cite[37--38]{diaz2008}
\item \cite[403]{morrison2004b}
\item \cite[152]{diaz2008}
\item \cite[201-2]{diaz2008}
\item \cites[240]{morrison2004b}[32]{morrison2004a}
\item \cite[33]{morrison2004a}
\end{citeonly}

\noindent In the short format, a work's first citation gives short
names and titles and omits all other publication information.
Consecutive citations of a work may omit the title or, as in the
eleventh note, where the title is the key mark of distinction, the
author's name. To use this format, start \biblatex with the preamble
option \opt{short}. See Section \ref{preops} for more information.

\textit{CMS} \ref{14.34} also shows how to render the passage above
with \textit{ibid.} Unlike previous editions of \textit{CMS}, the 17th
edition discourages its use. As such, \textit{ibid.} is no longer part
of Windy City's default format. Enable it with the preamble option
\opt{ibid} (again, see Section \ref{preops}). The combination of
options \opt{short} and \opt{ibid} yield the following:

\begin{citeonly}
\AtNextCitekey{\toggletrue{short}\toggletrue{firstshort}\toggletrue{ibid}}
\item \cite[3]{morrison2004a}
\AtNextCitekey{\toggletrue{short}\toggletrue{ibid}}
\item \cite[18]{morrison2004a}
\AtNextCitekey{\toggletrue{short}\toggletrue{ibid}}
\item \cite[18]{morrison2004a}
\AtNextCitekey{\toggletrue{short}\toggletrue{ibid}}
\item \cite[24--26]{morrison2004a}
\AtNextCitekey{\toggletrue{short}\toggletrue{firstshort}\toggletrue{ibid}}
\item \cite[401-2]{morrison2004b}
\AtNextCitekey{\toggletrue{short}\toggletrue{ibid}}
\item \cite[433]{morrison2004b}
\AtNextCitekey{\toggletrue{short}\toggletrue{firstshort}\toggletrue{ibid}}
\item \cite[37--38]{diaz2008}
\AtNextCitekey{\toggletrue{short}\toggletrue{ibid}}
\item \cite[403]{morrison2004b}
\AtNextCitekey{\toggletrue{short}\toggletrue{ibid}}
\item \cite[152]{diaz2008}
\AtNextCitekey{\toggletrue{short}\toggletrue{ibid}}
\item \cite[201-2]{diaz2008}
\AtNextMultiCite{\toggletrue{short}\toggletrue{firstshort}}
\item \cites[240]{morrison2004b}[32]{morrison2004a}
\AtNextCitekey{\toggletrue{short}\toggletrue{ibid}}
\item \cite[33]{morrison2004a}
\end{citeonly}

There are still other ways to save space: With the default format, you
can use the preamble option \opt{firstshort} to swap long first
citations for short ones (see Section \ref{preops}). Also with the
default format, you can shorten the author's name in the first
citation if the previous citation is of the same author. To do that,
use the preamble option \opt{idemtracker} (see Section \ref{preops}).
The entry option \opt{noauth} omits the author's name altogether (see
Section \ref{entryops}). And the field \bibfield{shorthand} allows you
to set an abbreviation to stand in place of the author's name, the
work's title, and other elements of a citation (see \ref{14.59}).

In total, Windy City allows a fair amount of control over the format
of notes and bibliographies, consistent with \textit{CMS}. The
remainder of this section, and the next few sections thereafter,
describe these and other features of the style in a more systematic
way.

\subsection{Preamble Options}
\label{preops}

A preamble option is an argument for the \cmd{usepackage} macro that
loads \biblatex. Preamble options affect the format of notes,
bibliographies, and reference lists. Some features of the style
require them.

All options described below are \opt{false} by default. Set them to
\opt{true} by passing the name of the option to \biblatex, with or
without an additional \opt{=true}. In other words, using the option
\opt{annotate} as an example, the following are equivalent:

\begin{verbatim}
   \usepackage[annotate,style=windycity]{biblatex}
   \usepackage[annotate=true,style=windycity]{biblatex}
\end{verbatim}

Bear in mind that Windy City uses many preamble options native to
\biblatex, a few of which you may want to change. These options are
set in \file{windycity.bbx}. In particular, the style sets
\opt{idemtracker} to \opt{false}. If you set it to \opt{true} (or to
some value that implies \opt{true}), Windy City will detect when the
first citation of a work follows another citation of a work by the
same author and print a short form of the author's name. Recall the
previous section with examples citing Toni Morrison's \textit{Song of
Solomon} after \textit{Beloved}. In contexts like those, do you really
need to remind readers of the author's full name? If you think not,
change \opt{idemtracker} to an appropriate value (see Section 3.1.2.3
of the user guide for \biblatex).\footnote{\textit{CMS} seems to have
no policy on this point. In the 16th edition, however, Figure 14.3
shows consecutive citations of works by the same author. Both
citations give the author's full name.} Using \opt{idemtracker} has no
other effect on the style's output.

\begin{optionlist}

\optitem[false]{annotate}{\opt{true}, \opt{false}}

\noindent This option is for printing annotated bibliographies.
Annotations will appear in block paragraphs below their associated
entries. To change the spacing between entries and annotations, change
the value of \cmd{bibitemsep}. Store the text of an annotation in the
\bibfield{annotation} field of the work's bibliography database entry.

\optitem[false]{collsonly}{\opt{true}, \opt{false}}

\noindent Citing individual works of a collection adds an entry for
each work to the bibliography. This could result in needless clutter,
especially after citing many volumes of a multivolume work. To exclude
individual works and print only an entry for the whole collection, use
\opt{collsonly}. It has no effect on many \bibtype{incollection}
entries, such as articles in books, which need (or ought to have) a
place in the bibliography, but does filter out chapters of books and
volumes of collections. For discussion of multivolume works, see
Section \ref{multivolume}.

\optitem[false]{firstshort}{\opt{true}, \opt{false}}

\noindent Use this option to print a short form of the first citation
of each work. The resulting citation consists mainly of the author's
name and the work's title. According to \textit{CMS}, this approach is
optional for documents with complete
bibliographies.\footnote{\textit{CMS} \ref{14.23}. See also
14.29--14.36.} You may use this option in conjunction with \opt{ibid}.
However, \opt{firstshort} adds nothing to \opt{short}. The latter
implies \opt{firstshort} but goes beyond it by giving a short format
to subsequent citations of a work.

\optitem[false]{ibid}{\opt{true}, \opt{false}}

\noindent This option controls whether consecutive citations of a work
on the same page receive an \textit{ibid}. The qualification ``on the
same page'' means that \textit{ibid.} always refers to a work cited on
the current page without an \textit{ibid.} The latter is not a
requirement of \textit{CMS} but seems reasonable, since it prevents
readers from having to look at another page to determine the referent
of an \textit{ibid.} For an example of its output, see Section
\ref{short} or \ref{14.34}. Recall that, as of the 17th edition,
\textit{CMS} discourages the use of \textit{ibid.} (see \ref{14.34}).

\optitem[false]{isbn}{\opt{true}, \opt{false}}

\noindent Use this option to print ISBNs in bibliographies. A work's
ISBN belongs in the \bibfield{isbn} field of its database entry. With
this option, the style will print ISBNs at the end of every entry in
the bibliography, though before annotations. To print the ISBN of a
particular work, see the \bibfield{isbn} entry option in Section
\ref{entryops}.

\optitem[false]{nolos}{\opt{true}, \opt{false}}

\noindent By default, every work with a \bibfield{shorthand} receives
an entry in the bibliography. If you wish to exclude them, say, to
avoid duplication with the output of \cmd{printshorthands}, use
\opt{nolos}.

\optitem[false]{reflist}{\opt{true}, \opt{false}}

\noindent Use this option to print a reference list, that is, a
bibliography in the author-date format. If you choose parenthetical
citations over notes, you should consider using \opt{reflist} to
maintain consistency with \textit{CMS}.

As mentioned in Section \ref{overview}, another way to print a
reference list is to use the \opt{reflist} \opt{env} option of
\cmd{printbibliography}. See Section \ref{overview} for more
information.

In line with \textit{CMS} \ref{15.41}, the choice of \opt{reflist}
affects the format of works in collections. Those works should give
publication information for the volume first, followed by publication
information for the collection. To that end, \opt{reflist} implies the
preamble option \opt{volfirst} (see below).

\optitem[false]{short}{\opt{true}, \opt{false}}

\noindent As shown in Section \ref{short}, this option prints
citations in a short format (see \textit{CMS} \ref{14.34}). The use of
\opt{short} has one feature in common with \opt{ibid}: Just as an
\textit{ibid.} appears only for consecutive citations of a work on the
same page, and so never refers to a citation on a previous page,
\opt{short} drops the title from consecutive citations of a work on
the same page, never in reference to a citation on a previous page. As
with \textit{ibid.}, this feature isn't required by \textit{CMS}, but
it prevents readers from having to look at a previous page to
determine which title a citation refers to.

In contexts where \opt{short} would drop a title from a citation, but
where no name occupies the author's position, it will print the work's
\bibfield{labeltitle}. This can be a short form of the title, either
the title minus the subtitle or the \bibfield{shorttitle}, if present
in the bibliography database. In those situations, the short format is
no different from the default.

As noted earlier, \opt{short} implies \opt{firstshort}. But recall
from Section \ref{short} that you can combine \opt{short} and
\opt{ibid} for even more concise notes.

\optitem[false]{volfirst}{\opt{true}, \opt{false}}

In some examples, \textit{CMS} shows two ways of citing works from
collections: with publication information for the volume preceding
that of the collection or \textit{vice versa}. (See especially
\textit{CMS} \ref{14.119}, \ref{14.121}, and \ref{14.122}, and compare
\ref{14.144} and \ref{15.41}.) In the standard format, it's unclear
whether \textit{CMS} has a preference. But the examples suggest that,
if one must be the default, it's to give priority to the collection.
For the alternative, use \opt{volfirst}. The preamble option
\opt{volfirst} has a global effect. It changes the format of all
relevant citations. The entry option \opt{volfirst} changes the format
on a case-by-case basis (see Section \ref{entryops}).

As mentioned above, \textit{CMS} \ref{15.41} makes clear that
\opt{volfirst} should be the default for reference lists. As such,
\opt{reflist} implies \opt{volfirst}.

For more information, see Section \ref{collorder}.

\end{optionlist}

\subsection{Entry Options}
\label{entryops}

An entry option is a value for the \bibfield{options} field of a
work's database entry. It affects the format of that particular work.
For options that affect the format of every work, see Section
\ref{preops}.

\begin{optionlist}

\optitem[false]{anonauth}{\opt{true}, \opt{false}}

\noindent This option prints the author's name of an anonymously
published work in brackets, as shown in \textit{CMS} \ref{14.79}:

\begin{citebib}
\item \cite{horsley1796}
\end{citebib}

\optitem[false]{anonqauth}{\opt{true}, \opt{false}}

\noindent This option is similar to the previous but adds a question
mark after the author's name, indicating doubt about the authorship.
The following example is from \ref{14.79}:

\begin{citebib}
\item \cite{hawkes1834}
\end{citebib}

\optitem[false]{isbn}{\opt{true}, \opt{false}}

\noindent Use this option to print the ISBN of a particular work in a
bibliography. The ISBN will appear at the end of the work's entry but
before an annotation. To print ISBNs of every work, see the
\bibfield{isbn} preamble option in Section \ref{preops}.

\optitem[false]{noauth}{\opt{true}, \opt{false}}

\noindent This option tells the style to bypass the author's position
of a work in notes and bibliographies. Citations will begin with the
title's position. Below is an example from \textit{CMS} \ref{14.105}:

\begin{citebib}
\item \cite{chaucer1966}
\end{citebib}

\noindent To bypass the author's position in a single note, use
\cmd{cite*} or \cmd{footcite*}. See Section \ref{citecmds}.

\optitem[false]{swapauth}{\opt{true}, \opt{false}}

To swap the places of a book's author with an editor or translator,
use \opt{swapauth}. An example appears in \textit{CMS} \ref{14.104}:

\begin{citebib}
\item \cite{pound1953}
\end{citebib}

\noindent This option works with the \bibtype{book},
\bibtype{collection}, and \bibtype{inbook} entry types. To put a
translator in the author's position when a book also has an editor,
use \opt{swapauth} with \opt{transfirst}. For more information, see
Section \ref{edtranspos}.

\optitem[false]{transfirst}{\opt{true}, \opt{false}}

\noindent According to \textit{CMS}, if a work has both an editor and
a translator, their names should appear in citations in the order in
which they appear on the work's title page (\ref{14.104}). By default,
the style lists editors first. Entries with the option
\bibfield{transfirst} reverse this order: Their translators print
first. If a work's translators and editors are identical, using
\bibfield{transfirst} reverses the order of their roles, say, from
\textit{edited and translated by} to \textit{translated and edited
by}. For more information, see Section \ref{edtranspos}.

\optitem[false]{volfirst}{\opt{true}, \opt{false}}

As an entry option, \opt{volfirst} does on a case-by-case basis what
the \opt{volfirst} preamble option does globally: When set to true, it
changes the format of a work in a collection so that, in
bibliographies and long citations, publication information for the
volume precedes that of the collection.

Some works need this option. Those cases are limited to citations of
volumes in collections in which different volumes have different
authors. Not using \opt{volfirst} would create the impression that the
author of the volume is also the author of the collection. For an
example, see \textit{CMS} \ref{14.122}.

As mentioned in Section \ref{preops}, the preamble option
\opt{reflist} implies the preamble option \opt{volfirst}. As an entry
option, then, \opt{volfirst} is for changing particular standard
citations.

For more information, see Section \ref{collorder}.

\end{optionlist}

\subsection{Citation Commands}
\label{citecmds}

The most important citation commands are already familiar from
\biblatex:

\begin{ltxsyntax}
\cmditem{cite}[prenote][postnote]{key}
\cmditem{footcite}[prenote][postnote]{key}
\cmditem{nocite}{key}
\cmditem*{nocite}|\{*\}|
\cmditem{parencite}[prenote][postnote]{key}
\end{ltxsyntax}

\noindent Insert notes with \cmd{cite} and \cmd{footcite}. Insert
parenthetical citations with \cmd{parencite}. Use \cmd{nocite} to add
works to bibliographies without citing them in the text. Use
\cmd{nocite} with a comma-separated list of entry keys to add
particular works. Use it with an asterisk to add every work in every
\file{bib} file listed in \cmd{bibliography}.

\begin{ltxsyntax}

\cmditem{bibentry}{key}

This command prints the bibliography entry of a work. Although mainly
for testing, you might use a series of these commands to print a
reading list, as for a syllabus. It has no effect on citation
tracking.

\cmditem{cite*}[prenote][postnote]{key}

Use this command to cite a work without printing anything in the
author's position. It comes in handy when the context of a citation
makes the author's name clear, such as when the name appears in the
work's title. From \textit{CMS} \ref{14.78}:

\begin{citebib}
\item \cite*[233]{franklin1868}
\item \cite*[234]{franklin1868}
\end{citebib}

\cmditem{crossref}{key}

Windy City uses this command internally when cross-reference a
previously cited \bibtype{collection}. You could use it in a text, but
it's only output is a work's \bibfield{labelname} and
\bibfield{labeltitle}, separated by a comma and a space.

\cmditem{footcite*}[prenote][postnote]{key}

Like \cmd{cite*}, this command suppresses the author's position of a
note but otherwise functions like \cmd{footcite}.

\cmditem{fullcite}[prenote][postnote]{key}

Mainly for testing, this command prints the first, long citation of a
work in the default format, regardless of whether the preamble options
\opt{short} or \opt{firstshort} are true. Like \cmd{bibentry}, it has
no effect on citation tracking.

\cmditem{fullcite*}[prenote][postnote]{key}

Like \cmd{fullcite}, this command prints the first, long citation of a
work in the default format but, like \cmd{cite*}, skips the author's
position. Also like \cmd{cite*}, the corresponding entry in the
bibliography retains the author's position. It has no effect on
citation tracking.

\cmditem{parencite*}[prenote][postnote]{key}

Use this command to print a parenthetical citation without the
author's position. The most likely context for this is a sentence in
which the author receives explicit mention. Here's an example from
\textit{CMS} 15.25:

\begin{quote} Fiorina et al. \parencite*{fiorina2005} and Fischer and
Hout \parencite*{fischer2006} reach more or less the same conclusions.
In contrast, Abramowitz and Saunders \parencite*{abramowitz2005}
suggest that the mass public is deeply divided between red states and
blue states and between churchgoers and secular voters. \end{quote}

The source for the passage above contains:

\begin{verbatim}
   \begin{quote} Fiorina et al. \parencite*{fiorina2005}...
   Fischer and Hout \parencite*{fischer2006}... Abramowitz
   and Saunders \parencite*{abramowitz2005}... \end{quote}
\end{verbatim}

\cmditem{refentry}{key}

Similar to \cmd{bibentry}, this command prints the reference list
entry of a work. It, too, has no effect on citation tracking.

\cmditem{reprint}[postnote]{key}

This command precedes a citation with ``reprinted in'' and then skips
the author's position and the work's title. This is useful for citing
a reprint of a work after citing the original:

\begin{citebib}
\item \cite{frankfurt1969}; \reprint[1--10]{frankfurt1988.1}
\end{citebib}

\noindent The output above results from:

\begin{verbatim}
   \cite{frankfurt1969}; \reprint[1--10]{frankfurt1988.1}
\end{verbatim}

\end{ltxsyntax}

\subsection{Additional Data Fields}
\label{datafields}

Windy City uses several data fields that are not available with
\biblatex.

\begin{marglist}

\item[bookbooktitle] This field is for the style's internal use. Do
not use it in a bibliography database.

\item[bookyear] Also for the style's internal use. Do not use it in a
bibliography database.

\item[editoraddon] Use this field to include additional editorial
information about a work. In Section \ref{entryops}, an example from
\textit{CMS} \ref{14.105} uses it: In the work's database entry,
\bibfield{editoraddon} contains ``from materials compiled by John M.
Manly and Edith Richert, with the assistance of Lilian J. Redstone et
al.'' This content prints after the editor's position, without
intervening punctuation. Again, in the bibliography, the result is:

\begin{bibonly}
\nocite{chaucer1966}
\end{bibonly}

\item[seriesaddon] This field is for additional information about a
book's series, such as \textit{2nd ser.} and \textit{n.s.}. For
examples, see \ref{14.126} and \ref{14.184}.

\item[shortbooktitle] This field is for the short form of a
\bibfield{booktitle}, just as \bibfield{shorttitle} is for
\bibfield{title}. Nevertheless, its use is internal. You never need to
use it in a bibliography database. Instead, always use
\bibfield{shorttitle}.

\item[shortmaintitle] This field is for the short form of a
\bibfield{maintitle}. Use it in a bibliography database when a work's
\bibfield{maintitle} may occupy the position of a regular title in a
short citation and needs a short form. This should only happen with
certain works in collections. See, for example, the citation of
\textit{The Complete Tales of Henry James} in Section
\ref{multivolume}.

\end{marglist}

\subsection{Data Fields for Series}

Often, the only question about a book's series is whether to count it
as a series at all, rather than as the title of a multivolume work.
After that, the most common question is how to format a series number.
Sometimes, the number appears alone, with no preceding abbreviation.
But it may also appear with \textit{vol.} or
\textit{no.}\footnote{\textit{CMS} \ref{14.123}}.

In order to avoid using multiple fields for essentially the same task,
the style uses \bibfield{number} for all of them. By default, it
places no abbreviation before the number. Add abbreviations to the
field as necessary. In the database entry for the example below,
\bibfield{number} contains \textit{vol. 6}:

\begin{citebib}
\item \cite{cochrane1987}
\end{citebib}

To indicate the run of a series, such as \textit{2nd ser.} or
\textit{n.s.}, use \bibfield{seriesaddon} (see Section
\ref{datafields}). However, this does not apply to journals, where
such labels usually modify a journal's title. For them, use
\bibfield{series}, as in this example from \ref{14.184}:

\begin{citebib}
\item \cite{moraes1950}
\end{citebib}

\subsection{Entry Types}
\label{entrytypes}

In a bibliography database, every entry has an entry type. The style
recognizes the standard ones for \BibTeX, as well as some that are
specific to \biblatex. A relatively small number of entry types are
basic. The style treats the rest as aliases of the basic ones.

\begin{typelist}
\RaggedRight

\typeitem{article}

Aliases: \bibtype{periodical}

\typeitem{book}

Aliases: \bibtype{booklet}, \bibtype{collection}, \bibtype{manual},
\bibtype{proceedings}, \bibtype{report}, \bibtype{techreport}

\typeitem{incollection}

Aliases: \bibtype{bookinbook}, \bibtype{conference},
\bibtype{inproceedings}, \bibtype{inbook}, \bibtype{letter},
\bibtype{suppbook}, \bibtype{suppcollection}

\typeitem{letter} No aliases
\typeitem{misc} No aliases
\typeitem{online} No aliases
\typeitem{patent} No aliases

\typeitem{reference}

Aliases: \bibtype{inreference}

\typeitem{review} No aliases

\typeitem{thesis}

Aliases: \bibtype{mastersthesis}, \bibtype{phdthesis},
\bibtype{unpublished}

\end{typelist}

\noindent For the most part, you may assign every work to the basic
entry types listed above. A PhD thesis, for example, may have the
entry type \bibtype{thesis} or \bibtype{phdthesis}; the output is the
same. An exception applies to books in collections: Every book in a
one-volume or multivolume collection needs the \bibtype{inbook} or
\bibtype{bookinbook} entry type.\footnote{Using \bibtype{inbook} for
books in collections departs somewhat from the specifications of
\BibTeX\ and \biblatex, though it seems harmless enough.} You may use
either \bibtype{inbook} or \bibtype{bookinbook}. If an entry belongs
to a type other than the ones listed above, the style processes it as
a book.

One comment about \bibtype{reference} and \bibtype{inreference}
entries: You may cross-reference \bibtype{inreference} entries to
\bibtype{reference} entries, as with articles in books, but you can
get the same output using one or the other entry type alone. Take an
example from \textit{CMS} \ref{14.232}:

\begin{citeonly}
\item \cite{salvation1980}
\end{citeonly}

A bibliography database could have an \bibtype{inreference} entry for
the article cross-referenced to a \bibtype{reference} entry for
\textit{Encyclopaedia Britannica}:

\begin{verbatim}
   @InReference{salvation1980,
     title = {salvation},
     crossref = {britannica1980}
   }
   @Reference{britannica1980,
     organization = {{\emph{Encyclopaedia Britannica}}},
     edition = {15},
     year = {1980}
   }
\end{verbatim}

\noindent This approach makes sense if you plan to cite more than one
article from the source.\footnote{Incidentally, reference work's don't
always have titles in italics. As a result, you need to handle it in
your bibliography database.} But you could also have a single entry of
either type with the same data, like this:

\begin{verbatim}
   @Reference{salvation1980,
     organization = {{\emph{Encyclopaedia Britannica}}},
     edition = {15},
     title = {salvation},
     year = {1980}
   }
\end{verbatim}

For unusually complicated ci\-ta\-tions---or those just not supported
by the style---consider using the \bibtype{misc} entry type. The style
formats these entries with a small number of fields but in way that
makes it a fallback for almost anything. The example below is from
\textit{CMS} 14.264:

\begin{citebib}
\item \cite{roosevelt1959}
\end{citebib}

\noindent The database entry for this work contains most of the format
in \bibfield{usera} (for notes) and \bibfield{userb} (for
bibliographies). In \bibfield{title}, the style needs manual
formatting, since works of this type may have titles in italics or
quotation marks. Some trickery with the year helps with formatting the
reference list entry.

\begin{verbatim}
   @Misc{roosevelt1959,
     author = {Roosevelt, Eleanor},
     title = {\mkbibquote{Is America Facing World
              Leadership?}},
     usera = {convocation speech, Ball State Teacher's
              College, May 6, \thefield{year}, radio
              broadcast, reel-to-reel tape, MPEG copy,
              1:12:49},
     userb = {Convocation Speech. Ball State Teacher's
              College. May 6, \thefield{year}. Radio
              broadcast. Reel-to-reel tape. MPEG copy.
              1:12:49},
     url = {http://libx.bsu.edu/cdm4/singleitem/collection/
            ElRoos/id/1},
     year = {1959}
   }
\end{verbatim}

\noindent No other example in this document relies on \bibtype{misc}.

\section{Editors, Translators, and Compilers}
\label{edtrans}

Windy City offers significant control over the handling of editors,
translators, and compilers. Taking advantage of it, however, may not
seem intuitive at first. This section covers the options and issues
that you need to grasp in order to master this aspect of the style.

\subsection{Types of Editors, Translators, and Compilers}
\label{edtransnames}

For the most part, Windy City associates the \bibfield{editor} and
\bibfield{translator} fields with the lowest level title within the
scope of an entry. In most cases, then, you can assign editors and
translators simply by adding the \bibfield{editor} and
\bibfield{translator} fields to a database entry. Below is the entry
for an example in \textit{CMS} \ref{14.104}:

\begin{verbatim}
   @Collection{adorno1999,
     author = {Adorno, Theodor W. and Benjamin, Walter},
     title = {The Complete Correspondence, 1928–1940},
     editor = {Lonitz, Henri},
     translator = {Walker, Nicholas},
     address = {Cambridge, MA},
     publisher = {Harvard University Press},
     year = {1999}
   }
\end{verbatim}

\begin{citebib}
\item \cite{adorno1999}
\end{citebib}

\noindent When the style processes this entry, it identifies the
editor and translator of the work with the names, respectively, in the
\bibfield{editor} and \bibfield{translator} fields. Since the entry
doesn't use the  \opt{transfirst} entry option (see Section
\ref{edtranspos}), the resulting output list the editor and
translator, in that order, after the authors and title.

Cross-referencing introduces a bit more complexity, though the
principle is the same: Within the scope of an entry, the style
associates \bibfield{editor} and \bibfield{translator} with the lowest
level title. In this example from \textit{CMS} \ref{14.30}, an essay
is cross-referenced to a collection:

\begin{verbatim}
   @InCollection{kaiser1964,
     author = {Kaiser, Ernest},
     title = {The Literature of Harlem},
     shorttitle = {Literature of Harlem},
     crossref = {clarke1964}
   }
   @Collection{clarke1964,
     editor = {Clarke, J. H.},
     title = {Harlem},
     subtitle = {A Community in Transition},
     address = {New York},
     publisher = {Citadel Press},
     year = {1964}
   }
\end{verbatim}

\begin{citebib}
\item \cite{kaiser1964}
\end{citebib}

\noindent Since \bibfield{editor} appears within the
\bibtype{collection} entry, Windy City associates the editor's name
with the title of that entry, \textit{Harlem}. If you moved
\bibfield{editor} from \bibtype{collection} to \bibtype{incollection},
the association would change to the \bibfield{title} of that entry,
\textit{The Literature of Harlem}.

Consider another example, this one of an essay in a collection:

\begin{citebib}
\item \cite{petrarca1948}
\end{citebib}

\noindent Hans Nachod translated ``The Ascent of Mont Ventoux,'' among
other works in the collection, but not \textit{every} work in the
collection. Thus, the \bibfield{translator} field must fall within the
scope of the \bibtype{incollection} entry:

\begin{verbatim}
   @InCollection{petrarca1948,
     author = {Petrarca, Francesco},
     title = {The Ascent of Mont Ventoux},
     translator = {Nachod, Hans},
     pages = {36–46},
     crossref = {cassirer1948}
   }
   @Collection{cassirer1948,
     editor = {Cassirer, Ernst and Kristeller, Paul Oskar and
               Randall, Jr., John Herman},
     title = {The Renaissance Philosophy of Man},
     address = {Chicago},
     publisher = ucp,
     year = {1948}
   }
\end{verbatim}

\noindent By the same token, since \bibfield{editor} falls within the
scope of \bibtype{collection}, the style associates it with the
\bibfield{title} of that entry, \textit{The Renaissance Philosophy of
Man}.

Sometimes, you need to designate the role of an editor with the field
\bibfield{editortype}. The values of \bibfield{editortype} are
\textit{maintitle}, \textit{series}, \textit{issuetitle}, and
\textit{compiler}.\footnote{You could use \textit{title} but may omit
it.} As for the latter, Windy City treats \textit{compiler} as a kind
of editor. So, if you need to add a compiler to an entry, add the
compiler's name to an available field for an editor, then add an
appropriate \bibfield{editortype} field with the value
\textit{compiler}. It may help to see the bibliography database entry
for an example in \textit{CMS} \ref{14.103}:

\begin{verbatim}
   @Book{schechter2011,
    editor = {Schechter, Harold, and Kurt Brown},
    editortype = {compiler},
    title = {Killer Verse},
    subtitle = {Poems of Murder and Mayhem},
    address = {London},
    publisher = {Everyman Paperback Classics},
    year = {2011}
   }
\end{verbatim}

\begin{citebib}
\item \cite{schechter2011}
\end{citebib}

Follow the same pattern for editors of a \bibfield{maintitle},
\bibfield{series}, or \bibfield{issuetitle}. Here's an example from
\textit{CMS} \ref{14.118}:

\begin{verbatim}
   @Book{james1963.5,
     author = {James, Henry},
     title = {1883–1884},
     volume = {5},
     maintitle = {The Complete Tales of Henry James},
     shortmaintitle = {Complete Tales of Henry James},
     editor = {Edel, Leon},
     editortype = {maintitle},
     address = {London},
     publisher = {Rupert Hart-Davis},
     year = {1963}
   }
\end{verbatim}

\begin{citebib}
\item \cite{james1963.5}
\end{citebib}

\noindent If there were no \bibfield{editortype} assigning the
\bibfield{editor} to the \bibfield{maintitle}, the style would
associate it with the volume. In short, for any title at a higher
level than \bibfield{title}, you need to use \bibfield{editortype} to
assign an editor to it.

One complication remains: There are other name lists for editors than
\bibfield{editor}. The style also uses \bibfield{editora},
\bibfield{editorb}, and \bibfield{editorc}.\footnote{The style uses
\bibfield{translatora} when it processes cross-referenced works. In a
bibliography database, only use \bibfield{translator}.}

Reserve \bibfield{editor} for the lowest level title in a bibliography
database entry. That should be the \bibfield{title} field. The next
level up, as it were, is for \bibfield{editora}, followed by
\bibfield{editorb}, and so on. Remember to include the appropriate
\bibfield{type} field to indicate an editor's role. These fields are
\bibfield{editortype} (for \bibfield{editor}),
\bibfield{editoratype} (for \bibfield{editora}),
\bibfield{editorbtype} (for \bibfield{editorb}), and
\bibfield{editorctype} (for \bibfield{editorc}).

As it happens, Windy City puts a lot of effort into sorting out where
to print the names of editors and translators, so it's a bit more
clever than the previous paragraph would suggest. In particular, if
you use \bibfield{editortype} to assign an \bibfield{editor} to a
\bibfield{maintitle} or some other higher level title, and have an
\bibfield{editora} with no corresponding \bibfield{editoratype}, Windy
City will assume that \bibfield{editora} is the editor of the
\bibfield{title}. Regardless, the best practice is the follow the rule
of thumb described above, reserving \bibfield{editor} for
\bibfield{title} and working up from there.

\subsection{Switching Places and Roles}
\label{edtranspos}

Normally, Windy City prints editors' names first. But if translators
are listed first on a work's title page (or in some other relevant
place), you may want to reverse the order and print the translators'
names first. For that, use the entry option \opt{transfirst}. Compare:

\begin{citebib}
\item \cite{doe2010a}
\item \cite{doe2010b}
\end{citebib}

If a work has no author, but has an editor, the style will print the
name of the editor in the author's position. The same happens if a
work has no author but has a translator. The following are examples
from \textit{CMS} \ref{14.103}:

\begin{citebib}
\item \cite[100]{egan2014}
\item \cite[34]{silverstein1974}
\end{citebib}

What if a work has no author but has editors and translators? Since
Windy City gives priority to editors, it defaults to printing the
editors' names in the author's position:

\begin{citebib}
\item \cite{smith2002a}
\end{citebib}

\noindent Use \opt{transfirst} to reverse them:

\begin{citebib}
\item \cite{smith2002b}
\end{citebib}

Similarly, if a work's editors and translators are identical, the
style will print the editors' role first, as in, ``Edited and
translated by\ldots'' Using \opt{transfirst} reverses them. Here's an
example from \textit{CMS} \ref{14.104}:

\begin{citebib}
\item \cite{menchu1999}
\end{citebib}

Since the style treats a compiler as a kind of editor, the comments
above apply to compilers: If a work has compilers and translators,
Windy City will print compilers' names first, unless you use
\opt{transfirst}.

In rare cases, you may want to swap the position of an author and an
editor or translator. \textit{CMS} gives this example in \ref{14.104}:

\begin{citebib}
\item \cite{pound1953}
\end{citebib}

\noindent You can accomplish this effect with the entry option
\opt{swapauth}, which works for \bibtype{book}, \bibtype{collection},
and \bibtype{inbook} entry types. But beware: Windy City doesn't look
ahead to see if there really is an editor or translator to take the
author's place. If it doesn't find one, it will still print the
author's name after the title, leaving the author's position empty.
Also, if it finds both an editor and a translator, and they're not the
same person, it will print the editor's name in the author's
position---again, giving priority to editors. If you want the
translator's name in the author's position, use \opt{swapauth} with
\opt{transfirst}.

Another caveat: For correct sorting in a bibliography or reference
list, a work that uses \opt{swapauth} needs a field like
\bibfield{sortname} to sort it by the name of the editor or translator
whose name will occupy the author's position. It would be nice if
Windy City could do this for you, but at present no feature of
\biblatex seems to allow on-the-fly changes to sorting. Below
is the entry for the example above:

\begin{verbatim}
   @Book{pound1953,
     options = {swapauth},
     author = {Pound, Ezra},
     title = {Literary Essays},
     editor = {Eliot, T. S.},
     sortname = {Eliot, T. S.},
     address = {New York},
     publisher = {New Directions},
     year = {1953}
   }
\end{verbatim}

\section{Collections}

Before you cite a collection or one of its volumes, you need to
consider how you want the citation to look and even what type of
collection it is. These issues have implications for how you structure
entries in your bibliography database and how you use citation
commands.

\subsection{The Order of Publication Data}
\label{collorder}

A work in a collection usually has a title and perhaps other
publication data distinct from that of the collection. When preparing
your bibliography database, the most basic choice to make about such a
work is which publication data has priority, the volume's or the
collection's. Consider an example from \textit{CMS} \ref{14.119}:

\begin{citebib}
\item \cite{armstrong2014}
\end{citebib}

\noindent The editor and title of the collection precedes that of the
volume. In notes and bibliographies in the default format,
\textit{CMS} gives you the option of reversing this order. So does
Windy City. You can reverse the order on a case-by-case or global
basis with the entry or preamble option \opt{volfirst}.

\begin{citebib}
\AtNextCitekey{\toggletrue{volfirst}}
\item \cite{armstrong2014}
\AtNextBibliography{\toggletrue{volfirst}}
\end{citebib}

In contexts where information for just one title appears (certain
short citations), the one with priority determines which one
identifies the work. By default, as you can see in the second note
below, that's the collection.

\begin{citeonly}
\item \cite{armstrong2014}
\item \cite[45]{armstrong2014}
\end{citeonly}

\noindent Whereas, using \opt{volfirst}:

\begin{citeonly}
\AtNextCitekey{\toggletrue{volfirst}}
\item \cite{armstrong2014}
\AtNextCitekey{\toggletrue{volfirst}}
\item \cite[45]{armstrong2014}
\end{citeonly}

Aside from the change in editors and titles, notice that the volume
number is missing from the second note in the passage above. When the
collection has priority, so that the collection's title identifies the
work, the volume number should appear in the citation. There are
enough examples in \textit{CMS} to make that clear. But when the
volume has priority, the volume number seems optional. It might even
cause confusion, since the collection is what comes in volumes, not
the volumes themselves. In any event, Windy City prints the volume
number in the first case but not in the second.

Sometimes, you may always want the publication data for volumes to
have priority. If so, compile your documents with the \opt{volfirst}
preamble option, which will affect all collections. More likely,
though, it will make sense in some cases to give priority to the
volume and in others to the collection. That's why there's a
\opt{volfirst} entry option for changing particular citations. In
fact, as touched on in Section \ref{entryops}, some works need the
entry option. Here's an example from \textit{CMS} \ref{14.122}:

\begin{citebib}
\item \cite{barrows1959}
\end{citebib}

Without \opt{volfirst}, the resulting entry would place the author's
name before the collection's title, which could give the impression
that the author wrote every volume of the collection, among them the
one cited above. \textit{CMS} does suggest a format that would give
priority to the collection, placing the editor's name first and the
author's name after the volume's title. But Windy City doesn't support
it. Perhaps someday. For now, be aware that, in rare cases, the
style's default format isn't always the right or best choice.

How does Windy City determine which entries in a bibliography database
are affected by \opt{volfirst}? Below are the entries for the first
example above:

\begin{verbatim}
   @InBook{armstrong2014,
     editor = {Armstrong, Tenisha},
     title = {To Save the Soul of America, January 1961–August
              1962},
     shorttitle = {To Save the Soul of America},
     volume = {7},
     year = {2014},
     crossref = {carson1992}
   }
   @Collection{carson1992,
     editor = {Carson, Clayborne},
     title = {The Papers of Martin Luther King, Jr.},
     address = {Berkeley},
     publisher = {University of California Press},
     year = {1992–}
   }
\end{verbatim}

An \bibtype{inbook} entry, which should always be cross-referenced to
a \bibtype{collection} entry, is subject to \opt{volfirst} if it has a
\bibfield{volume} field and lacks certain fields that a volume of a
collection of this sort shouldn't have. You can find the exact details
in \file{windycity.bbx}. With one exception, you can get the same
results with a \bibtype{book} entry:

\begin{verbatim}
   @Book{carson2014,
     editor = {Armstrong, Tenisha},
     title = {To Save the Soul of America, January 1961–August
              1962},
     volume = {7},
     maintitle = {The Papers of Martin Luther King, Jr.},
     editora = {Carson, Clayborne},
     editoratype = {maintitle},
     address = {Berkeley},
     publisher = {University of California Press},
     year = {2014}
   }
\end{verbatim}

A \bibtype{book} is subject to \opt{volfirst} if it has
\bibfield{volume} and \bibfield{maintitle} fields. You may prefer a
\bibtype{book} entry if you have no interest in citing the associated
\bibtype{collection} and only intend to cite the volume. The problem
is if the volumes of the collection have different publication dates,
as they do for \textit{The Papers of Martin Luther King, Jr}. If the
publication data for the volume go last, the bibliography and long
citation should list the publication date as 2014. If the publication
data for the collection go last, that date is 1992–. A \bibtype{book}
entry, however, has just one field for a publication date, so using
\opt{volfirst} on \bibfield{carson2014} prints 2014 when the correct
date is 1992–. Otherwise, the \bibtype{book} and \bibtype{inbook}
entries listed above are interchangeable. You can find more examples
of both approaches in \file{windycity.bib}.

As mentioned in Sections \ref{preops} and \ref{entryops}, reference
lists give priority to a volume's publication data (see
\textit{CMS} \ref{15.41}). Although that may seem a bit arbitrary,
Windy City follows along by making the \opt{reflist} preamble option
imply \opt{volfirst}. Remember that when a volume and a collection
have different publication dates, a reference list entry prints both
dates, which in turn requires you to use the \bibtype{inbook} approach
described above.

\begin{refonly}
\nocite{armstrong2014}
\end{refonly}

\subsection{Collections as Single Works}
\label{multivolume}

Although its discussion is a bit obscure, \textit{CMS} treats some
multivolume collections as single works---but only, it seems, if every
volume of the collection has the same title and publication date. To
illustrate the distinction between a collection that counts as a
single work and one that doesn't, \textit{CMS} gives the following
examples in \ref{14.118}:

\begin{citeonly}
\item \cite[4:243]{byrne1981}
\item \cite*[32--33]{james1963.5}
\item \cite[4:245]{byrne1981}
\item \cite*[34]{james1963.5}
\end{citeonly}

In citations of \textit{The Lisle Letters}, volume numbers and pages
are separated by a colon. With \textit{The Complete Tales of Henry
James}, only the second citation follows this pattern. In the first,
the volume number appears earlier, after the editor's name. Why?
Apparently, \textit{The Lisle Letters} count as a single, multivolume
work because every volume has the same title and publication date. Not
so \textit{The Complete Tales of Henry James}, in which volumes have
different titles and publication dates.

To get the right output, your bibliography database and citations need
to reflect this distinction. Think of it this way: If a multivolume
collection meets the criteria of a single work (all volumes have the
same title and publication date), your bibliography database should
have just one entry to which all citations of the collection refer,
regardless of whether they cite particular volumes or the collection
as a whole. Here's the entry for \textit{The Lisle Letters}:

\begin{verbatim}
   @Collection{byrne1981,
     editor = {Byrne, Muriel St. Clare},
     title = {The Lisle Letters},
     volumes = {6},
     address = {Chicago},
     publisher = ucp,
     year = {1981}
   }
\end{verbatim}

To cite a particular volume of the collection, include the volume
number in the citation's \bibfield{postnote}. For citations of pages,
remember the format from \textit{CMS} \ref{14.118}: Volume numbers and
pages are separated with a colon. Here's the source for the first
citation of \textit{The Lisle Letters}:

\begin{verbatim}
   \cite[4:243]{byrne1981}
\end{verbatim}

\noindent To cite a volume by itself, without a page reference, or to
cite chapters, sections, and other parts of the work, remember to use
the appropriate abbreviations (for some examples, see \textit{CMS}
\ref{14.120}, 15.23, and \ref{15.41}):

\begin{verbatim}
   \cite[vol. 3, chap. 9]{byrne1981}
\end{verbatim}

What if you leave the \bibfield{postnote} empty? In that case, Windy
City assumes that you mean to cite the collection as a whole. As such,
the first, long citation of the work will print the total number of
volumes in the collection. Subsequent entries will indicate the
collection in whatever short form corresponds to the preamble options.
The following shows the default output for two such citations of the
collection:

\begin{citeonly}
\item \cite{byrne1981}
\item \cite{byrne1981}
\end{citeonly}

For collections like \textit{The Complete Tales of Henry James}, which
don't count as single works, every volume needs to have its own entry
in your bibliography database. Here's the entry for the volume cited
in \textit{CMS} \ref{14.118}:

\begin{verbatim}
   @Book{james1963.5,
     author = {James, Henry},
     maintitle = {The Complete Tales of Henry James},
     shortmaintitle = {Complete Tales of Henry James},
     editor = {Edel, Leon},
     editortype = {maintitle},
     volume = {5},
     title = {1883–1884},
     address = {London},
     publisher = {Rupert Hart-Davis},
     year = {1963}
   }
\end{verbatim}

Since the volume number is part of the entry and needs to print in
different places depending on the context, don't include it in the
\bibfield{postnote}. Let Windy City handle it. Here's the source for
the first citation of \textit{The Complete Tales Henry James}:

\begin{verbatim}
   \cite*[32--33]{james1963.5}
\end{verbatim}

Neither type of collection uses cross-referencing in the bibliography
database. For \textit{The Lisle Letters}, cross-referencing would
introduce needless complexity. A single work should have a single
entry, not multiple, cross-referenced entries. For \textit{The
Complete Tales Henry James}, cross-referencing would result in errors.
That's because, with different titles and publication dates, not all
data for the collection is true of particular volumes. To cite the
collection as a whole, as in \textit{CMS} \ref{14.117}, add a separate
entry:

\begin{verbatim}
   @Collection{james1962,
     author = {James, Henry},
     title = {The Complete Tales of Henry James},
     shorttitle = {Complete Tales of Henry James},
     editor = {Edel, Leon},
     volumes = {12},
     address = {London},
     publisher = {Rupert Hart-Davis},
     year = {1962–64}
   }
\end{verbatim}

\section{Examples from \emph{CMS} Chap. 14, ``Notes and
Bibliography''}
\label{notes}

Examples in this section reproduce those in \textit{CMS} Chapter 14.
To help with cross-checking, subsection numbers and headings are from
\textit{CMS}.

\subsection{Basic Format, with Examples and Variations}
\setcounter{subsection}{14}

\setcounter{subsubsection}{22}
\subsubsection{Notes and bibliography—examples and variations}
% 14.23 Notes and bibliography—examples and variations
\label{14.23}

\begin{citebib}
\item \cite[87-88]{strayed2012}
\item \cite[261, 265]{strayed2012}
\item \cite[32]{daum2015}
\item \cite[134--35]{daum2015}
\item \cite[188]{grazer2015}
\item \cite[190]{grazer2015}
\item \cite[242--55]{garcia1988}
\item \cite[33]{garcia1988}
\item \cite[310]{gould1984a}
\item \cite[309]{gould1984a}
\item \cite[484--85]{bagley2015}
\item \cite[501]{bagley2015}
\item \cite[311]{liu2015}
\item \cite[312]{liu2015}
\end{citebib}

\setcounter{subsection}{1}
\subsection{Notes}
\setcounter{subsection}{14}

\setcounter{subsubsection}{29}
\subsubsection{Basic structure of the short form}
% 14.30: Basic structure of the short form
\label{14.30}

\begin{citebib}
\item \cite[24--25]{morley1995}
\item \cite{schwartz1992}
\item \cite{kaiser1964}
\item \cite[43]{morley1995}
\item \cite[138]{schwartz1992}
\item \cite[189--90]{kaiser1964}
\end{citebib}

\setcounter{subsubsection}{33}
\subsubsection{Shortened citations versus ``ibid''}
% 14.34:
\label{14.34}

See Section \ref{short} for a discussion of how to enable the short
format and the use of \textit{ibid.} First, the short format:

\begin{citeonly}
\AtNextCitekey{\toggletrue{short}\toggletrue{firstshort}}
\item \cite[3]{morrison2004a}
\AtNextCitekey{\toggletrue{short}}
\item \cite[18]{morrison2004a}
\AtNextCitekey{\toggletrue{short}}
\item \cite[18]{morrison2004a}
\AtNextCitekey{\toggletrue{short}}
\item \cite[24--26]{morrison2004a}
\AtNextCitekey{\toggletrue{short}\toggletrue{firstshort}}
\item \cite[401-2]{morrison2004b}
\AtNextCitekey{\toggletrue{short}}
\item \cite[433]{morrison2004b}
\AtNextCitekey{\toggletrue{short}\toggletrue{firstshort}}
\item \cite[37--38]{diaz2008}
\AtNextCitekey{\toggletrue{short}}
\item \cite[403]{morrison2004b}
\AtNextCitekey{\toggletrue{short}}
\item \cite[152]{diaz2008}
\AtNextCitekey{\toggletrue{short}}
\item \cite[201-2]{diaz2008}
\AtNextMultiCite{\toggletrue{short}}
\item \cites[240]{morrison2004b}[32]{morrison2004a}
\AtNextCitekey{\toggletrue{short}} \item \cite[33]{morrison2004a}
\end{citeonly}

\noindent With \textit{ibid.}:

\begin{citeonly}
\AtNextCitekey{\toggletrue{short}\toggletrue{firstshort}\toggletrue{ibid}}
\item \cite[3]{morrison2004a}
\AtNextCitekey{\toggletrue{short}\toggletrue{ibid}}
\item \cite[18]{morrison2004a}
\AtNextCitekey{\toggletrue{short}\toggletrue{ibid}}
\item \cite[18]{morrison2004a}
\AtNextCitekey{\toggletrue{short}\toggletrue{ibid}}
\item \cite[24--26]{morrison2004a}
\AtNextCitekey{\toggletrue{short}\toggletrue{firstshort}\toggletrue{ibid}}
\item \cite[401-2]{morrison2004b}
\AtNextCitekey{\toggletrue{short}\toggletrue{ibid}}
\item \cite[433]{morrison2004b}
\AtNextCitekey{\toggletrue{short}\toggletrue{firstshort}\toggletrue{ibid}}
\item \cite[37--38]{diaz2008}
\AtNextCitekey{\toggletrue{short}\toggletrue{ibid}}
\item \cite[403]{morrison2004b}
\AtNextCitekey{\toggletrue{short}\toggletrue{ibid}}
\item \cite[152]{diaz2008}
\AtNextCitekey{\toggletrue{short}\toggletrue{ibid}}
\item \cite[201-2]{diaz2008}
\AtNextMultiCite{\toggletrue{short}\toggletrue{firstshort}}
\item \cites[240]{morrison2004b}[32]{morrison2004a}
\AtNextCitekey{\toggletrue{short}\toggletrue{ibid}}
\item \cite[33]{morrison2004a}
\end{citeonly}

\setcounter{subsubsection}{58}
\subsubsection{Abbreviations for frequently cited works}
% 14.59 Abbreviations for frequently cited works
\label{14.59}

In the first note below, \textit{CMS} prints the total number of
volumes. To call it a mistake is perhaps too strong, but it's
inconsistent with \ref{14.118}, which is Windy City's guiding example
for this type of citation.

You may override the default announcement of a \bibfield{shorthand} by
adding your preferred content to \bibfield{shorthandintro}. In
\file{windycity.bib}, see \bibfield{chicago2017}.

\begin{citebib}
% Better to use furet199 in 14.99, without shorthand:
%\item \cite[368]{furet1999}
\item \cite[1:126]{shurtleff1853}
\item \cite[2:330]{shurtleff1853}
\end{citebib}

\setcounter{subsection}{2}
\subsection{Author's Name}
\setcounter{subsection}{14}

\setcounter{subsubsection}{74}
\subsubsection{One author}
% 14.75 One author

\begin{citebib}
\item \cite[33]{shields2013}
\item \cite[677]{chun2015}
\item \cite[5]{mccune2014}
\item \cite[100--101]{shields2013}
\item \cite[681]{chun2015}
\item \cite[105--11]{mccune2014}
\end{citebib}

\subsubsection{Two or more authors (or editors)}
% 14.76 Two or more authors (or editors)
\label{14.76}

\begin{citebib}
\item \cite[xvi]{sorrells2015}
\item \cite[20--21]{levitt2005}
\item \cite[422]{umbers2015}
\item \cite[xx-xxi]{sorrells2015}
\item \cite[158]{gmuca2015}
\item \cite[160]{gmuca2015}
\end{citebib}

\subsubsection{Two or more authors (or editors) with same family name}
% 14.77 Two or more authors (or editors) with same family name

\begin{citebib}
\item \cite[14]{kendris2010}
\item \cite[27--28]{kendris2010}
\end{citebib}

\subsubsection{Author's name in title}
% 14.78 Author's name in title
\label{14.78}

\begin{citebib}
\item \cite*[233]{franklin1868}
\item \cite*[234]{franklin1868}
\end{citebib}

\subsubsection{No listed author (anonymous works)}
% 14.79 No listed author (anonymous works)
\label{14.79}

See Section \ref{entryops} on the \opt{anonauth} and \opt{anonauthq}
entry options.

\begin{citebib}
\item \cite{anon1610}
\item \cite{anon1547}
\item \cite{horsley1796}
\item \cite{hawkes1834}
\end{citebib}

\subsubsection{Pseudonyms}
% 14.80 Pseudonyms

To print the author's real name in brackets after a pseudonym, use the
\bibfield{nameaddon} field. If the real name is unknown, and you want
to indicate that a name is a pseudonym, put 'pseud.' in
\bibfield{nameaddon}.

\begin{citebib}
\item \cite{carre1982}
\item \cite{stendhal1925}
\end{citebib}

\setcounter{subsubsection}{82}
\subsubsection{Authors known by a given name}
% 14.83 Authors known by a given name

\begin{citebib}
\item \cite{elizabeth2000}
\end{citebib}

\subsubsection{Organization as author}
% 14.84 Organization as author
\label{14.84}

If an organization is the work's author, remember to add an extra pair
of brackets around the name of the organization in your bibliography
database.

\begin{citebib}
\AtNextCitekey{\clearfield{shorthand}}
\item \cite{chicago2017}
\item \cite{iso1997}
\end{citebib}

\setcounter{subsection}{3}
\subsection{Title of Work}
\setcounter{subsection}{14}

\setcounter{subsubsection}{88}
\subsubsection{Subtitles in cited works and the use of the colon}
% 14.89 Subtitles in cited works and the use of the colon

\begin{citebib}
\item \cite{gladwell2013}
\end{citebib}

\subsubsection{Two subtitles in a cited work}
% 14.90 Two subtitles in a cited work

\begin{citebib}
\item \cite{sereny1999}
\end{citebib}

\setcounter{subsubsection}{91}
\subsubsection{``And other stories'' and such}
% 14.92 ``And other stories'' and such

\begin{citebib}
\item \cite[104]{maclean1976}
\end{citebib}

\subsubsection{Dates in titles of cited works}
% 14.93 Dates in titles of cited works

\begin{citebib}
\item \cite{beiser2014}
\end{citebib}

\subsubsection{Quoted titles and other terms within cited titles of works}
% 14.94 Quoted titles and other terms within cited titles of works

\begin{citebib}
\item \cite{levitt2005}
\item \cite{mchugh1980}
\end{citebib}

\subsubsection{Italicized titles and other terms within cited titles of works}
% 14.95 Italicized titles and other terms within cited titles of works

\begin{citebib}
\item \cite{vanwagenen1973}
\end{citebib}

\subsubsection{Question marks or exclamation points in titles of cited works}
% 14.96 Question marks or exclamation points in titles of cited works

\begin{citebib}
\item \cite[63]{berra2002}
\item \cite[183]{oram2007}
\item \cite[778]{tessler2014}
\item \cite[336]{batson1990}
\item \cite[55--56]{berra2002}
\item \cite[184]{oram2007}
\item \cite[780]{tessler2014}
\item \cite[337]{batson1990}
\end{citebib}

\setcounter{subsubsection}{98}
\subsubsection{Translated titles of cited works}
% 14.99 Translated titles of cited works

\begin{citebib}
% Gives a comma before postnote, not a semicolon as in CMS:
%\item \cite[includes a summary in German]{wereszycki1977}
\item \cite[272]{kern1938}
\item \cite{pirumova1977b}
\item \cite{furet1999}
\end{citebib}

\setcounter{subsection}{4}
\subsection{Books}
\setcounter{subsection}{14}

\setcounter{subsubsection}{100}
\subsubsection{Form of author's name and title of book in source citations}
% 14.101 Form of author’s name and title of book in source citations

\begin{citebib}
\item \cite[79--80]{gawande2014}
\item \cite[191]{gawande2014}
\end{citebib}

\setcounter{subsubsection}{102}
\subsubsection{Editor in place of author}
% 14.103 Editor in place of author
\label{14.103}

\begin{citebib}
\item \cite[100]{egan2014}
\item \cite[33]{schechter2011}
\item \cite[34]{silverstein1974}
\item \cite[301--2]{egan2014}
\item \cite[54--56]{schechter2011}
\item \cite[38]{silverstein1974}
\end{citebib}

\subsubsection{Editor or translator in addition to author}
% 14.104 Editor or translator in addition to author
\label{14.104}

\begin{citebib}
\item \cite{bonnefoy1995}
\item \cite{menchu1999}
\item \cite{adorno1999}
\item \cite{pound1953}
\end{citebib}

\subsubsection{Other contributors listed on the title page}
% 14.105 Other contributors listed on the title page
\label{14.105}

\begin{citebib}
\item \cite{chaucer1966}
\item \cite{cullen1961}
\item \cite{hayek1994}
\item \cite{prather1998}
\item \cite{williams1990}
\end{citebib}

\subsubsection{Chapter in a single-author book}
% 14.106 Chapter in a single-author book

\begin{citebib}
\item \cite[211]{brower2015.8}
\item \cite{samples2006.7}
\item \cite[30-31]{samples2006.7}
\end{citebib}

\subsubsection{Contribution to a multiauthor book}
% 14.107 Contribution to a multiauthor book

\begin{citebib}
\item \cite[325]{miller2014}
\item \cite{ellet1968}
\end{citebib}

\subsubsection{Several contributions to the same multiauthor book}
% 14.108 Several contributions to the same multiauthor book

\begin{citebib}
\item \cite[84--87]{keating1968}
\item \cite[362--70]{lippincott1968}
\item \cite{draper1987}
\item \cite{harrington1987}
\end{citebib}

\subsubsection{Book-length work within a book}
% 14.109 Book-length work within a book

\begin{citebib}
\item \cite{bernard1990a}
\item \cite{updike1995a}
\end{citebib}

\subsubsection{Introductions, prefaces, afterwords, and the like}
% 14.110 Introductions, prefaces, afterwords, and the like

% The bibliography database entry for Toni Morrison's foreword to
% \textit{Song of Solomon} doesn't list a page range. Unlike the
% second example, which shows an introduction by authors different
% than that of the main text, it doesn't need to.

\begin{citebib}
\item \cite{morrison2004b.f}
\item \cite{mansfield2000}
\end{citebib}

\subsubsection{Letters in published collections}
% 14.111 Letters in published collections

\begin{citebib}
\item \cite[133--34]{adams1867}
\item \cite{jackson1676}
\end{citebib}

\setcounter{subsubsection}{112}
\subsubsection{Editions other than the first}
% 14.113 Editions other than the first

\begin{citebib}
\item \cite[401--2]{einsohn2011}
\item \cite[101]{boudett2013}
\item \cite{strunk2000}
\end{citebib}

\subsubsection{Reprint editions and modern editions}
% 14.114 Reprint editions and modern editions

% You can have at most one \bibfield{origdate} per entry. So, if the
% citation is to a work in a collection, say, an article or book in an
% anthology, the style assumes that \bibfield{origdate} is for the
% collection, not for the individual work.

\begin{citebib}
\item \cite[152--53]{barzun1994}
\item \cite{bahadur2014}
\item \cite{emerson1985}
\item \cite{schweitzer1966}
\end{citebib}

\subsubsection{Microform editions}
% 14.115 Microform editions

The citation of Farwell comes close to \textit{CMS} but isn't an exact
match. The problem is with the field \bibfield{howpublished}, which
seems like the best choice to contain ``microfiche'' but which, as
must happen in other cases, prints after the \bibfield{postnote}, ``p.
67, 3C12.''

\begin{citebib}
\item \cite[p. 67, 3C12]{farwell1997}
\item \cite{tauber1958}
\end{citebib}

\setcounter{subsubsection}{116}
\subsubsection{Citing a multivolume work as a whole}
% 14.117 Citing a multivolume work as a whole
\label{14.117}

\begin{citebib}
\item \cite{aristotle1983}
\item \cite{byrne1981}
\item \cite{james1962}
\end{citebib}

\subsubsection{Citing a particular volume in a note}
% 14.118 Citing a particular volume in a note
\label{14.118}

For a discussion of how to handle these types of works, see Section
\ref{multivolume}.

\begin{citebib}
\item \cite[4:243]{byrne1981}
\item \cite*[32--33]{james1963.5}
\item \cite[4:245]{byrne1981}
\item \cite*[34]{james1963.5}
\end{citebib}

\subsubsection{Citing a particular volume in a bibliography}
% 14.119 Citing a particular volume in a bibliography
\label{14.119}

\begin{citebib}
\item \cite{armstrong2014}
\end{citebib}

\noindent With preamble or entry option \opt{volfirst}:

\begin{citebib}
\AtNextCitekey{\toggletrue{volfirst}}
\item \cite{armstrong2014}
\AtNextBibliography{\toggletrue{volfirst}}
\end{citebib}

\subsubsection{Chapters and other parts of individual volumes}
% 14.120 Chapters and other parts of individual volumes
\label{14.120}

There are some peculiarities with this example. In the book, but not
online, \textit{CMS} errs in printing `.ed' rather than `editeb by' in
the bibliography and neglects to invert the author's name. More
worrisome are the striking differences between the note and
bibliography. They may represent alternative ways of formatting the
data, as other examples do. But the note seems inconsistent with
\textit{CMS} \ref{14.118}, and so doesn't make much sense as an
alternative. Windy City ignores it and in both cases follows the
example of the bibliography.

\begin{citebib}
\item \cite[180]{chen2010.3}
\end{citebib}

\subsubsection{One volume in two or more books}
% 14.121 One volume in two or more books
\label{14.121}

\begin{citebib}
\item \cite[351]{lach1977}
\item \cite{harley1994}
\end{citebib}

\noindent With preamble or entry option \opt{volfirst}:

\begin{citebib}
\AtNextCitekey{\toggletrue{volfirst}}
\item \cite[351]{lach1977}
\AtNextCitekey{\toggletrue{volfirst}}
\item \cite{harley1994}
\AtNextBibliography{\toggletrue{volfirst}}
\end{citebib}

\subsubsection{Authors and editors of multivolume works}
% 14.122 Authors and editors of multivolume works
\label{14.122}

The work by Barrows is an example of one that seems to need the entry
option \opt{volfirst} (see Section \ref{collorder}). Without it, the
resulting entry would give the impression that he wrote every volume
in the collection. With Donne, on the other hand, you may choose one
format or the other.

\begin{citebib}
\item \cite{barrows1959}
\item \cite*{donne1995}
\end{citebib}

\noindent

Citing Donne with the preamble or entry option \opt{volfirst}:

\begin{citebib}
\AtNextCitekey{\toggletrue{volfirst}}
\item \cite{donne1995}
\AtNextBibliography{\toggletrue{volfirst}}
\end{citebib}

\subsubsection{Series titles, numbers, and editors}
% 14.123 Series titles, numbers, and editors
\label{14.123}

\begin{citebib}
\item \cite{lei2014}
\item \cite{mazrim2011}
\item \cite{wauchope1950}
\item \cite{allen2009}
\end{citebib}

\subsubsection{Series or multivolume work?}
% 14.124 Series or multivolume work?

\begin{citebib}
\item \cite{boyer1986}
\item \cite{cochrane1987}
\end{citebib}

\setcounter{subsubsection}{125}
\subsubsection{``Old series'' and ``new series''}
% 14.126 ``Old series'' and ``new series''
\label{14.126}

\begin{citebib}
\item \cite{boxer1953}
\item \cite{palmatary1950}
\end{citebib}

\subsubsection{Place, publisher, and date}
% 14.127 Place, publisher, and date

\begin{citebib}
\item \cite{woolf1927}
\end{citebib}

\subsubsection{Place and date only, for books published before 1900}
% 14.128 Place and date only, for books published before 1900

\begin{citebib}
\item \cite{goldsmith1766}
\item \cite{cervantes1605}
\end{citebib}

\setcounter{subsubsection}{131}
\subsubsection{No place of publication}
% 14.132 No place of publication

\begin{citebib}
\item \cite{windsor1910}
\item \cite{vliet1890}
\end{citebib}

\setcounter{subsubsection}{136}
\subsubsection{Self-published or privately published books}
% 14.137 Self-published or privately published books

\begin{citebib}
\item \cite{karavaev2015}
\item \cite{shumaker2014}
\end{citebib}

\setcounter{subsubsection}{139}
\subsubsection{Copublication}
% 14.140 Copublication

\begin{citebib}
\item \cite{strauss1962}
\end{citebib}

\subsubsection{Distributed books}
% 14.141 Distributed books

\begin{citebib}
\item \cite{willke2007}
\end{citebib}

\setcounter{subsubsection}{143}
\subsubsection{Multivolume works published over more than one year}
% 14.144 Multivolume works published over more than one year
\label{14.144}

\begin{citebib}
\item \cite*[329]{hayek2011}
\item \cite{tillich1951}
\end{citebib}

\noindent Citing Hayek with the preamble or entry option
\opt{volfirst}:

\begin{citebib}
\AtNextCitekey{\toggletrue{volfirst}}
\item \cite[329]{hayek2011}
\AtNextBibliography{\toggletrue{volfirst}}
\end{citebib}

\subsubsection{No date of publication}
% 14.145 No date of publication

\begin{citebib}
\item \cite{boston}
\item \cite{edinburgh1750}
\item \cite{edinburgh}
\end{citebib}

\subsubsection{Forthcoming publications}
% 14.146 Forthcoming publications

\begin{citebib}
\item \cite{author}
\item \cite[345--46]{writer}
\item \cite{contributor}
\end{citebib}

\setcounter{subsubsection}{158}
\subsubsection{Books requiring a specific application or device (e-books)}
% 14.159 Books requiring a specific application or device (e-books)

\begin{citebib}
\item \cite{borel2015}
\end{citebib}

\setcounter{subsubsection}{160}
\subsubsection{Books consulted online}
% 14.161 Books consulted online

The first and third notes below present a challenge: If a bibliography
database entry contains an address for a work, such as a DOI, Windy
City prints it in the work's first, long citation. Such is the case
with the second note below. To cite an address for just part of a
work, but print one for the whole work in the bibliography, you need
to override the style's default behavior. The first and third notes do
this with a command that temporarily clears the work's DOI from its
bibliography database entry. Here's an example from the source:

\begin{verbatim}
   \AtNextCitekey{\clearfield{doi}}
   \item \cite[chap. 3, \url{https://doi.org/10.1093/acprof:
               oso/9780199343638.003.0004}]{bonds2014}
\end{verbatim}

\begin{citebib}
\AtNextCitekey{\clearfield{doi}}
\item \cite[chap. 3, \url{https://doi.org/10.1093/acprof:oso/9780199343638.003.0004}]{bonds2014}
\item \cite[59]{lystra2004}
\AtNextCitekey{\clearfield{doi}}
\item \cite[chap. 11, \url{https://doi.org/10.1093/acprof:oso/9780199343638.003.0012}]{bonds2014}
\item \cite[60--61]{lystra2004}
\end{citebib}

\subsubsection{Freely available electronic editions of older works}
% 14.162 Freely available electronic editions of older works

\begin{citebib}
\item \cite{james2008}
\end{citebib}

\subsubsection{Books on CD-ROM and other fixed media}
% 14.163 Books on CD-ROM and other fixed media

\begin{citebib}
\item \cite[1.4]{chicago2003}
\end{citebib}

\setcounter{subsection}{5}
\subsection{Periodicals}
\setcounter{subsection}{14}

\setcounter{subsubsection}{170}
\subsubsection{Journal volume, issue, and date}
% 14.171 Journal volume, issue, and date

The note for Harper includes the month of publication. Windy City
includes it as well in the bibliography, even though \textit{CMS}
omits it. As for Lock's entry in the bibliography, \textit{CMS}
clearly errs in printing the surname twice.

\begin{citebib}
\item \cite[155]{lock2015}
\item \cite[651]{wesoky2015}
\item \cite[645]{harper2014}
\item \cite[60]{wilder2013}
\item \cite[52]{beattie1974}
\end{citebib}

\subsubsection{Forthcoming journal articles}
% 14.172 Forthcoming journal articles

\begin{citebib}
\item \cite{authora}
\end{citebib}

\setcounter{subsubsection}{173}
\subsubsection{Journal page references}
% 14.174 Journal page references

\begin{citebib}
\item \cite{gold2015}
\item \cite[2--3]{paudyal2015}
\end{citebib}

\subsubsection{Journal articles consulted online}
% 14.175 Journal articles consulted online

\begin{citebib}
\item \cite[268]{whitney1929}
\item \cite[260--61]{schoenfield2016}
\end{citebib}

\subsubsection{Access dates for journal articles}
% 14.176 Access dates for journal articles

\begin{citebib}
\item \cite[81]{narr2015}
\item \cite[88--89]{narr2015}
\end{citebib}

\setcounter{subsubsection}{177}
\subsubsection{Journal special issues}
% 14.178 Journal special issues

\begin{citebib}
\item \cite[351--81]{tezuka2013}
\end{citebib}

\setcounter{subsubsection}{179}
\subsubsection{Articles published in installments}
% 14.180 Articles published in installments

\begin{citebib}
\item \cite[312]{brown1978}
\end{citebib}

\setcounter{subsubsection}{181}
\subsubsection{Place where journal is published}
% 14.182 Place where journal is published

\begin{citebib}
\item \cite[65--70]{luu1999}
\item \cite{garrett1975}
\end{citebib}

\subsubsection{Translated or edited article}
% 14.183 Translated or edited article

\begin{citebib}
\item \cite{authorb}
\item \cite{authorc}
\end{citebib}

\subsubsection{New series for journal volumes}
% 14.184 New series for journal volumes
\label{14.184}

\begin{citebib}
\item \cite[414]{sewall1896}
\item \cite{moraes1950}
\end{citebib}

\subsubsection{Short titles for articles}
% 14.185 Short titles for articles

\begin{citebib}
\item \cite[223]{rosenblum2015}
\item \cite[225]{rosenblum2015}
\end{citebib}

\subsubsection{Abstracts}
% 14.186 Abstracts

\begin{citebib}
\item \cite{matute2015}
\end{citebib}

\setcounter{subsubsection}{187}
\subsubsection{Basic citation format for magazine articles}
% 14.188 Basic citation format for magazine articles

\begin{citebib}
\item \cite[48]{saulnier2008}
\item \cite[59]{lepore2015}
\end{citebib}

\subsubsection{Magazine articles consulted online}
% 14.189 Magazine articles consulted online

\begin{citebib}
\item \cite{vick2015}
\item \cite[5]{hanemann1926}
\end{citebib}

\subsubsection{Magazine departments}
% 14.190 Magazine departments

\begin{citebib}
\item \cite{marx2015}
\item \cite{wallraff2008}
\item \cite{gourmet2000}
\end{citebib}

\subsubsection{Basic citation format for newspaper articles}
% 14.191 Basic citation format for newspaper articles

\begin{citebib}
\item \cite{editorial1990}
\item \cite{royko1992}
\item \cite{forester2000}
\item \cite{samenow2016}
\end{citebib}

\setcounter{subsubsection}{194}
\subsubsection{Regular columns or features}
% 14.195 Regular columns or features

\begin{citebib}
\item \cite{jaffe2015}
\end{citebib}

\setcounter{subsubsection}{196}
\subsubsection{Weekend supplements, magazines, and the like}
% 14.197 Weekend supplements, magazines, and the like

\begin{citebib}
\item \cite[48]{ghansah2015}
\end{citebib}

\setcounter{subsubsection}{198}
\subsubsection{Unsigned newspaper articles}
% 14.199 Unsigned newspaper articles

\begin{citebib}
\item \cite{nytimes2002}
\end{citebib}

\subsubsection{News services and news releases}
% 14.200 News services and news releases

\begin{citebib}
\item \cite{ap2015}
\end{citebib}

\setcounter{subsubsection}{201}
\subsubsection{Book reviews}
% 14.202 Book reviews

\begin{citebib}
\item \cite[B13--B14]{ratliff1999}
\item \cite{brehm2015}
\end{citebib}

\setcounter{subsubsection}{203}
\subsubsection{Unsigned reviews}
% 14.204 Unsigned reviews

This example fails to match \textit{CMS}. The date should appear
immediately after the newspaper's title and not in parentheses.
Presumably, \textit{CMS} puts the date first because the date is a
more important part of a magazine's or newspaper's citation.
Nevertheless, the format below is consistent with many other types of
articles.

For reference lists, unsigned reviews have no plausible place for the
publication year but where the default format would put it. The
easiest solution is to give unsigned reviews the same format in
reference lists as in bibliographies.

\begin{citebib}
\item \cite{zeitung1828}
\end{citebib}

\setcounter{subsection}{6}
\subsection{Papers, Contracts, and Reports}
\setcounter{subsection}{14}

\setcounter{subsubsection}{214}
\subsubsection{Theses and dissertations}
% 14.215 Theses and dissertations

\begin{citebib}
\item \cite[59]{vedrashko2006}
\item \cite{choi2008}
\end{citebib}

\setcounter{subsubsection}{216}
\subsubsection{Lectures and papers or posters presented at meetings}
% 14.217 Lectures and papers or posters presented at meetings

\begin{citebib}
\item \cite{hong2015}
\end{citebib}

\subsubsection{Working papers and the like}
% 14.218 Working papers and the like

\begin{citebib}
\item \cite{lucki1980}
\end{citebib}

\setcounter{subsubsection}{219}
\subsubsection{Pamphlets, reports, and the like}
% 14.220 Pamphlets, reports, and the like

\begin{citebib}
\item \cite{lifestyles1996}
% FIX: Printing date:
%\item \cite{mcdonalds2014}
\item \cite[¶2,620]{standardtax1996}
\end{citebib}

\setcounter{subsection}{7}
\subsection{Special Types of References}
\setcounter{subsection}{14}

\setcounter{subsubsection}{231}
\subsubsection{Reference works consulted in physical formats}
% 14.232 Reference works consulted in physical formats
\label{14.232}

Some reference works show full publication information in the same way
as books. Use the \bibtype{book} entry type for them. The first three
citations below are different. They need the \bibtype{reference} or
\bibtype{inreference} entry type. See Section \ref{entrytypes} for
more information. Following the suggestion in \textit{CMS} 14.232,
\bibtype{reference} and \bibtype{inreference} works don't appear in
bibliographies and reference lists.

\begin{citebib}
\item \cite{salvation1980}
\item \cite{hootananny2009}
\item \cite{dab1937}
\item \cite[s.vv. \mkbibquote{police ranks}, \mkbibquote{postal addresses}]{timestyle2003}
\item \cite[6.8.2]{mla2008}
\end{citebib}

\subsubsection{Reference works consulted online}
% 14.233 Reference works consulted online

Like some of the reference works in the previous section, the ones
below need the \bibtype{reference} or \bibtype{inreference} entry
type. As odd as it may seem, but consistent with \textit{CMS}, they,
too, aren't included in bibliographies and reference lists. See
Section \ref{entrytypes} for more information.

\begin{citeonly}
\item \cite{toscanini2016}
\item \cite{cairns2016}
\item \cite{wikipedia2016}
\item \cite{merriam2016}
\end{citeonly}

\subsubsection{Citing individual reference entries by author}
% 14.234 Citing individual reference entries by author

Reference works like the following have the same format as articles in
collections. Otherwise, Windy City formats them incorrectly.

\begin{citebib}
\item \cite{isaacson2005}
\end{citebib}

\setcounter{subsubsection}{245}
\subsubsection{Citing specific editions of classical references}
% 14.246 Citing specific editions of classical references

\begin{citebib}
\item \cite{epictetus1916}
\end{citebib}

\setcounter{subsubsection}{250}
\subsubsection{Modern editions of the classics}
% 14.251 Modern editions of the classics

\begin{citebib}
\item \cite{aristotle1983}
\item \cite{maimonides1965}
\end{citebib}

\setcounter{subsubsection}{257}
\subsubsection{Patents}
% 14.258 Patents

\begin{citebib}
\item \cite{iizuka1986}
\end{citebib}

\setcounter{subsubsection}{259}
\subsubsection{Citations taken from secondary sources}
% 14.260 Citations taken from secondary sources

\begin{citebib}
\item \cite[269]{zukofsky1931}, quoted in \cite[78]{costello1981}
\end{citebib}

\section{Examples from \emph{CMS} Chap. 15, ``Author-Date
References''}
\label{paren}

Examples in this section reproduce those in \textit{CMS} Chapter 15.
To help with cross-checking, subsection numbers and headings are from
\textit{CMS}. Since parenthetical citations are relatively simple, and
since the format of references lists is derivative of the default,
the examples below are more selective than those in the previous
section.

\subsection{Basic Format, with Examples and Variations}
\setcounter{subsection}{15}

\setcounter{subsubsection}{8}
\subsubsection{Author-date references—examples and variations}
% 15.9 Author-date references—examples and variations

\begin{citeref}
\item \parencite[87--88]{strayed2012}
\item \parencite[32]{daum2015}
\item \parencite[188]{grazer2015}
\item \parencite[242--55]{garcia1988}
\item \parencite[310]{gould1984a}
\item \parencite[484--85]{bagley2015}
\item \parencite[312]{liu2015}
\end{citeref}

\setcounter{subsection}{1}
\subsection{Reference Lists and Text Citations}
\setcounter{subsection}{15}

\setcounter{subsubsection}{13}
\subsubsection{Placement of dates in reference list entries}
% 15.14 Placement of dates in reference list entries

\begin{citeref}
\item \parencite{pager2015}
\item \parencite{unger2014}
\end{citeref}

\setcounter{subsubsection}{19}
\subsubsection{Reference list entries with same author(s), same year}
% 15.20 Reference list entries with same author(s), same year

\begin{citeref}
\item \parencite[218]{fogel2004b}
\item \parencite[45--46]{fogel2004a}
\end{citeref}

\setcounter{subsubsection}{21}
\subsubsection{Text citations---basic form}
% 15.22 Text citations—basic form

Ignore the error in \textit{CMS}: In reference lists, a title goes
after the year, not before.

\begin{citeref}
\item \parencite{hetherington2015,grove2015}
\end{citeref}

\begin{citeref}
\item \parencite{doershuk2017}
\item \parencite{doershuk2016}
\end{citeref}

\setcounter{subsubsection}{26}
\subsubsection{Several references to the same source}
% 15.27 Several references to the same source

\begin{citeref}% 'wood\-grained' prevents wood--[break]grained
\item Complexion figures prominently in Morgan's descriptions. When
Jasper compliments his mother's choice of car (a twelve-cylinder
Mediterranean roadster with leather and wood\-grained interior), ``his
cheeks blotch indignantly, painted by jealousy and rage''
\parencite[47]{chaston2000}. On the other hand, his mother's mask
never changes, her ``even-tanned good looks''
\parencite[56]{chaston2000}, ``burnished visage''
\parencite[101]{chaston2000}, and ``air-brushed confidence''
\parencite[211]{chaston2000} providing the foil to the drama in her
midst.
\end{citeref}

\setcounter{subsubsection}{28}
\subsubsection{Text citations of works with more than three authors}
% 15.29 Text citations of works with more than three authors

\begin{citeref}
\item \parencite{schonen2017a}
\item \parencite{schonen2017b}
\end{citeref}

\setcounter{subsection}{2}
\subsection{Author-Date References: Special Cases}
\setcounter{subsection}{15}

\setcounter{subsubsection}{33}
\subsubsection{Author-date format for anonymous works (no listed author)}
% 15.34 Author-date format for anonymous works (no listed author)

See Section \ref{entryops} on the \opt{anonauth} and \opt{anonauthq}
entry options.

\begin{citeref}
\item \parencite{anon1610}
\item \parencite{anon1547}
\item \parencite{horsley1796}
\item \parencite{hawkes1834}
\end{citeref}

\subsubsection{Pseudonyms in author-date references}
% 15.35 Pseudonyms in author-date references

\begin{citeref}
\item \parencite{stendhal1925}
\end{citeref}

\subsubsection{Editor in place of author in text citations}
% 15.36 Editor in place of author in text citations

\begin{citeref}
\item \parencite{silverstein1974}
\item \parencite{soltes1999}
\end{citeref}

\subsubsection{Organization as author in author-date references}
% 15.37 Organization as author in author-date references

In the reference list, \textit{CMS} errs in printing `:1997' after
`ISO 4'. Compare it with the nearly identical example in \ref{14.84}.

\begin{citeref}
\item \parencite{iso1997.ref}
\end{citeref}

\setcounter{subsubsection}{39}
\subsubsection{Reprint editions and modern editions—more than one date}
% 15.40 Reprint editions and modern editions—more than one date

\begin{citeref}
\item \parencite{austen2003}
\item \parencite{maitland1998}
\end{citeref}

\subsubsection{Multivolume works published over more than one year}
% 15.41 Multivolume works published over more than one year
\label{15.41}

\begin{citeref}
\item \parencite[1:133]{tillich1951}
%\item \parencite[vol. 2]{tillich1951}
\item \parencite[329]{hayek2011}
\end{citeref}

\subsubsection{Cross-references to multiauthor books in reference lists}
% 15.42 Cross-references to multiauthor books in reference lists

\begin{citeref}
\item \parencite{draper1987}
\item \parencite{harrington1987}
\item \parencite{zukowsky1987}
\end{citeref}

\setcounter{subsubsection}{43}
\subsubsection{No date of publication in author-date references}
% 15.44 No date of publication in author-date references

\begin{citeref}
\item \parencite{nano1750}
\item \parencite{nano}
\end{citeref}

\subsubsection{``Forthcoming'' in author-date references}
% 15.45 ``Forthcoming'' in author-date references

\begin{citeref}
\item \parencite{faraday}
\end{citeref}

\setcounter{subsubsection}{46}
\subsubsection{Parentheses or comma with issue number}
% 15.47 Parentheses or comma with issue number

\begin{citeref}
\item \parencite{glass2014}
\item \parencite{meyerovitch1959}
\end{citeref}

\subsubsection{Colon with volume number}
% 15.48 Colon with volume number

\begin{citeref}
\item \parencite{gunderson2015}
\end{citeref}

\subsubsection{Newspapers and magazines in reference lists}
% 15.49 Newspapers and magazines in reference lists

\begin{citeref}
\item \parencite{nytimes2002}
\end{citeref}

\setcounter{subsubsection}{54}
\subsubsection{Patents or other documents cited by more than one date}
% 15.55 Patents or other documents cited by more than one date

\begin{citeref}
\item \parencite{iizuka1986}
\end{citeref}

\subsubsection{``Quoted in'' in author-date references}
% 15.56 “Quoted in” in author-date references

\begin{citeref}
\item In Louis Zukofsky’s ``Sincerity and Objectification,''
from the February 1931 issue of \textit{Poetry} magazine
\parencite[quoted in][]{costello1981}\ldots
\end{citeref}

\defbibnote{sh}{This section shows the output of
\cmd{printshorthands}. By default, works from this list also appear in
bibliographies and reference lists. To exclude them, use the preamble
option \opt{nolos} (see Section \ref{preops}). Note that the style
does not automatically italicize a \bibfield{shorthand}. Whether you
should italicize it depends on whether you should italicize the work's
title (14.60).\\}%

\defbibnote{bib}{This section shows the default output of
\cmd{printbibliography}. In the next section, the same works appear in
the author-date format.\\}%

\defbibnote{ref}{This section shows the output of
\cmd{printbibliography} for a reference list (see \opt{reflist} in
Section \ref{preops}). The works below are the same as those in the
previous section but in the author-date format.\\}%

\printshorthands[prenote=sh]
\refstepcounter{sh}\label{sh}
\printbibliography[notkeyword=notinbib,prenote=bib]
\refstepcounter{bib}\label{bib}
\printbibliography[%
  env=reflist,
  heading=references,
  notkeyword=notinref,
  prenote=ref]
\refstepcounter{ref}\label{ref}
\end{document}
