% Last modified: Fri 23 Nov 2018 03:24:23 PM CST
\documentclass[11pt,letterpaper,oneside]{article}
\usepackage{windycity}

\begin{document}
\title{Windy City}
\subtitle{A Chicago Style for \biblatex}
\author{Brian Michael Chase}
\email{brianmichaelchase@gmail.com}
\website{https://github.com/brianchase/windycity}
\version{2018.11.23}
\maketitle
\tableofcontents\markboth{Contents}{Contents}
\newpage

\section{Introduction}
\label{intro}

\nfootnote{Copyright \textcopyright\ 2018 Brian Michael Chase. Under
the terms of the \LaTeX\ Project Public License, version 1.3,
permission is granted to copy, distribute, or modify this software.
See \url{http://www.ctan.org/tex-archive/macros/latex/base/lppl.txt}.}

Windy City is a style for \biblatex that formats notes,
bibliographies, parenthetical citations, and reference lists according
to the \textit{The Chicago Manual of Style}
(\textit{CMS}).\footnote{\cite{chicago2010} Support for the 17th
edition is in development.} It accurately handles a wide range of
citations and includes a set of options and commands to accommodate
special circumstances. It also has extensive support for citing and
arranging different kinds of editors, translators, and compilers
within a single citation. These features make Windy City especially
suitable for academic work.

The following sections assume familiarity with \textit{CMS} and
\biblatex. Section \ref{basics} gives a brief overview of the style's
features. Section \ref{edtrans} discusses the assignment of editors,
translators, and compilers. Sections \ref{notes16} and \ref{paren16}
reproduce examples from various parts of \textit{CMS} with a focus on
notes and bibliographies. It will help to compare the results
throughout these sections with the corresponding information in this
document's bibliography database and \LaTeX\ file,
\file{windycity.bib} and \file{windycity.tex}, respectively.

Aside from \biber, which is necessary for a few options, Windy City
has no requirements beyond those of \biblatex version 3.8 or later.

\section{Some Basics}
\label{basics}

Consider a few footnotes: One.\footcite{beattie1974}
Two.\footcite[51]{beattie1974} Three.\footcite[51]{beattie1974}
Four.\footcite[35]{shields2008} And five.\footcite[51]{beattie1974}

By default, the first citation of a work takes a long form, similar to
its entry in the bibliography. If consecutive citations of a work
appear on the same page, citations after the first receive an
\textit{ibid}.\footnote{In line with the 17th edition of \textit{CMS},
which discourages the use of \textit{ibid}., Windy City will
eventually drop it from the standard format but allow it as an option
(see \textit{CMS}, 17th edition, 14.34).\nocite{chicago2017} To
disable it in the current version, see Section \ref{preops}.}
Otherwise, subsequent citations of a work usually list short forms of
the author's name and the work's title.

The short form of a name usually consists of a surname. If the name of
an editor, translator, or compiler occupies the author's position, the
short form of the name omits abbreviations of editorial roles, such as
\textit{ed.} and \textit{trans.}, which appear only in first
citations.\footnote{See \textit{CMS} 14.18, 14.27, and 14.76.} If
necessary to distinguish names, the short form will include first and
middle initials.

Several factors determine the short form of a title. If a work's entry
in the bibliography database includes a \bibfield{shorttitle} or
\bibfield{shortbooktitle} (see Section \ref{datafields}), the style
will print one of them, giving preference to \bibfield{shorttitle}. If
neither field is available, the short form is the work's title minus
its subtitle.

If a work uses an abbreviation---what \biblatex calls a
\bibfield{shorthand}---it replaces \textit{ibid}, the author's name,
and related elements in subsequent citations. Citations like
this are scattered throughout these pages in reference to \textit{The
Chicago Manual of Style}, which here takes the abbreviation
\textit{CMS}. By default, the first citation of a work with a
\bibfield{shorthand} announces it in parentheses at the end of the
citation, as in the following:

\begin{itemize}
\item[] \fullcite[1:126]{shurtleff1853}
\end{itemize}

\noindent You may override the default announcement, and format it
however you like, by adding your preferred content to
\bibfield{shorthandintro}.

The next note shows another feature of the
style.\footcite{camus1991a,camus1991b} It contains consecutive, first
citations of works by the same author. The style prints the author's
full name in the first citation and the short form in the second. If
these citations were not consecutive, both citations would include the
full name; only subsequent citations would use the short form.

With certain collections, still other results are possible (see
Section \ref{collect}). For examples of how the style truncates lists
of names, see Section \ref{notes16}---in particular, the examples from
\textit{CMS} \ref{14.76}. For options to skip long first citations,
drop parts of citations, change the order of editors and translators,
and more, see Sections \ref{preops} and \ref{entryops}.

In addition to notes, the style supports parenthetical citations and
reference lists in the author-date format of \textit{CMS}. You may use
parenthetical citations in conjunction with notes. But you most choose
between the standard format of a bibliography, which is the style's
default, and that of a reference list (see \opt{reflist} in Section
\ref{preops}). For examples of parenthetical citations and reference
list entries, see Section \ref{paren16} and, at the end of this
document, \ref{ref}. The latter shows in the author-date format the
same sources as the \ref{bib}.

\subsection{Preamble Options}
\label{preops}

A preamble option is an argument for the \cmd{usepackage} macro that
loads \biblatex. Preamble options affect the format of notes,
bibliographies, and reference lists. Some features of the style
require them.

\begin{optionlist}

\optitem[false]{annotate}{\opt{true}, \opt{false}}

Use this option to print an annotated bibliography. Annotations will
appear in block paragraphs below their associated entries. To change
the spacing between entries and annotations, change the value of
\cmd{bibitemsep}.

\optitem[false]{collsonly}{\opt{true}, \opt{false}}

By default, citing individual works of a collection adds an entry for
each work to the bibliography.\footnote{The style's defaults for
\opt{mincrossrefs} and \opt{minxrefs} is \liningnums{2}. This means
that the style requires at least two cross-references to a collection
before adding it to the bibliography.} This could result in needless
clutter, especially after citing many volumes of a multivolume work.
To exclude individual works and print only an entry for the collection
as a whole, use \opt{collsonly}. This option sets \opt{mincrossrefs}
and \opt{minxrefs} to \liningnums{1} and then filters from the
bibliography all \bibtype{inbook} entries that cross-reference other
entries and all \bibtype{incollection} entries that cite chapters of a
cross-referenced book. Thus, it has no effect on many
\bibtype{incollection} entries, such as articles in books, which need
their own place in the bibliography. For some discussion of
collections and multivolume works, see Section \ref{collect}.

\optitem[false]{firstshort}{\opt{true}, \opt{false}}

Use this option to print a shortened form of the first citation of
each work. The resulting citation consists mainly of the author's name
and the title. According to \textit{CMS}, this approach is optional
for documents that have a complete
bibliography.\footnote{\cite[14.14]{chicago2010} See also
14.24--14.28. By default, the style adds every cited work to the
bibliography.}

\optitem[true]{ibid}{\opt{true}, \opt{false}}

This option controls whether consecutive citations to a work on the
same page receive an \textit{ibid.}. Unlike the other entry options
discussed here, the style sets this one to \textit{true} by default.
Nevertheless, since the 17th edition of \textit{CMS} discourages the
use of \textit{ibid.} (see 14.34), a subsequent release of the style
will set it to \textit{false}.

\optitem[false]{isbn}{\opt{true}, \opt{false}}

Use this option to print ISBNs in bibliographies. A work's ISBN
belongs in the \bibfield{isbn} field of its database entry. With this
option, the style will print ISBNs at the end of every entry in the
bibliography, though before annotations. To print the ISBN of a
particular work, see the \bibfield{isbn} entry option in Section
\ref{entryops}.

\optitem[false]{nolos}{\opt{true}, \opt{false}}

By default, every work with a \bibfield{shorthand} receives an entry
in the bibliography. If you wish to exclude them, say, to avoid
duplication with the output of \cmd{printshorthands}, use \opt{nolos}.

\optitem[false]{reflist}{\opt{true}, \opt{false}}

Use this option to print what \textit{CMS} calls a reference list, a
bibliography in the author-date format. If you choose parenthetical
citations over notes, you should consider using \opt{reflist} to
maintain consistency with \textit{CMS}. You can accomplish the same
effect by giving the option \opt{env=reflist} to
\cmd{printbibliography}. Thus, if you have more than one bibliography
in a document, as in this one, you may print them in different
formats.

\end{optionlist}

\subsection{Entry Options}
\label{entryops}

An entry option is a value for the \bibfield{options} field of a
work's database entry. It affects the format of that particular work.
For options that affect the format of every work, see the previous
section.

\begin{optionlist}

\optitem[false]{anonauth}{\opt{true}, \opt{false}}

This option prints the author's name of an anonymously published work
in brackets, as shown in \textit{CMS} \ref{14.80}:

\begin{itemize}
\item[N] \fullcite{horsley1796}

\item[B] \bibentry{horsley1796}
\end{itemize}

\optitem[false]{anonqauth}{\opt{true}, \opt{false}}

This option is similar to the previous but adds a question mark after
the author's name, indicating doubt about the authorship. The
following example also comes from \ref{14.80}:

\begin{itemize}
\item[N] \fullcite{cook1730}

\item[B] \bibentry{cook1730}
\end{itemize}

\optitem[false]{isbn}{\opt{true}, \opt{false}}

Use this option to print the ISBN of a particular work in
bibliographies. The ISBN will appear at the end of the work's entry
but before its annotation. To print ISBNs of every work, see the
\bibfield{isbn} preamble option in the previous section.

\optitem[false]{noauth}{\opt{true}, \opt{false}}

This option tells the style to bypass the author's position of a work,
both in notes and in bibliographies. Citations will begin with the
title's position. This option is rarely necessary. But consider an
example from \textit{CMS} 14.89:

\begin{itemize}
\item[N] \fullcite{crow1966}

\item[B] \bibentry{crow1966}
\end{itemize}

\noindent Without \opt{noauth}, the style prints:

% I use \AtNextCitekey to set 'noauth' to false to avoid having two
% nearly identical entries in windycity.bib for crow1966. Having those
% entries caused the bibliography for the author-date format to print
% the entry for crow1966 as 1966b. I don't use \AtNextCitekey for
% similar examples below.

\begin{itemize}
\AtNextCitekey{\togglefalse{noauth}}
\item[N] \fullcite{crow1966}

\AtNextCitekey{\togglefalse{noauth}}
\item[B] \bibentry{crow1966}
\end{itemize}

\noindent If you choose the former, remember that unless you adjust
how the work is sorted in the bibliography by, for example, setting
\bibfield{sortname} to \textit{Chaucer}, the style will sort it by the
first editor's surname, \textit{Crow}. See also \cmd{cite*} in Section
\ref{citecmds}.

\optitem[false]{transfirst}{\opt{true}, \opt{false}}

According to \textit{CMS}, if a work has both an editor and a
translator, their names should appear in citations in the order in
which they appear on the work's title page (14.88). By default, the
style lists editors first. However, entries with the option
\bibfield{transfirst} reverse this order: Their translators print
first. If a work's translators and editors are identical, using
\bibfield{transfirst} reverses the order of their roles---for example,
from \textit{edited and translated by} to \textit{translated and
edited by}. For more on this option, and examples, see Section
\ref{edtranspos}.

\end{optionlist}

\subsection{Citation Commands}
\label{citecmds}

The most important citation commands are already familiar from
\biblatex:

\begin{ltxsyntax}

\cmditem{cite}[prenote][postnote]{key}
\cmditem{footcite}[prenote][postnote]{key}
\cmditem{nocite}{key}
\cmditem*{nocite}|\{*\}|
\cmditem{parencite}[prenote][postnote]{key}

\end{ltxsyntax}

\noindent Insert notes with \cmd{cite} and \cmd{footcite}. Insert
parenthetical citations with \cmd{parencite}. Use \cmd{nocite} to add
works to bibliographies without citing them in the text. Use
\cmd{nocite} with a comma-separated list of entry keys to add
particular works. Use it with an asterisk to add every work in every
\file{bib} file listed in \cmd{bibliography}.

\begin{ltxsyntax}

\cmditem{bibentry}{key}

This command prints the bibliography entry of a work. Although mainly
for testing, you might use it for printing lists of works, as for a
syllabus. It has no effect on citation tracking.

\cmditem{cite*}[prenote][postnote]{key}

Use this command to cite a work without printing anything in the
author's position of a note. This is most likely useful when the
context makes clear who the author is. Consider this example from
\ref{14.78}:

\begin{itemize}
\item[N] \fullcite*[44--45]{mccullers1999}

\item[B] \bibentry{mccullers1999}
\end{itemize}

\noindent Ordinarily, the style would print the author's name in the
note, but \textit{CMS} allows you to drop it. You may do so with
\cmd{cite*}.\footnote{Technically, the example uses \cmd{fullcite*},
since there is no need here for citation tracking.}

\cmditem{citerp}[postnote]{key}

This command precedes a citation with ``Reprinted in'' and then
bypasses the author's position and title. This is useful for citing
reprints, especially of articles, after citing the original. For
example, the first citation below uses \cmd{cite}, the second uses
\cmd{citerp}:

\begin{itemize}
\item[] \cite{frankfurt1969} \citerp[1--10]{frankfurt1988.1}
\end{itemize}

\cmditem{footfullcite}[prenote][postnote]{key}

A slight modification of \cmd{fullcite} (see below), this command
prints the first, complete citation of a work in a footnote. Like
\cmd{fullcite}, it does not include the work in citation tracking.
Thus, more so than \cmd{fullcite}, it is mainly for testing.

\cmditem{fullcite}[prenote][postnote]{key}

This command prints the first, complete citation of a work in the
format of a note. Unlike \cmd{footfullcite}, which prints the same
content in a footnote, \cmd{fullcite} prints citations in place. This
command is mainly useful for testing, though, like \cmd{bibentry}, you
may use it for printing lists of works. Since it has no effect on
citation tracking, and ignores the preamble option
\bibfield{firstshort} (see Section \ref{preops}), using it will never
result in an \textit{ibid} or a shortened citation.

\cmditem{fullcite*}[prenote][postnote]{key}

Mainly for testing, this command prints the first, complete citation
of a work in the format of a note but without the author's position.
Like \cmd{cite*}, but unlike the entry option \opt{noauth} (see
Section \ref{entryops}), the corresponding entries in bibliographies
retain the author's position. However, like \cmd{fullcite}, and unlike
\cmd{cite*}, it has no effect on citation tracking. There is perhaps
no practical use for this command aside from testing.

\cmditem{fullparencite}[prenote][postnote]{key}

Also mainly for testing, this command prints the first, complete
parenthetical citation of a work. It has no effect on citation
tracking.

\cmditem{fullparencite*}[prenote][postnote]{key}

This command does for parenthetical citations what \cmd{fullcite*}
does for notes: It prints the first, complete parenthetical citation
of a work but without the author's position and without affecting
citation tracking. It, too, may have no practical use aside from
testing.

\cmditem{parencite*}[prenote][postnote]{key}

Use this command to print a parenthetical citation without the
author's position. The most likely contexts for this are sentences in
which the author receives explicit mention.\footnote{For discussion
and examples in \textit{CMS}, see especially 15.24.}

\cmditem{refentry}{key}

Similar to \cmd{bibentry}, this command prints the reference list
entry of a work.

\end{ltxsyntax}

\subsection{Additional Data Fields}
\label{datafields}

The style recognizes several data fields that are not available with
\biblatex.

\begin{marglist}

\item[bookbooktitle] This field is for the style's internal use. Do
not use it in a bibliography database.

\item[bookyear] Also for the style's internal use. Do not use it in a
bibliography database.

\item[editoraddon] With this field, you may include additional
editorial information about a work. An example of this appeared
earlier, the citation of \textit{Chaucer Life-Records} in Section
\ref{entryops}, where the editors' names are followed, without
punctuation, by ``from materials compiled by John M. Manly and Edith
Richert, with the assistance of Lilian J. Redstone et al.''

\item[seriesaddon] This field is for additional information about a
book's series, such as \textit{2nd ser.} and \textit{n.s.}. For
examples, see \ref{14.132} and \ref{14.195}.

\item[shortbooktitle] This field is for the short form of a
\bibfield{booktitle}, just as \bibfield{shorttitle} is for
\bibfield{title}. Nevertheless, its use is mainly internal. You never
need to use it in a bibliography database. Instead, always use
\bibfield{shorttitle}. For more information on how the style handles
titles, see Section \ref{authtitles}.

\item[volumea] Use this field when an entry requires a second volume
number. For an example, see Section \ref{collect}.

\item[volumetitle] Use this field when a volume has a title. In the
following example, from \textit{CMS} \ref{14.123}, the
\bibfield{volumetitle} is \textit{1883--1884}:

\begin{itemize}
\item[N] \fullcite*[32--33]{james1963.5}

\item[B] \bibentry{james1963.5}
\end{itemize}

\end{marglist}

\subsection{Data Fields for Authors and Titles}
\label{authtitles}

The style allows for some flexibility over how you designate authors
and titles in bibliography databases. In particular, some standard
data fields are optional: \bibfield{bookauthor}, \bibfield{booktitle},
and \bibfield{booksubtitle}. No bibliography database entry needs
them, though using them causes no harm. As such, you may designate a
book's author, title, and subtitle in one of three ways:

\begin{lstlisting}
  ...
  author = {Doe, Jane},
  title = {A Book's Title},
  subtitle = {A Book's Subtitle},
  ...
  bookauthor = {Doe, Jane},
  booktitle = {A Book's Title},
  booksubtitle = {A Book's Subtitle},
  ...
  bookauthor = {Doe, Jane},
  title = {A Book's Title},
  subtitle = {A Book's Subtitle},
  ...
  author = {Doe, Jane},
  booktitle = {A Book's Title},
  booksubtitle = {A Book's Subtitle},
  ...
\end{lstlisting}

\noindent Note that a \bibfield{subtitle} must go with a
\bibfield{title} and a \bibfield{booksubtitle} must go with a
\bibfield{booktitle}. Also, cross-referencing is no concern, so one
work may cross-reference another while each retains its own
\bibfield{author}, \bibfield{title}, and \bibfield{subtitle}. For
example:

\begin{lstlisting}
  @InCollection{doe2014a,
    crossref = {doe2014},
    author = {Doe, Jane},
    title = {An Article's Title},
    subtitle = {An Article's Subtitle},
    ...
  }
  @Collection{doe2014,
    author = {Doe, Jane and Doe, John},
    title = {A Book's Title},
    subtitle = {A Book's Subtitle},
    ...
  }
\end{lstlisting}

\noindent Much the same holds for editors and translators. See Section
\ref{edtrans}.

\subsection{Data Fields for Collections}
\label{collect}

Typical examples of \bibtype{collection} and \bibtype{incollection}
entries appear in Section \ref{notes16}. But not all such works are
typical. This section addresses some of them.

Although its discussion is a bit obscure, \textit{CMS} treats some
multivolume works as if they formed a single unit---but only if every
volume of the collection has the same title and publication
year.\footnote{See the examples of short citations in 14.29 and
14.122--14.125.} The style accommodates these citations. Below is a
note with an example from \ref{14.123}.\footcite[243]{byrne1981.4} In
citing the fourth volume of the collection, the style prints the
volume number not in the usual place, between the title and
publisher's address, but near the end of the note, separated from the
page number by a colon. Citing the fifth volume immediately
thereafter, on the same page, produces an \textit{ibid} with the same
arrangement of volume and page number.\footcite[91]{byrne1981.5} This
occurs even though the fourth and fifth volumes are distinct entries
in the bibliography database.

If citations to other volumes are not consecutive, or if consecutive
citations are not on the same page, the style will render them in the
short form, even if those volumes have not been cited before. For
example, here is an intervening note.\footnote{Hello!} And here is a
citation of the sixth volume.\footcite[23--32]{byrne1981.6}

Since notes of this sort must indicate the volume number, the style
always prints it, regardless of whether the citation includes a page
reference. So, if the last note had no page reference, the style would
have printed ``Ibid., vol. 6.'' The first note would have shown the
volume number in the usual place:

\begin{itemize}
\item[] \fullcite{byrne1981.4}
\end{itemize}

Citing multivolume works in this way requires you to cross-reference
individual volumes to their collection. In the bibliography database
for this document, the entry for the fourth volume of Byrne's
collection is:

\begin{lstlisting}
  @InBook{byrne1981.4,
    crossref = {byrne1981},
    volume = {4},
    year = {1981}
  }
\end{lstlisting}

\noindent The above is the minimum information necessary to get the
correct output.\footnote{As noted later, using \bibtype{inbook} for
books in collections may seem to stretch the usual profile of the
entry type. To avoid this confusion, while getting the same results,
you could use \bibtype{bookinbook}.}

Adding more information has no effect. Thus, you could change the
entry to:

\begin{lstlisting}
  @InBook{byrne1981.4,
    crossref = {byrne1981},
    title = {The Lisle Letters},
    editor = {Byrne, Muriel St. Clare},
    volume = {4},
    address = {Chicago},
    publisher = ucp,
    year = {1981}
  }
\end{lstlisting}

\noindent However you fill them out, entries for individual volumes
must cross-reference the col\-lec\-tion---in this case,
\bibfield{byrne1981}:

\begin{lstlisting}
  @Collection{byrne1981,
    editor = {Byrne, Muriel St. Clare},
    title = {The Lisle Letters},
    volumes = {6},
    address = {Chicago},
    publisher = ucp,
    year = {1981}
  }
\end{lstlisting}

Occasionally, getting acceptable output for a multivolume work
requires some trial-and-error. Consider a difficult example:

\begin{itemize}
\item[N] \fullcite{kierkegaard1987}

\item[B] \bibentry{kierkegaard1987}
\end{itemize}

\noindent This edition of \textit{Either/Or} is a two-volume work,
each volume of which is a volume of \textit{Kierkegaard's Writings}.
In the bibliography database, \bibfield{volume} contains \textit{3 and
4}, though ordinarily it would contain a single number for a single
volume. The style responds by preceding \textit{3 and 4} with the
abbreviation \textit{vols.}, rather than the usual, \textit{vol}.
Meanwhile, \bibfield{volumes} contains \textit{2}, indicating the two
volumes of \textit{Either/Or}, and \bibfield{maintitle} contains
\textit{Kierkegaard's Writings}. No cross-referencing occurs with the
entry.\footnote{You could drop \bibfield{volume} for simpler output.
But there are occasions when you need it along with
\bibfield{volumes}, such as when a two-volume work comprises a single
volume of a multivolume collection. This is the case with \textit{The
Wealth of Nations} cited in Section \ref{edtransnames}.}

The greatest difficulty, though, is how to cite particular works from
\textit{Either/Or}. A citation should include the volume of the book.
But each volume is also a volume of \textit{Kierkegaard's Writings},
which makes each volume a volume twice over. Plus, each volume is a
collection, which means that each volume is a collection (consisting
of essays) within a collection (the two-volume book) within a
collection (of \textit{Kierkegaard's Writings}).

The title pages of \textit{Either/Or} might suggest a solution. They
refer to the volumes as \textit{Part I} and \textit{Part II}. But
using \bibfield{part} is incorrect, since its function in entries for
books is to indicate parts of a volume, not volumes of a collection.
What should you do?

One solution is to keep a separate entry in the bibliography database
for each volume, without cross-referencing to the original entry, and
to use \bibfield{note} for holding \textit{Part I} and \textit{Part
II}. Citing particular works from each volume is then a
straightforward matter of making \bibtype{incollection} entries for
those works and cross-referencing them to the new \bibtype{collection}
entries. The output is at least adequate, though the placement of
\textit{Part I} may seem odd:

\begin{itemize}
\item[N] \fullcite[290]{kierkegaard1987.1.8a}

\item[B] \bibentry{kierkegaard1987.1.8a}
\end{itemize}

\noindent Alternatively, you could have the style treat
\textit{Either/Or} in the same way that it treats \textit{The Lisle
Letters} above. Here, too, you have options. First, cross-reference
each volume of \textit{Either/Or} to the collection, making sure to
include the year and volume in each \bibtype{inbook} entry. Then,
cross-reference individual works to their respective volume:

\begin{lstlisting}
  @InCollection{kierkegaard1987.1.8,
    crossref = {kierkegaard1987.1},
    title = {Rotation of Crops},
    pages = {281--300},
  }
  @InBook{kierkegaard1987.1,
    crossref = {kierkegaard1987},
    volume = {1},
    year = {1987}
  }
  @Collection{kierkegaard1987,
    author = {Kierkegaard, S{\o}ren},
    title = {Either/Or},
    editor = {Hong, Howard V. and Hong, Edna H.},
    translator = {Hong, Howard V. and Hong, Edna H.},
    address = {Princeton, NJ},
    publisher = {Princeton University Press},
    year = {1987}
  }
\end{lstlisting}

\noindent These entries drop the \bibfield{note} and
\bibfield{maintitle} and use \bibfield{volume} for the volume of
\textit{Either/Or}. The resulting output is fairly clean:

\begin{itemize}
\item[N] \fullcite[290]{kierkegaard1987.1.8b}

\item[B] \bibentry{kierkegaard1987.1.8b}
\end{itemize}

But what if you want to include the \bibfield{maintitle} with its
volume number? Add \bibfield{maintitle} to \bibtype{collection} along
with a new field, \bibfield{volumea}, for the volume of
\textit{Kierkegaard's Writings}. The result:

\begin{itemize}
\item[N] \fullcite[290]{kierkegaard1987.1.8c}

\item[B] \bibentry{kierkegaard1987.1.8c}
\end{itemize}

By default, the style adds an entry to the bibliography for every work
that you cite from a collection. If you cite two or more works from a
collection, the style will add a separate entry for the collection
(see \ref{bib}, under Byrne). If you prefer a different threshold,
load \biblatex with a different value for the preamble option
\opt{mincrossrefs}. A value of \textit{1}, for example, will add an
entry for the collection if you cite from it at least once.

Multivolume works like \textit{The Lisle Letters} are perhaps best
accounted for in a bibliography by a single entry for the collection
and no entries for individual volumes. For that, use the preamble
option \opt{collsonly} (see Section \ref{preops}).

\subsection{Data Fields for Series}

Often, the only question about a book's series is whether to count it
as a series at all, rather than as the title of a multivolume work.
After that, the most common question is how to format a series number.
Sometimes, the number appears alone, with no preceding abbreviation.
But it may also appear with \textit{vol.} or
\textit{no.}\footcite[\ref{14.128}. See also \ref{14.130} and
\ref{14.132}.]{chicago2010}. In order to avoid using multiple fields
for essentially the same task, the style uses \bibfield{number} for
all of them. By default, it places no abbreviation before the number.
Add abbreviations to the field as necessary. In the database entry for
the example below, \bibfield{number} contains \textit{vol. 6}:

\begin{itemize}
\item[N] \fullcite{cochrane1987}

\item[B] \bibentry{cochrane1987}
\end{itemize}

To indicate the run of a series, such as \textit{2nd ser.} or
\textit{n.s.}, use \bibfield{seriesaddon} (see Section
\ref{datafields}). However, this does not apply to journals, where
such labels usually modify a journal's title. For them, use
\bibfield{series}, as in this example from \ref{14.195}:

\begin{itemize}
\item[N] \fullcite{moraes1950}

\item[B] \bibentry{moraes1950}
\end{itemize}

\subsection{Entry Types and Their Aliases}

In a bibliography database, every entry has an entry type. The style
recognizes the standard ones for \BibTeX, as well as some that are
specific to \biblatex. A relatively small number of entry types are
basic. The style treats the rest as aliases of the basic ones.

\begin{typelist}\raggedright

\typeitem{article}

Aliases: \bibtype{periodical}

\typeitem{book}

Aliases: \bibtype{booklet}, \bibtype{collection}, \bibtype{manual},
\bibtype{proceedings}, \bibtype{reference}, \bibtype{report},
\bibtype{techreport}

\typeitem{incollection}

Aliases: \bibtype{bookinbook}, \bibtype{conference},
\bibtype{inproceedings}, \bibtype{inbook}, \bibtype{inreference},
\bibtype{letter}, \bibtype{suppbook}, \bibtype{suppcollection}

\typeitem{letter} Aliases: none
\typeitem{misc} Aliases: none
\typeitem{online} Aliases: none
\typeitem{patent} Aliases: none
\typeitem{review} Aliases: none

\typeitem{thesis}

Aliases: \bibtype{mastersthesis}, \bibtype{phdthesis},
\bibtype{unpublished}

\end{typelist}

\noindent For the most part, you may assign every work to the basic
entry types listed above. A PhD thesis, for example, may have the
entry type \bibtype{thesis} or \bibtype{phdthesis}; the output is the
same either way. An exception applies to books in collections: Every
book in a one-volume or multivolume collection needs the
\bibtype{inbook} or \bibtype{bookinbook} entry type.\footnote{Using
\bibtype{inbook} for books in collections departs somewhat from the
specifications of \BibTeX\ and \biblatex, though it seems harmless
enough.} You may choose either \bibtype{inbook} or
\bibtype{bookinbook} entries have the same format. If an entry belongs
to a type other than the ones listed above, the style processes it as
a book.

For unusually complicated ci\-ta\-tions---at any rate, those too
complicated for the style---consider using the \bibtype{misc} entry
type. The style formats these entries with a small number of fields
but in way that makes it a fallback for almost anything. The example
below is from 14.277:

\begin{itemize}
\item[N] \fullcite{roosevelt1959}

\item[B] \bibentry{roosevelt1959}

\item[P] \fullparencite{roosevelt1959}

\item[R] \refentry{roosevelt1959}
\end{itemize}

\noindent The database entry for this work contains most of the format
in \bibfield{usera} (for the note) and \bibfield{userb} (for
bibliographies). In \bibfield{title}, the style needs manual
formatting, since works of this type may have titles in italics or
quotation marks. Some trickery with the year helps with formatting the
reference list entry.

\begin{lstlisting}
  @Misc{roosevelt1959,
    author = {Roosevelt, Eleanor},
    title = {\mkbibquote{Is America Facing World Leadership?}},
    usera = {convocation speech, Ball State Teacher's College,
             May 6, \thefield{year}, radio broadcast, Windows
             Media Audio, 47:46},
    userb = {Convocation Speech. Ball State Teacher's College.
             May 6, \thefield{year}. Radio broadcast.
             Windows Media Audio. 47:46},
    url = {http://libx.bsu.edu/cdm4/item_viewer.php?CISOROOT=
           /ElRoos&CISOPTR=0&CISOBOX=1&REC=2},
    year = {1959}
  }
\end{lstlisting}

\noindent No other example in this document relies on \bibtype{misc}.

\section{Editors, Translators, and Compilers}
\label{edtrans}

The style offers a wide range of different classifications of editors
and some flexibility over the handling of translators and compilers.
This section explains how to use these features.

\subsection{Basic Placement of Editors, Translators, and Compilers}
\label{edtranspos}

Normally, the style prints editors' names first. However, if
translators are listed first on a work's title page (or in some other
relevant place), you may want to reverse the order and print the
translators' names first. For that, use the entry option
\opt{transfirst} (see Section \ref{entryops}). Compare:

\begin{itemize}
\item[N] \fullcite{doe2010a}

\item[B] \bibentry{doe2010a}

\item[N] \fullcite{doe2010b}

\item[B] \bibentry{doe2010b}
\end{itemize}

If a work has no author, but has an editor, the style will print the
name of the editor in the author's position, as in this example from
\textit{CMS} \ref{14.87}:

\begin{itemize}
\item[N] \fullcite[94]{young2007}

\item[B] \bibentry{young2007}
\end{itemize}

\noindent A similar switch happens if a work has no author but has a
translator:

\begin{itemize}
\item[N] \fullcite[34]{silverstein1974}

\item[B] \bibentry{silverstein1974}
\end{itemize}

Again, the style prints editors' names first, so if a work has no
author but has editors and translators, the style will print the
editors' names in the author's position:

\begin{itemize}
\item[N] \fullcite{smith2002a}

\item[B] \bibentry{smith2002a}
\end{itemize}

\noindent Use \opt{transfirst} to reverse them:

\begin{itemize}
\item[N] \fullcite{smith2002b}

\item[B] \bibentry{smith2002b}
\end{itemize}

Similarly, if a work's editors and translators are identical, the
style will print the editors' role first, as in, ``Edited and
translated by\ldots'' Use \opt{transfirst} to reverse them, as in this
example from \textit{CMS} \ref{14.88}:

\begin{itemize}
\item[N] \fullcite{menchu1999}

\item[B] \bibentry{menchu1999}
\end{itemize}

Since the style treats a compiler as a kind of editor, the comments
above apply to compilers, too: If a work has compilers and
translators, the style will print compilers' names first, unless you
use \opt{transfirst}. For more on compilers, see Section
\ref{edtransnames}.

\subsection{Types of Editors, Translators, and Compilers}
\label{edtransnames}

The style recognizes five kinds of editors, those of a
\bibfield{title}, \bibfield{booktitle}, \bibfield{maintitle},
\bibfield{issuetitle}, or \bibfield{series}. The style recognizes two
kinds of translators, those of a \bibfield{title} or
\bibfield{booktitle}. Although the style treats compilers as a kind of
editor, it recognizes just two types of them, those of a
\bibfield{title} or \bibfield{booktitle}.\footnote{As will become
clear soon, indicate a compiler by giving the value \textit{compiler}
to the appropriate \bibfield{editortype}.}

As confusing as all this may seem, the assignments of editors and
translators are often automatic. The style's default is to associate
\bibfield{editor} and \bibfield{translator} with the lowest level
title within the scope of an entry. For most sources, then, no more
work is necessary than just adding \bibfield{editor} and
\bibfield{translator} fields to a database entry. For example, this
entry shows data for \ref{14.88}:

\begin{lstlisting}
  @Book{menchu1999,
    options = {transfirst},
    author = {Mench{\'u}, Rigoberta},
    title = {Crossing Borders},
    editor = {Wright, Ann},
    translator = {Wright, Ann},
    address = {New York},
    publisher = {Verso},
    year = {1999}
  }
\end{lstlisting}

\noindent When the style processes this entry, it identifies Ann
Wright as the book's translator and editor.\footnote{The entry option
\opt{transfirst} ensures that her role as translator is mentioned
first. See Section \ref{entryops}.}

Cross-referencing introduces a bit more complexity, but the principle
remains the same: Within the scope of an entry, the style associates
\bibfield{editor} and \bibfield{translator} with the lowest level
title. In this example, which shows an essay cross-referenced to a
collection, \bibfield{editor} appears within the collection's database
entry:

\begin{lstlisting}
  @InCollection{thoreau2007.7,
    crossref = {thoreau2007},
    title = {Walking},
    pages = {185--222},
  }
  @Collection{thoreau2007,
    author = {Thoreau, Henry David},
    title = {Excursions},
    editor = {Moldenhauer, Joseph J.},
    series = {The Writings of Henry D. Thoreau},
    publisher = {Princeton University Press},
    address = {Princeton, NJ},
    year = {2009}
  }
\end{lstlisting}

\begin{itemize}
\item[N] \fullcite{thoreau2007.7}

\item[B] \bibentry{thoreau2007.7}
\end{itemize}

\noindent As a result, the style identifies Joseph Moldenhauer as the
editor of \textit{Excursions}. If you moved \bibfield{editor} from
\bibtype{collection} to \bibtype{incollection}, the
style would identify him as the editor of ``Walking.'' If instead you
want him as the editor of the series, leave \bibfield{editor} in
\bibtype{collection} but add a line to the entry:

\begin{lstlisting}
  editortype = {series},
\end{lstlisting}

Other values of \bibfield{editortype} are \textit{maintitle},
\textit{issuetitle}, and \textit{compiler}. So, to add a compiler to
an entry, add the compiler's name to \bibfield{editor}, then add
\bibfield{editortype} with the value \textit{compiler}.\footnote{Of
course, if \bibfield{editor} is occupied, do the same for
\bibfield{editora} and \bibfield{editoratype}, and so on.}

\begin{lstlisting}
  @InCollection{orwell2009.17,
    crossref = {orwell2009},
    title = {Politics and the English Language},
    pages = {270--86}
  }
  @Collection{orwell2009,
    author = {Orwell, George},
    title = {All Art is Propaganda},
    subtitle = {Critical Essays},
    editor = {Packer, George},
    editortype = {compiler},
    address = {Boston},
    publisher = {Mariner Books},
    year = {2009}
  }
\end{lstlisting}

\begin{itemize}
\item[N] \fullcite{orwell2009.17}

\item[B] \bibentry{orwell2009.17}
\end{itemize}

As already mentioned, this way of handling editors and translators
allows you to assign them to particular works in a collection. This is
most often useful for citing translators, as in the following:

\begin{itemize}
\item[N] \fullcite{cassirer1948.4}

\item[B] \bibentry{cassirer1948.4}
\end{itemize}

\noindent Hans Nachod translated ``The Ascent of Mont Ventoux,'' among
other works in the collection, but not \textit{every} work in the
entire collection. Thus, \bibfield{translator} must fall within the
scope of \bibtype{incollection}:

\begin{lstlisting}
  @InCollection{cassirer1948.4,
    crossref = {cassirer1948},
    author = {Petrarca, Francesco},
    title = {The Ascent of Mont Ventoux},
    translator = {Nachod, Hans},
    pages = {36--46}
  }
  @Collection{cassirer1948,
    title = {The Renaissance Philosophy of Man},
    editor = {Cassirer, Ernst and Kristeller, Paul Oskar and
              Randall, Jr., John Herman},
    address = {Chicago},
    publisher = ucp,
    year = {1948}
  }
\end{lstlisting}

\noindent By the same token, since \bibfield{editor} falls within
\bibtype{collection}, the style assigns it to \textit{The Renaissance
Philosophy of Man}.

One more complication remains: There are other name lists for editors
than just \bibfield{editor}. The style also recognizes
\bibfield{editora}, \bibfield{editorb}, and
\bibfield{editorc}.\footnote{The style also recognizes
\bibfield{translatora}, though only for internal purposes.}

Designate editors much like you designate titles. Reserve
\bibfield{editor} for the lowest level title in a work that you wish
to cite, say, an essay in a collection or a single volume in a
multivolume work. The next level up, as it were, is for
\bibfield{editora}, followed by \bibfield{editorb}, and so on.
Remember to include the appropriate \bibfield{type} field to indicate
an editor's role. These fields are \bibfield{editortype} (for
\bibfield{editor}), \bibfield{editoratype} (for \bibfield{editora}),
\bibfield{editorbtype} (for \bibfield{editorb}), and
\bibfield{editorctype} (for \bibfield{editorc}). For example:

\begin{itemize}
\item[N] \fullcite{smith1981}

\item[B] \bibentry{smith1981}
\end{itemize}

\noindent Since the work above has two sets of editors, the database
entry needs to use \bibfield{editor} and \bibfield{editora}:

\begin{lstlisting}
  @Book{smith1981,
    author = {Smith, Adam},
    title = {An Inquiry into the Nature and Causes of the
             Wealth of Nations},
    shorttitle = {The Wealth of Nations},
    editor = {Todd, W. B.},
    volumes = {2},
    volume = {2},
    maintitle = {The Glasgow Edition of the Works and
                 Correspondence of Adam Smith},
    editora = {Campbell, R. H. and Skinner, A. S.},
    editoratype = {maintitle},
    address = {Indianapolis},
    publisher = {Liberty Fund},
    year = {1981},
    origlocation = {Oxford},
    origpublisher = {Oxford University Press},
    origdate = {1976}
  }
\end{lstlisting}

\section{Examples from \emph{CMS} Chap. 14, ``Notes and
Bibliography''} \label{notes16}

Examples in this section reproduce those in \textit{CMS} Chapter 14.
To help with cross-checking, subsection numbers and headings below
follow those of the book. \textit{N} indicates a note, \textit{B}
indicates a bibliography entry. Bear in mind that the book's examples
occasionally show inconsistencies, a few of them very likely errors:

\begin{enumerate} \item \ref{14.78} includes \textit{3 vols.} in a
note for \textit{The Letters of George Meredith}, inconsistent with
examples in \ref{14.123} and \ref{14.125}

\item \ref{14.82} has \textit{Trans} instead of \textit{Translated by}
in a bibliography entry for \textit{The Charterhouse of Parma}.

\item \ref{14.132} omits a comma after \textit{ed.} in a note for
\textit{South China in the Sixteenth Century}.

\item \ref{14.187} inverts an author's name in a note for ``Learning
to be an `American Lady'?'' and omits a page reference in a
bibliography entry for ``Non-Subject-Matter Outcomes of Schooling.''
\end{enumerate}

\noindent Also, the style always prints \bibfield{title} before
\bibfield{maintitle}. In several sections, \textit{CMS} offers the
option of reversing this order, and at least once, in \ref{14.120},
reverses them without showing the alternative. This potential source
of confusion affects \ref{14.120}, \ref{14.124}, \ref{14.126}, and
\ref{14.127}.

\subsection{Books}
\setcounter{subsection}{14}

\setcounter{subsubsection}{74}
\subsubsection{One author}
% 14.75

\begin{itemize}
\item[N] \fullcite{shields2008}

\item[B] \bibentry{shields2008}

\item[N] \fullcite{martin2000}

\item[B] \bibentry{martin2000}
\end{itemize}

\subsubsection{Two or more authors (or editors)}
% 14.76
\label{14.76}

\begin{itemize}
\item[N] \fullcite[32]{jacobs1997}

\item[B] \bibentry{jacobs1997}

\item[N] \fullcite[20--21]{levitt2005}

\item[B] \bibentry{levitt2005}

\item[N] \fullcite[243]{sechzer1996}

\item[B] \bibentry{sechzer1996}
\end{itemize}

\setcounter{subsubsection}{77}
\subsubsection{Author's name in title}
% 14.78
\label{14.78}

\begin{itemize}
\item[N] \fullcite*[44--45]{mccullers1999}

\item[B] \bibentry{mccullers1999}

% Error in book: inclusion of '3 vols.' is inconsistent with examples
% in 14.123 and 14.125:
%\item[N] \fullcite*[125]{meredith1970.1}

%\item[B] \bibentry{meredith1970.1}
\end{itemize}

\subsubsection{Anonymous works---unknown authorship}
% 14.79

\begin{itemize}
\item[N] \fullcite{anon1610}

\item[B] \bibentry{anon1610}

\item[N] \fullcite{anon1547}

\item[B] \bibentry{anon1547}
\end{itemize}

\subsubsection{Anonymous works---known authorship}
% 14.80
\label{14.80}

\begin{itemize}
\item[N] \fullcite{horsley1796}

\item[B] \bibentry{horsley1796}

\item[N] \fullcite{cook1730}

\item[B] \bibentry{cook1730}
\end{itemize}

%\subsubsection{Pseudonyms---unknown authorship}
% 14.81

% See comment for 14.82, below.

%\begin{itemize}
%\item[N] \fullcite{centinel1981}

%\item[B] \bibentry{centinel1981}
%\end{itemize}

\setcounter{subsubsection}{81}
\subsubsection{Pseudonyms---known authorship}
% 14.82
\label{14.82}

If you want to give an author's real name, put it in the
\bibfield{nameaddon} field. If the real name is unknown, and you want
to indicate that the printed name is a pseudonym, put 'pseud.' in
\bibfield{nameaddon}. \textit{CMS} discusses the latter issue in
14.81, but the style has problems formatting its example.

\begin{itemize}
\item[N] \fullcite{carre1982}

\item[B] \bibentry{carre1982}

\item[N] \fullcite{stendhal1925}

% Error in book: 'Trans' instead of 'Translated by':
\item[B] \bibentry{stendhal1925}
\end{itemize}

\setcounter{subsubsection}{86}
\subsubsection{Editor in place of author}
% 14.87
\label{14.87}

\begin{itemize}
\item[N] \fullcite[94]{young2007}

\item[B] \bibentry{young2007}

\item[N] \fullcite[34]{silverstein1974}

\item[B] \bibentry{silverstein1974}
\end{itemize}

\subsubsection{Editor or translator in addition to author}
% 14.88
\label{14.88}

\begin{itemize}
\item[N] \fullcite{bonnefoy1995}

\item[B] \bibentry{bonnefoy1995}

\item[N] \fullcite{menchu1999}

\item[B] \bibentry{menchu1999}

\item[N] \fullcite{adorno1999}

\item[B] \bibentry{adorno1999}
\end{itemize}

\subsubsection{``With the assistance of'' and the like}
% 14.89

% Use the \bibfield{note} field to include this sort of information.
% Since \cmd{biblatex} handles capitalization automatically, don't put
% the first letter of the note in uppercase.

\begin{itemize}
\item[N] \fullcite{crow1966}

\item[B] \bibentry{crow1966}

\item[N] \fullcite{cullen1961}

\item[B] \bibentry{cullen1961}

\item[N] \fullcite{prather1998}

\item[B] \bibentry{prather1998}
\end{itemize}

\setcounter{subsubsection}{90}
\subsubsection{Authors of forewords and the like}
% 14.91

\begin{itemize}
\item[N] \fullcite{hayek1994}

\item[B] \bibentry{hayek1994}
\end{itemize}

\subsubsection{Organization as author}
% 14.92

% If an organization is the work's author, remember to add an extra
% pair of brackets around the name of the organization in your
% bibliography database.

\begin{itemize}
% Use '\fullcitens' to suppress shorthand:
\item[N] \fullcitens{chicago2010}

\item[B] \bibentry{chicago2010}
\end{itemize}

\setcounter{subsubsection}{96}
\subsubsection{Subtitles---the colon}
% 14.97

\begin{itemize}
\item[N] \fullcite{weiss2008}

\item[B] \bibentry{weiss2008}
\end{itemize}

%\subsubsection{Two subtitles}
% 14.98

\setcounter{subsubsection}{101}
\subsubsection{Titles within titles}
% 14.102

\begin{itemize}
\item[N] \fullcite{mchugh1980}

\item[B] \bibentry{mchugh1980}
\end{itemize}

\setcounter{subsubsection}{102}
\subsubsection{Italicized terms within titles}
% 14.103

\begin{itemize}
\item[N] \fullcite{vanwagenen1973}

\item[B] \bibentry{vanwagenen1973}
\end{itemize}

\setcounter{subsubsection}{104}
\subsubsection{Question marks or exclamation points in book titles}
% 14.105

\begin{itemize}
\item[N] \fullcite[63]{berra2002}

\item[B] \bibentry{berra2002}

\item[N] \fullcite[183]{oram2007}

\item[B] \bibentry{oram2007}
\end{itemize}

\setcounter{subsubsection}{107}
\subsubsection{Translated titles supplied by author or editor}
% 14.108

\begin{itemize}
%\item[N] \fullcite[; includes a summary in German]{wereszycki1977}
\item[N] \fullcite{wereszycki1977}

\item[B] \bibentry{wereszycki1977}
\end{itemize}

\setcounter{subsubsection}{108}
\subsubsection{Original plus published translation}
% 14.109

\begin{itemize}
\item[N] \fullcite{furet1999}

\item[B] \bibentry{furet1999}
\end{itemize}

%\subsubsection{Unpublished translation of title standing in for original}
% 14.110

% This works, though not the punctuation in the publisher's name:
%\begin{itemize}
%\item[N] \fullcite{pirumova1977}

%\item[B] \bibentry{pirumova1977}
%\end{itemize}

\setcounter{subsubsection}{110}
\subsubsection{Chapter in a single-author book}
% 14.111

\begin{itemize}
\item[N] \fullcite[117--63]{phibbs1987a}

\item[B] \bibentry{phibbs1987a}

\item[N] \fullcite{samples2006.7}

\item[B] \bibentry{samples2006.7}
\end{itemize}

\subsubsection{Contribution to a multiauthor book}
% 14.112
\label{14.112}

\begin{itemize}
\item[N] \fullcite[14]{carr1996a}

\item[B] \bibentry{carr1996a}

\item[N] \fullcite{ellet1968a}

\item[B] \bibentry{ellet1968a}
\end{itemize}

\setcounter{subsubsection}{113}
\subsubsection{Book-length work within a book}
% 14.114

\begin{itemize}
\item[N] \fullcite{bernard1990a}

\item[B] \bibentry{bernard1990a}
\end{itemize}

\setcounter{subsubsection}{115}
\subsubsection{Introductions, prefaces, afterwords, and the like}
% 14.116

\begin{itemize}
\item[N] \fullcite{polakow1993a}

\item[B] \bibentry{polakow1993a}

\item[N] \fullcite{prose2000}

\item[B] \bibentry{prose2000}
\end{itemize}

\subsubsection{Letters in published collections}
% 14.117

\begin{itemize}
\item[N] \fullcite[133--34]{adams1867}

\item[B] \bibentry{adams1867}

\item[N] \fullcite{jackson1676}

\item[B] \bibentry{jackson1676}
\end{itemize}

\setcounter{subsubsection}{117}
\subsubsection{Editions other than the first}
% 14.118

\begin{itemize}
\item[N] \fullcite[43]{harper-dorton2002}

\item[B] \bibentry{harper-dorton2002}

\item[N] \fullcite[199]{babb1989}

\item[B] \bibentry{babb1989}
\end{itemize}

\subsubsection{Reprint editions and modern editions}
% 14.119

% You can have at most one \bibfield{origdate} per entry. So, if the
% citation is to a work in a collection, say, an article or book in an
% anthology, the style assumes that \bibfield{origdate} is for the
% collection, not for the individual work.

\begin{itemize}
%\item[N] \fullcite{emerson1985}

%\item[B] \bibentry{emerson1985}

\item[N] \fullcite[152--53]{barzun1994}

\item[B] \bibentry{barzun1994}

\item[N] \fullcite{schweitzer1966}

\item[B] \bibentry{schweitzer1966}
\end{itemize}

\subsubsection{Microform editions}
% 14.120
\label{14.120}

% In the book, the \bibfield{maintitle} precedes the \bibfield{title}.
% Hence, the book reads 'vol. 12' instead of 'vol. 12 of'.

\begin{itemize}
\item[N] \fullcite[p. 67, 3C12]{farwell1997}

\item[B] \bibentry{farwell1997}
\end{itemize}

\setcounter{subsubsection}{121}
\subsubsection{Citing a multivolume work as a whole}
% 14.122
\label{14.122}

\begin{itemize}
\item[N] \fullcite{aristotle1983}

\item[B] \bibentry{aristotle1983}

\item[N] \fullcite{byrne1981}

\item[B] \bibentry{byrne1981}
\end{itemize}

\subsubsection{Citing a particular volume in a note}
% 14.123
\label{14.123}

\begin{itemize}
\item[N] \fullcite[243]{byrne1981.4}

\item[B] \bibentry{byrne1981.4}

\item[N] \fullcite*[32--33]{james1963.5}

\item[B] \bibentry{james1963.5}
\end{itemize}

\subsubsection{Citing a particular volume in a bibliography}
% 14.124
\label{14.124}

\begin{itemize}
\item[N] \fullcite{pelikan1971}

\item[B] \bibentry{pelikan1971}
\end{itemize}

\subsubsection{Chapters and other parts of individual volumes}
% 14.125
\label{14.125}

\begin{itemize}
\item[N] \fullcite[893--95]{bonnefoy1991a}

\item[N] \fullcite[chap. 6]{bonnefoy1991a}

\item[B] \bibentry{bonnefoy1991a}
\end{itemize}

\subsubsection{One volume in two or more books}
% 14.126
\label{14.126}

% The example with Harley is complicated by how \textit{CMS} shows the
% bibliography entry. One option gives the publication year of the
% title. The other gives the publication years of the main title.

\begin{itemize}
\item[N] \fullcite[351]{lach1977}

\item[B] \bibentry{lach1977}

\item[N] \fullcite{harley1994a}

\item[B] \bibentry{harley1994a}
\end{itemize}

\noindent Alternatively:

\begin{itemize}
\item[N] \fullcite{harley1994b}

\item[B] \bibentry{harley1994b}
\end{itemize}

\subsubsection{Authors and editors of multivolume works}
% 14.127
\label{14.127}

\begin{itemize}
\item[N] \fullcite{barrows1959}

\item[B] \bibentry{barrows1959}

\item[N] \fullcite{donne1995}

\item[B] \bibentry{donne1995}
\end{itemize}

\subsubsection{Series titles, numbers, and editors}
% 14.128
\label{14.128}

\begin{itemize}
\item[N] \fullcite{hundert1992}

\item[B] \bibentry{hundert1992}

\item[N] \fullcite{fowler1989}

\item[B] \bibentry{fowler1989}
\end{itemize}

\subsubsection{Series editor}
% 14.129

\begin{itemize}
\item[N] \fullcite{howell1998}

\item[B] \bibentry{howell1998}
\end{itemize}

\subsubsection{Series or multivolume work?}
% 14.130
\label{14.130}

\begin{itemize}
\item[N] \fullcite{boyer1986}

\item[B] \bibentry{boyer1986}

\item[N] \fullcite{cochrane1987}

\item[B] \bibentry{cochrane1987}
\end{itemize}

%\subsubsection{Multivolume work within a series}
% 14.131
% This won't work. The style doesn't print, in a single entry, the
% titles of multiple volumes in a series.

\setcounter{subsubsection}{131}
\subsubsection{``Old series'' and ``new series''}
% 14.132
\label{14.132}

\begin{itemize}
% Error in book: 'ed.' instead of 'ed.,':
\item[N] \fullcite{boxer1953}

\item[B] \bibentry{boxer1953}

\item[N] \fullcite{palmatary1950}

\item[B] \bibentry{palmatary1950}
\end{itemize}

\setcounter{subsubsection}{137}
\subsubsection{``No place.''}
% 14.138

\begin{itemize}
\item[N] \fullcite{windsor1910}

\item[B] \bibentry{windsor1910}
\end{itemize}

\setcounter{subsubsection}{151}
\subsubsection{``No date.''}
% 14.152

\begin{itemize}
\item[N] \bibentry{boston}
\end{itemize}

\setcounter{subsubsection}{165}
\subsubsection{Books downloaded from a library or bookseller}
% 14.166

\begin{itemize}
\item[N] \fullcite{austen2007}

\item[B] \bibentry{austen2007}
\end{itemize}

\subsubsection{Books consulted online}
% 14.167

\begin{itemize}
\item[N] \fullcite{antokoletz2008}

\item[B] \bibentry{antokoletz2008}
\end{itemize}

\subsubsection{Books on CD-ROM and other fixed media}
% 14.168

\begin{itemize}
\item[N] \fullcite{hicks1996}

\item[B] \bibentry{hicks1996}
\end{itemize}

%\subsubsection{Freely available electronic editions of older works}
% 14.169
% Close but not exact. The style is more consistent:
%\begin{itemize}
%\item[N] \fullcite[bk. 6, chap. 1]{james1996a}

%\item[B] \bibentry{james1996a}
%\end{itemize}

\setcounter{subsection}{1}
\subsection{Periodicals}
\setcounter{subsection}{14}

\setcounter{subsubsection}{175}
\subsubsection{Journal article---title}
% 14.176

\begin{itemize}
\item[N] \fullcite{manjivar2006}

\item[B] \bibentry{manjivar2006}
\end{itemize}

\setcounter{subsubsection}{177}
\subsubsection{Question marks or exclamation points in article titles}
% 14.178

\begin{itemize}
\item[N] \fullcite[336]{batson1990}

\item[B] \bibentry{batson1990}

\item[N] \fullcite{abrams2007}

\item[B] \bibentry{abrams2007}
\end{itemize}

\setcounter{subsubsection}{179}
\subsubsection{Journal volume, issue, and date}
% 14.180

\begin{itemize}
\item[N] \fullcite[153]{meban2008}

\item[B] \bibentry{meban2008}
\end{itemize}

\setcounter{subsubsection}{180}
\subsubsection{No volume number or date only}
% 14.181

\begin{itemize}
\item[N] \fullcite{beattie1974}

\item[B] \bibentry{beattie1974}
\end{itemize}

\subsubsection{Forthcoming articles}
% 14.182

\begin{itemize}
\item[N] \fullcite{authora}

\item[B] \bibentry{authora}
\end{itemize}

\setcounter{subsubsection}{183}
\subsubsection{Electronic journal articles---URI or DOI}
% 14.184

\begin{itemize}
\item[N] \fullcite[268]{whitney1929}

\item[B] \bibentry{whitney1929}
\end{itemize}

\subsubsection{Electronic journal articles---access dates}
% 14.185

\begin{itemize}
\item[N] \fullcite[377]{charles2008}

\item[B] \bibentry{charles2008}
\end{itemize}

\setcounter{subsubsection}{186}
\subsubsection{Special issues}
% 14.187
\label{14.187}

% Since 'issuetitle' is sometimes but not always in quotes, the
% default is not to use them. If you need quotes, as with
% 'sassler2000' below, use '\mkbibquote' in your bibliography
% database.

\begin{itemize}
% Error in book: surname-first name instead of reverse:
\item[N] \fullcite[201--2]{sassler2000}

\item[B] \bibentry{sassler2000}

% Error in book: missing page reference
\item[N] \fullcite{good1999}

\item[B] \bibentry{good1999}
\end{itemize}

\setcounter{subsubsection}{188}
\subsubsection{Articles published in installments}
% 14.189

% The formating is correct for the note, but the style has no way of
% citing both parts of the article in one bibliography entry. I could
% fix this by adding a 'parts' field.

\begin{itemize}
\item[N] \fullcite{brown1978}

\item[B] \bibentry{brown1978}
\end{itemize}

\setcounter{subsubsection}{190}
\subsubsection{Place where journal is published}
% 14.191

\begin{itemize}
\item[N] \fullcite{luu1999}

\item[B] \bibentry{luu1999}

\item[N] \fullcite{garrett1975}

\item[B] \bibentry{garrett1975}
\end{itemize}

\subsubsection{Translated or edited article}
% 14.192
\label{14.192}

\begin{itemize}
\item[N] \fullcite{authorb}

\item[B] \bibentry{authorb}

\item[N] \fullcite{authorc}

\item[B] \bibentry{authorc}
\end{itemize}

\setcounter{subsubsection}{193}
\subsubsection{Translated article titles}
% 14.194

\begin{itemize}
\item[N] \fullcite{kern1938}

\item[B] \bibentry{kern1938}
\end{itemize}

\setcounter{subsubsection}{194}
\subsubsection{New series for journal volumes}
% 14.195
\label{14.195}

\begin{itemize}
\item[N] \fullcite[414]{sewall1896}

\item[B] \bibentry{sewall1896}

\item[N] \fullcite{moraes1950}

\item[B] \bibentry{moraes1950}
\end{itemize}

\setcounter{subsubsection}{196}
\subsubsection{Abstract}
% 14.197

\begin{itemize}
\item[N] \fullcite{hoover2008}

\item[B] \bibentry{hoover2008}
\end{itemize}

\setcounter{subsubsection}{198}
\subsubsection{Citing magazines by date}
% 14.199

\begin{itemize}
\item[N] \fullcite[81]{lepore2008}

\item[B] \bibentry{lepore2008}
\end{itemize}

\subsubsection{Online magazine articles}
% 14.200

\begin{itemize}
\item[N] \fullcite{cole1992}

\item[B] \bibentry{cole1992}
\end{itemize}

\setcounter{subsubsection}{201}
\subsubsection{Magazine departments}
% 14.202

\begin{itemize}
\item[N] \fullcite{wallraff2008}

\item[B] \bibentry{wallraff2008}

\item[N] \fullcite{gourmet2000}

\item[B] \bibentry{gourmet2000}
\end{itemize}

\subsubsection{Newspaper citations---basic elements}
% 14.203

\begin{itemize}
\item[N] \fullcite{royko1992}

\item[B] \bibentry{royko1992}

\item[N] \fullcite{forester2000}

\item[B] \bibentry{forester2000}
\end{itemize}

\setcounter{subsubsection}{204}
\subsubsection{Regular columns}
% 14.205

\begin{itemize}
\item[N] \fullcite{fields2008}

\item[B] \bibentry{fields2008}
\end{itemize}

\setcounter{subsubsection}{206}
\subsubsection{Unsigned newspaper articles}
% 14.207

\begin{itemize}
\item[N] \fullcite{nytimes2002}

\item[B] \bibentry{nytimes2002}
\end{itemize}

%\setcounter{subsubsection}{208}
%\subsubsection{Weekend supplements, magazines, and the like}
% 14.209

%\subsubsection{News releases}
% 14.213

\setcounter{subsubsection}{214}
\subsubsection{Book reviews}
% 14.215
\label{14.215}

\begin{itemize}
\item[N] \fullcite{ratliff1999}

\item[B] \bibentry{ratliff1999}
\end{itemize}

%\subsubsection{Reviews of plays, movies, television programs,
%concerts, and the like}
% 14.216

%\setcounter{subsubsection}{216}
%\subsubsection{Unsigned reviews}
% 14.217

%\begin{itemize}
%\item[N] \fullcite{zeitung1828}

%\item[B] \bibentry{zeitung1828}
%\end{itemize}

%\subsection{Interviews and Personal Communications}
%\setcounter{subsection}{14}

\setcounter{subsection}{2}
\subsection{Unpublished and Informally Published Material}
\setcounter{subsection}{14}

\setcounter{subsubsection}{223}
\subsubsection{Theses and dissertations}
% 14.224

\begin{itemize}
\item[N] \fullcite[59]{vedrashko2006}

\item[B] \bibentry{vedrashko2006}

\item[N] \fullcite{choi2008}

\item[B] \bibentry{choi2008}
\end{itemize}

\setcounter{subsubsection}{225}
\subsubsection{Lectures, papers presented at meetings, and the like}
% 14.226

\begin{itemize}
\item[N] \fullcite{derasmo2000}

\item[B] \bibentry{derasmo2000}
\end{itemize}

\noindent The 15th edition of \textit{The Chicago Manual of Style}
shows this citations of an event spanning several
days:\footcite[17.215]{chicago2003}

\begin{itemize}
\item[N] \fullcite{nass2000}

\item[B] \bibentry{nass2000}
\end{itemize}

\setcounter{subsubsection}{227}
\subsubsection{Working papers and other unpublished works}
% 14.228

\begin{itemize}
\item[N] \fullcite{lucki1980}

\item[B] \bibentry{lucki1980}
\end{itemize}

\setcounter{subsubsection}{229}
\subsubsection{Patents}
% 14.230

\begin{itemize}
\item[N] \fullcite{iizuka1986}

\item[B] \bibentry{iizuka1986}
\end{itemize}

%\subsection{Special Types of References}
%\setcounter{subsection}{14}

%\subsubsection{Dictionaries and encyclopedias}
% 14.247

%\subsubsection{Dictionaries and encyclopedias online}
% 14.248

\setcounter{subsubsection}{248}
\subsubsection{Pamphlets, reports, and the like}
% 14.249

\begin{itemize}
\item[N] \fullcite{lifestyles1996}

\item[B] \bibentry{lifestyles1996}
\end{itemize}

\section{Examples from \emph{CMS} Chap. 15, ``Author-Date
References''}
\label{paren16}

Examples in this section reproduce those in \textit{CMS} Chapter 15.
To help with cross-checking, section numbers and headings below follow
those of the book. \textit{P} indicates a parenthetical citation
\textit{R}, indicates a reference list entry. Since parenthetical
citations are relatively simple and the format of references lists is
largely derivative of the standard format, the examples below are much
more selective than those in the previous section.

As with the previous section, bear in mind that the book's examples
occasionally show inconsistencies and, very likely, errors.
Specifically, \ref{15.9} includes the publication year twice: in the
expected place for this format and at the end of the entry.

\subsection{Reference Lists and Text Citations}
\setcounter{subsection}{15}

\setcounter{subsubsection}{8}
\subsubsection{Author-date references---examples and variations}
% 15.9
\label{15.9}

\begin{itemize}
\item[P] \fullparencite[99--100]{pollan2006}

\item[R] \refentry{pollan2006}

\item[P] \fullparencite[42]{greenberg2008}

\item[R] \refentry{greenberg2008}

\item[P] \fullparencite[331--32]{blair1977}

\item[R] \refentry{blair1977}

\item[P] \fullparencite[758]{novak2008}

\item[R] \refentry{novak2008}
\end{itemize}

\setcounter{subsubsection}{18}
\subsubsection{Reference list entries with same author(s), same year}
% 15.19
\label{15.19}

\begin{itemize}
\item[P] \parencite[218]{fogel2004b}

\item[R] \refentry{fogel2004b}

\item[P] \parencite[45--46]{fogel2004a}

% Error in book: includes year at end of entry
\item[R] \refentry{fogel2004a}
\end{itemize}

\setcounter{subsection}{1}
\subsection{Special Cases}
\setcounter{subsection}{15}

\setcounter{subsubsection}{33}
\subsubsection{Pseudonyms in author-date references}
% 15.34

\begin{itemize}
\item[P] \fullparencite{stendhal1925}

\item[R] \refentry{stendhal1925}
\end{itemize}

\subsubsection{Editor in place of author in text citations}
% 15.35

\begin{itemize}
\item[P] \fullparencite{silverstein1974}

\item[R] \refentry{silverstein1974}

\item[P] \fullparencite{soltes1999}

\item[R] \refentry{soltes1999}
\end{itemize}

\setcounter{subsubsection}{37}
\subsubsection{Reprint editions and modern editions---more than one
date}
% 15.38

\begin{itemize}
\item[P] \fullparencite{austen2003}

\item[R] \refentry{austen2003}
\end{itemize}

\setcounter{subsubsection}{40}
\subsubsection{``No date'' in author-date references}
% 15.41

\begin{itemize}
\item[P] \fullparencite{nano1750}

%\item[R] \refentry{nano1750}

\item[P] \fullparencite{nano}

%\item[R] \refentry{nano}
\end{itemize}

\subsubsection{``Forthcoming'' in author-date references}
% 15.42

\begin{itemize}
\item[P] \fullparencite{faraday}

\item[R] \refentry{faraday}
\end{itemize}

\setcounter{subsection}{2}
\subsection{Peridocals}
\setcounter{subsection}{15}

\setcounter{subsubsection}{45}
\subsubsection{Parentheses with issue number}
% 15.46

\begin{itemize}
\item[P] \fullparencite{meyerovitch1959}

\item[R] \refentry{meyerovitch1959}

\item[P] \fullparencite{morasse2008}

\item[R] \refentry{morasse2008}
\end{itemize}

\defbibnote{sh}{This section shows the output of
\cmd{printshorthands}. By default, works from this list also appear in
bibliographies and reference lists. To exclude them, use the preamble
option \opt{nolos} (see Section \ref{preops}). Note that the style
does not automatically italicize a \bibfield{shorthand}. Whether you
should italicize it depends on whether you should italicize the work's
title (14.55).\\}%

\defbibnote{bib}{This section shows the default output of
\cmd{printbibliography}. In the next section, the same works appear in
the author-date format.\\}%

\defbibnote{ref}{This section shows the output of
\cmd{printbibliography} for a reference list (see \opt{reflist} in
Section \ref{preops}). The works below are the same as those in the
previous section but in the author-date format.\\}%

\printshorthands[prenote=sh]
\refstepcounter{sh}\label{sh}
\printbibliography[prenote=bib]
\refstepcounter{bib}\label{bib}
\printbibliography[%
  env=reflist,
  heading=references,
  prenote=ref]
\refstepcounter{ref}\label{ref}

\end{document}
