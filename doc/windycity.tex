% Last modified: Mon 10 Dec 2018 11:51:54 AM CST
\documentclass[11pt,letterpaper,oneside]{article}
\usepackage{windycity}

\begin{document}
\title{Windy City}
\subtitle{A Chicago Style for \biblatex}
\author{Brian Michael Chase}
\email{brianmichaelchase@gmail.com}
\website{https://github.com/brianchase/windycity}
\version{2018.12.09}
\maketitle
\tableofcontents\markboth{Contents}{Contents}

\section{Introduction}

\nfootnote{Copyright \textcopyright\ 2018 Brian Michael Chase. Under
the terms of the \LaTeX\ Project Public License, version 1.3,
permission is granted to copy, distribute, or modify this software.
See \url{http://www.ctan.org/tex-archive/macros/latex/base/lppl.txt}.}

Windy City is a style for \biblatex that formats notes,
bibliographies, parenthetical citations, and reference lists according
to the \textit{The Chicago Manual of Style}
(\textit{CMS}).\footnote{\cite{chicago2017}} It accurately handles a
wide range of citations in different formats and includes a set of
options and commands to accommodate special circumstances. It also has
extensive support for citing and arranging different kinds of editors,
translators, and compilers within a single citation. These features
make Windy City especially suitable for academic work.

The following sections assume familiarity with \textit{CMS} and
\biblatex. Section \ref{overview} gives a brief overview of the
style's features. Section \ref{edtrans} discusses the assignment of
editors, translators, and compilers. Section \ref{multivolume}
discusses several issues with multivolume works. Sections \ref{notes}
and \ref{paren} reproduce examples from various parts of \textit{CMS}
with a focus on notes and bibliographies. It will help to compare the
results throughout these sections with the corresponding information
in this document's bibliography database and \LaTeX\ file,
\file{windycity.bib} and \file{windycity.tex}, respectively.

Aside from \biber, which is necessary for a few options, Windy City
has no requirements beyond those of \biblatex version 3.8 or later.

\section{Overview}
\label{overview}

This section covers basic information about Windy City. If you're
completely new to \biblatex, you should probably glance at its
documentation and try one of the styles that come with it, if only to
get a sense of the basic commands. For the impatient, examples in
Sections \ref{default}, \ref{short}, \ref{notes}, and \ref{paren}
might be of more immediate interest.

\subsection{Getting Started}

If you already know how to use \biblatex, getting started with Windy
City is easy. Locate \biblatex on your system, and copy Windy City's
files into their respective directories:

\begin{itemize}[label=]
\item \ldots\path{/biblatex/windycity.dbx}
\item \ldots\path{/biblatex/bbx/windycity.bbx}
\item \ldots\path{/biblatex/cbx/windycity.cbx}
\item \ldots\path{/biblatex/lbx/american-windycity.lbx}
\end{itemize}

\noindent Next, tell \biblatex to load Windy City with the load-time
option \opt{style}:

\begin{verbatim}
   \usepackage[style=windycity]{biblatex}
\end{verbatim}

For some entries in your bibliography database, you may need to add
fields or make other adjustments to get the right output. However,
since Windy City relies as much as possible on standard \BibTeX
fields, and secondarily on \biblatex fields, you may not need to make
major changes. The examples in this document and its accompanying
bibliography database, \file{windycity.bib}, should serve as a guide
for how to manage your input for nearly every circumstance that the
style is meant to handle.

\subsection{The Default Format}
\label{default}

For a first set of examples, consider this passage from \textit{CMS}
\ref{14.30}:

\begin{citeonly}
\item \cite[24--25]{morley1995}
\item \cite{schwartz1992}
\item \cite{kaiser1964}
\item \cite[43]{morley1995}
\item \cite[138]{schwartz1992}
\item \cite[189--90]{kaiser1964}
\end{citeonly}

The output shows Windy City's default format. The first citation of a
work is similar to its entry in the bibliography. It includes all or
most of its bibliographic information. Subsequent citations are
shorter, usually consisting of a shortened form of the author's name
and a shortened form of the work's title.

Windy City supports variations on this format. For information on
short forms of citation, including the use of \textit{ibid.}, see
Section \ref{short}. For options to skip parts of citations, change
the order of editors and translators, and more, see Sections
\ref{preops} and \ref{entryops}. For parenthetical citations, see
examples in Section \ref{paren}.

The block below shows Windy City's default bibliography for the
previously cited works:

\begin{bibonly}
\nocite{kaiser1964,morley1995,schwartz1992}
\end{bibonly}

\noindent You may also print a bibliography in an author-date format,
what \textit{CMS} calls a reference list:

\begin{refonly}
\nocite{kaiser1964,morley1995,schwartz1992}
\end{refonly}

To make every instance of \cmd{printbibliography} use the author-date
format, load \biblatex with the preamble option \opt{reflist}:

\begin{verbatim}
   \usepackage[reflist,style=windycity]{biblatex}
\end{verbatim}

\noindent Note that \opt{reflist=true} has the same effect:

\begin{verbatim}
   \usepackage[reflist=true,style=windycity]{biblatex}
\end{verbatim}

To use the author-date format on a case-by-case basis, run
\cmd{printbibliography} with an appropriate \opt{env} option. In order
to use Windy City's author-date format, a so-called ``bib
environment'' must set the style's internal \opt{reflist} toggle to
\opt{true}. Windy City's own such environment is called \opt{reflist},
which you may use as follows:

\begin{verbatim}
   \printbibliography[env=reflist]
\end{verbatim}

As you proceed through the text, note that all examples of citations
and bibliographies are outputs of the style from commands that you can
inspect in the document's source, \file{windycity.tex}, and in its
style file, \file{windycity.sty}. Almost all citations are from
\cmd{cite} or \cmd{parencite}. A few are from more specialized
commands, such as \cmd{cite*} or \cmd{cites}. All example
bibliographies are outputs of the style from \cmd{printbibliography}.
All bibliographic data resides in \file{windycity.bib}.

\subsection{Short Citations}
\label{short}

Aside from parenthetical citations, which \textit{CMS} covers in
Chapter 15, alternative formats receive scant documentation.
Nevertheless, \textit{CMS} does give options. Consider this example
from \textit{CMS} \ref{14.34}:\footnote{Switching formats within a
document isn't a feature of the style and isn't at all convenient.}

\begin{citeonly}
\AtNextCitekey{\toggletrue{short}\toggletrue{firstshort}}
\item \cite[3]{morrison2004a}
\AtNextCitekey{\toggletrue{short}}
\item \cite[18]{morrison2004a}
\AtNextCitekey{\toggletrue{short}}
\item \cite[18]{morrison2004a}
\AtNextCitekey{\toggletrue{short}}
\item \cite[24--26]{morrison2004a}
\AtNextCitekey{\toggletrue{short}\toggletrue{firstshort}}
\item \cite[401-2]{morrison2004b}
\AtNextCitekey{\toggletrue{short}}
\item \cite[433]{morrison2004b}
\AtNextCitekey{\toggletrue{short}\toggletrue{firstshort}}
\item \cite[37--38]{diaz2008}
\AtNextCitekey{\toggletrue{short}}
\item \cite[403]{morrison2004b}
\AtNextCitekey{\toggletrue{short}}
\item \cite[152]{diaz2008}
\AtNextCitekey{\toggletrue{short}}
\item \cite[201-2]{diaz2008}
\AtNextMultiCite{\toggletrue{short}}
\item \cites[240]{morrison2004b}[32]{morrison2004a}
\AtNextCitekey{\toggletrue{short}}
\item \cite[33]{morrison2004a}
\end{citeonly}

Compare that with the style's default output:

\begin{citeonly}
\item \cite[3]{morrison2004a}
\item \cite[18]{morrison2004a}
\item \cite[18]{morrison2004a}
\item \cite[24--26]{morrison2004a}
\item \cite[401-2]{morrison2004b}
\item \cite[433]{morrison2004b}
\item \cite[37--38]{diaz2008}
\item \cite[403]{morrison2004b}
\item \cite[152]{diaz2008}
\item \cite[201-2]{diaz2008}
\item \cites[240]{morrison2004b}[32]{morrison2004a}
\item \cite[33]{morrison2004a}
\end{citeonly}

In the short format, a work's first citation is much shorter than in
the default, and consecutive citations of a work may omit the title.
To use this format, start \biblatex with the preamble option
\opt{short}. See Section \ref{preops} for more information.

\textit{CMS} \ref{14.34} also shows how to render the passage with
\textit{ibid.} Unlike previous editions of \textit{CMS}, the 17th
edition discourages its use. As such, \textit{ibid.} is no longer part
of Windy City's default format. Enable it with the preamble option
\opt{ibid} (again, see Section \ref{preops}). The combination of
options \opt{short} and \opt{ibid} yield the following:

\begin{citeonly}
\AtNextCitekey{\toggletrue{short}\toggletrue{firstshort}\toggletrue{ibid}}
\item \cite[3]{morrison2004a}
\AtNextCitekey{\toggletrue{short}\toggletrue{ibid}}
\item \cite[18]{morrison2004a}
\AtNextCitekey{\toggletrue{short}\toggletrue{ibid}}
\item \cite[18]{morrison2004a}
\AtNextCitekey{\toggletrue{short}\toggletrue{ibid}}
\item \cite[24--26]{morrison2004a}
\AtNextCitekey{\toggletrue{short}\toggletrue{firstshort}\toggletrue{ibid}}
\item \cite[401-2]{morrison2004b}
\AtNextCitekey{\toggletrue{short}\toggletrue{ibid}}
\item \cite[433]{morrison2004b}
\AtNextCitekey{\toggletrue{short}\toggletrue{firstshort}\toggletrue{ibid}}
\item \cite[37--38]{diaz2008}
\AtNextCitekey{\toggletrue{short}\toggletrue{ibid}}
\item \cite[403]{morrison2004b}
\AtNextCitekey{\toggletrue{short}\toggletrue{ibid}}
\item \cite[152]{diaz2008}
\AtNextCitekey{\toggletrue{short}\toggletrue{ibid}}
\item \cite[201-2]{diaz2008}
\AtNextMultiCite{\toggletrue{short}\toggletrue{firstshort}}
\item \cites[240]{morrison2004b}[32]{morrison2004a}
\AtNextCitekey{\toggletrue{short}\toggletrue{ibid}}
\item \cite[33]{morrison2004a}
\end{citeonly}

There are still other ways to save space: With the default format, you
can use the preamble option \opt{firstshort} to swap long first
citations for short ones (see Section \ref{preops}). The entry option
\opt{noauth} prints a citation without the author's position (see
Section \ref{entryops}). And the field \bibfield{shorthand} allows you
to specify an abbreviation to stand in place of the author's name, the
work's title, and other elements that ordinarily appear in citations
(see \ref{14.59}).

In total, Windy City allows a fair amount of control over the format
of notes and bibliographies, consistent with \textit{CMS}. The
remainder of this section, and the next few section thereafter,
describe these and other features of the style in a more systematic
way.

\subsection{Preamble Options}
\label{preops}

A preamble option is an argument for the \cmd{usepackage} macro that
loads \biblatex. Preamble options affect the format of notes,
bibliographies, and reference lists. Some features of the style
require them.

All options described below are \opt{false} by default. Set them to
\opt{true} by passing the name of the option to \biblatex, with or
without an additional \opt{=true}. In other words, using the option
\opt{annotate} as an example, the following are equivalent:

\begin{verbatim}
   \usepackage[annotate,style=windycity]{biblatex}
   \usepackage[annotate=true,style=windycity]{biblatex}
\end{verbatim}

\begin{optionlist}

\optitem[false]{annotate}{\opt{true}, \opt{false}}

\noindent This option is for printing annotated bibliographies.
Annotations will appear in block paragraphs below their associated
entries. To change the spacing between entries and annotations, change
the value of \cmd{bibitemsep}. Store the text of an annotation in the
\bibfield{annotation} field of the work's bibliography database entry.

\optitem[false]{collsonly}{\opt{true}, \opt{false}}

\noindent Citing individual works of a collection adds an entry for
each work to the bibliography. This could result in needless clutter,
especially after citing many volumes of a multivolume work. To exclude
individual works and print only an entry for the whole collection, use
\opt{collsonly}. It has no effect on many \bibtype{incollection}
entries, such as articles in books, which need (or ought to have) a
place in the bibliography, but does filter out chapters of books and
volumes of collections. For discussion of collections and multivolume
works, see Section \ref{multivolume}.

\optitem[false]{firstshort}{\opt{true}, \opt{false}}

\noindent Use this option to print a shortened form of the first
citation of each work. The resulting citation consists mainly of the
author's name and the work's title. According to \textit{CMS}, this
approach is optional for documents with complete
bibliographies.\footnote{\textit{CMS} \ref{14.23}. See also
14.29--14.36.} You may use this option in conjunction with \opt{ibid}.
However, \opt{firstshort} adds nothing to \opt{short}. The latter
implies \opt{firstshort} but goes beyond it by giving a short format
to subsequent citations of a work.

\optitem[false]{ibid}{\opt{true}, \opt{false}}

\noindent This option controls whether consecutive citations of a work
on the same page receive an \textit{ibid}. The qualification ``on the
same page'' means that \textit{ibid.} always refers to a work cited on
the current page without an \textit{ibid.} The latter is not a
requirement of \textit{CMS} but seems reasonable, since it prevents
readers from having to look at another page to determine the referent
of an \textit{ibid.} For an example of its output, see Section
\ref{short} or \ref{14.34}. Recall that, as of the 17th edition,
\textit{CMS} discourages the use of \textit{ibid.} (see \ref{14.34}).

\optitem[false]{isbn}{\opt{true}, \opt{false}}

\noindent Use this option to print ISBNs in bibliographies. A work's
ISBN belongs in the \bibfield{isbn} field of its database entry. With
this option, the style will print ISBNs at the end of every entry in
the bibliography, though before annotations. To print the ISBN of a
particular work, see the \bibfield{isbn} entry option in Section
\ref{entryops}.

\optitem[false]{nolos}{\opt{true}, \opt{false}}

\noindent By default, every work with a \bibfield{shorthand} receives
an entry in the bibliography. If you wish to exclude them, say, to
avoid duplication with the output of \cmd{printshorthands}, use
\opt{nolos}.

\optitem[false]{reflist}{\opt{true}, \opt{false}}

\noindent Use this option to print a reference list, that is, a
bibliography in the author-date format. If you choose parenthetical
citations over notes, you should consider using \opt{reflist} to
maintain consistency with \textit{CMS}. As mentioned in Section
\ref{overview}, this is one of two ways to print reference lists with
Windy City. The other is to use the \opt{env} option of
\cmd{printbibliography}. See Section \ref{overview} for more
information.

\optitem[false]{short}{\opt{true}, \opt{false}}

\noindent As shown in Section \ref{short}, this option prints
citations in a short format (see \textit{CMS} \ref{14.34}). The use of
\opt{short} has one feature in common with \opt{ibid}: Just as an
\textit{ibid.} appears only for consecutive citations of a work on the
same page, and so never refers to a citation on a previous page,
\opt{short} drops the title from consecutive citations of a work on
the same page, never in reference to a citation on a previous page. As
with \textit{ibid.}, this feature isn't required by \textit{CMS}, but
it prevents readers from having to look at a previous page to
determine which title a citation refers to.

In contexts where \opt{short} would drop a title from a citation, but
where no name occupies the author's position, it will print the work's
\bibfield{labeltitle}. This can be a shortened form of the title,
either the title minus the subtitle or the \bibfield{shorttitle}, if
present in the bibliography database. In those situations, the short
format is no different from the default.

As noted earlier, \opt{short} implies \opt{firstshort}, so using them
together has no affect beyond using \opt{short} by itself. However,
recall from Section \ref{short} that you can combine \opt{short} and
\opt{ibid} for even more concise notes.

\end{optionlist}

\subsection{Entry Options}
\label{entryops}

An entry option is a value for the \bibfield{options} field of a
work's database entry. It affects the format of that particular work.
For options that affect the format of every work, see Section
\ref{preops}.

\begin{optionlist}

\optitem[false]{anonauth}{\opt{true}, \opt{false}}

\noindent This option prints the author's name of an anonymously
published work in brackets, as shown in \textit{CMS} \ref{14.79}:

\begin{citebib}
\item \cite{horsley1796}
\end{citebib}

\optitem[false]{anonqauth}{\opt{true}, \opt{false}}

\noindent This option is similar to the previous but adds a question
mark after the author's name, indicating doubt about the authorship.
The following example is from \ref{14.79}:

\begin{citebib}
\item \cite{hawkes1834}
\end{citebib}

\optitem[false]{isbn}{\opt{true}, \opt{false}}

\noindent Use this option to print the ISBN of a particular work in a
bibliography. The ISBN will appear at the end of the work's entry but
before an annotation. To print ISBNs of every work, see the
\bibfield{isbn} preamble option in Section \ref{preops}.

\optitem[false]{noauth}{\opt{true}, \opt{false}}

\noindent This option tells the style to bypass the author's position
of a work in notes and bibliographies. Citations will begin with the
title's position. Below is an example from \textit{CMS} \ref{14.105}:

\begin{citebib}
\item \cite{chaucer1966}
\end{citebib}

\noindent To bypass the author's position in a single note, use
\cmd{cite*} or \cmd{footcite*}. See Section \ref{citecmds}.

\optitem[false]{transfirst}{\opt{true}, \opt{false}}

\noindent According to \textit{CMS}, if a work has both an editor and
a translator, their names should appear in citations in the order in
which they appear on the work's title page (\ref{14.104}). By default,
the style lists editors first. Entries with the option
\bibfield{transfirst} reverse this order: Their translators print
first. If a work's translators and editors are identical, using
\bibfield{transfirst} reverses the order of their roles, say, from
\textit{edited and translated by} to \textit{translated and edited
by}. For more on this option, with examples, see Section
\ref{edtranspos}.

\end{optionlist}

\subsection{Citation Commands}
\label{citecmds}

The most important citation commands are already familiar from
\biblatex:

\begin{ltxsyntax}
\cmditem{cite}[prenote][postnote]{key}
\cmditem{footcite}[prenote][postnote]{key}
\cmditem{nocite}{key}
\cmditem*{nocite}|\{*\}|
\cmditem{parencite}[prenote][postnote]{key}
\end{ltxsyntax}

\noindent Insert notes with \cmd{cite} and \cmd{footcite}. Insert
parenthetical citations with \cmd{parencite}. Use \cmd{nocite} to add
works to bibliographies without citing them in the text. Use
\cmd{nocite} with a comma-separated list of entry keys to add
particular works. Use it with an asterisk to add every work in every
\file{bib} file listed in \cmd{bibliography}.

\begin{ltxsyntax}

\cmditem{bibentry}{key}

This command prints the bibliography entry of a work. Although mainly
for testing, you might use a series of these commands to print a
reading list, as for a syllabus. It has no effect on citation
tracking.

\cmditem{cite*}[prenote][postnote]{key}

Use this command to cite a work without printing anything in the
author's position. It comes in handy when the context of a citation
makes the author's name clear, such as when the name appears in the
work's title. From \textit{CMS} \ref{14.78}:

\begin{citebib}
\item \cite*[233]{franklin1868}
\item \cite*[234]{franklin1868}
\end{citebib}

\cmditem{crossref}{key}

Windy City uses this command internally when cross-reference a
previously cited \bibtype{collection}. You could use it in a text, but
it's only output is a work's \bibfield{labelname} and
\bibfield{labeltitle}, separated by a comma and a space.

\cmditem{footcite*}[prenote][postnote]{key}

Like \cmd{cite*}, this command suppresses the author's position of a
note but otherwise functions like \cmd{footcite}.

\cmditem{fullcite}[prenote][postnote]{key}

Mainly for testing, this command prints the first, long citation of a
work in the default format, regardless of whether the preamble options
\opt{short} or \opt{firstshort} are true. Like \cmd{bibentry}, it has
no effect on citation tracking.

\cmditem{fullcite*}[prenote][postnote]{key}

Like \cmd{fullcite}, this command prints the first, long citation of a
work in the default format but, like \cmd{cite*}, skips the author's
position. Also like \cmd{cite*}, the corresponding entry in the
bibliography retains the author's position. It has no effect on
citation tracking.

\cmditem{parencite*}[prenote][postnote]{key}

Use this command to print a parenthetical citation without the
author's position. The most likely context for this is a sentence in
which the author receives explicit mention. Here's an example from
\textit{CMS} 15.25:

\begin{quote} Fiorina et al. \parencite*{fiorina2005} and Fischer and
Hout \parencite*{fischer2006} reach more or less the same conclusions.
In contrast, Abramowitz and Saunders \parencite*{abramowitz2005}
suggest that the mass public is deeply divided between red states and
blue states and between churchgoers and secular voters. \end{quote}

The source for the passage above contains:

\begin{verbatim}
   \begin{quote} Fiorina et al. \parencite*{fiorina2005}...
   Fischer and Hout \parencite*{fischer2006}... Abramowitz
   and Saunders \parencite*{abramowitz2005}... \end{quote}
\end{verbatim}

\cmditem{refentry}{key}

Similar to \cmd{bibentry}, this command prints the reference list
entry of a work. It, too, has no effect on citation tracking.

\cmditem{reprint}[postnote]{key}

This command precedes a citation with ``Reprinted in'' and then skips
the author's position and the work's title. This is useful for citing
a reprint of a work after citing the original:

\begin{citebib}
\item \cite{frankfurt1969} \reprint[1--10]{frankfurt1988.1}
\end{citebib}

\noindent The output above results from:

\begin{verbatim}
   \cite{frankfurt1969} \reprint[1--10]{frankfurt1988.1}
\end{verbatim}

\end{ltxsyntax}

\subsection{Additional Data Fields}
\label{datafields}

Windy City uses several data fields that are not available with
\biblatex.

\begin{marglist}

\item[bookbooktitle] This field is for the style's internal use. Do
not use it in a bibliography database.

\item[bookyear] Also for the style's internal use. Do not use it in a
bibliography database.

\item[editoraddon] Use this field to include additional editorial
information about a work. In Section \ref{entryops}, an example from
\textit{CMS} \ref{14.105} uses it: In the work's database entry,
\bibfield{editoraddon} contains ``from materials compiled by John M.
Manly and Edith Richert, with the assistance of Lilian J. Redstone et
al.'' This content prints after the editor's position, without
intervening punctuation. Again, in the bibliography, the result is:

\begin{bibonly}
\nocite{chaucer1966}
\end{bibonly}

\item[seriesaddon] This field is for additional information about a
book's series, such as \textit{2nd ser.} and \textit{n.s.}. For
examples, see \ref{14.126} and \ref{14.184}.

\item[shortbooktitle] This field is for the short form of a
\bibfield{booktitle}, just as \bibfield{shorttitle} is for
\bibfield{title}. Nevertheless, its use is internal. You never need to
use it in a bibliography database. Instead, always use
\bibfield{shorttitle}. For more information on how the style handles
titles, see Section \ref{authtitles}.

\item[volumea] Use this field when an entry requires a second volume
number. Very few works are likely to need it. For an example, see the
entry for \bibfield{kierkegaard1987c} in \file{windycity.bib} and the
resulting output in Section \ref{multivolume}.

\item[volumetitle] Use this field when a volume has a title. In the
following example, from \textit{CMS} \ref{14.123}, the
\bibfield{volumetitle} is \textit{1883--1884}:

\begin{citeonly}
\item \cite*[32--33]{james1963.5}
\end{citeonly}

\end{marglist}

\subsection{Data Fields for Authors and Titles}
\label{authtitles}

The style allows for some flexibility over how you designate authors
and titles in a bibliography database. In particular, some standard
data fields are optional: \bibfield{bookauthor}, \bibfield{booktitle},
and \bibfield{booksubtitle}. No bibliography database entry needs
them, though using them causes no harm. As such, you may designate a
book's author, title, and subtitle in one of three ways:

\begin{verbatim}
   ...
   author = {Doe, Jane},
   title = {A Book's Title},
   subtitle = {A Book's Subtitle},
   ...
   bookauthor = {Doe, Jane},
   booktitle = {A Book's Title},
   booksubtitle = {A Book's Subtitle},
   ...
   bookauthor = {Doe, Jane},
   title = {A Book's Title},
   subtitle = {A Book's Subtitle},
   ...
   author = {Doe, Jane},
   booktitle = {A Book's Title},
   booksubtitle = {A Book's Subtitle},
   ...
\end{verbatim}

\noindent Note that a \bibfield{subtitle} must go with a
\bibfield{title} and a \bibfield{booksubtitle} must go with a
\bibfield{booktitle}. Also, a work may cross-reference another while
both retain their own \bibfield{author}, \bibfield{title}, and
\bibfield{subtitle}. For example:

\begin{verbatim}
   @InCollection{doe2014a,
     author = {Doe, Jane},
     title = {An Article's Title},
     subtitle = {An Article's Subtitle},
     crossref = {doe2014}
     ...
   }
   @Collection{doe2014,
     author = {Doe, Jane and Doe, John},
     title = {A Book's Title},
     subtitle = {A Book's Subtitle},
     ...
   }
\end{verbatim}

\noindent Much the same holds for editors and translators. See Section
\ref{edtrans}.

\subsection{Data Fields for Series}

Often, the only question about a book's series is whether to count it
as a series at all, rather than as the title of a multivolume work.
After that, the most common question is how to format a series number.
Sometimes, the number appears alone, with no preceding abbreviation.
But it may also appear with \textit{vol.} or
\textit{no.}\footnote{\textit{CMS} \ref{14.123}}.

In order to avoid using multiple fields for essentially the same task,
the style uses \bibfield{number} for all of them. By default, it
places no abbreviation before the number. Add abbreviations to the
field as necessary. In the database entry for the example below,
\bibfield{number} contains \textit{vol. 6}:

\begin{citebib}
\item \cite{cochrane1987}
\end{citebib}

To indicate the run of a series, such as \textit{2nd ser.} or
\textit{n.s.}, use \bibfield{seriesaddon} (see Section
\ref{datafields}). However, this does not apply to journals, where
such labels usually modify a journal's title. For them, use
\bibfield{series}, as in this example from \ref{14.184}:

\begin{citebib}
\item \cite{moraes1950}
\end{citebib}

\subsection{Entry Types}
\label{entrytypes}

In a bibliography database, every entry has an entry type. The style
recognizes the standard ones for \BibTeX, as well as some that are
specific to \biblatex. A relatively small number of entry types are
basic. The style treats the rest as aliases of the basic ones.

\begin{typelist}
\RaggedRight

\typeitem{article}

Aliases: \bibtype{periodical}

\typeitem{book}

Aliases: \bibtype{booklet}, \bibtype{collection}, \bibtype{manual},
\bibtype{proceedings}, \bibtype{report}, \bibtype{techreport}

\typeitem{incollection}

Aliases: \bibtype{bookinbook}, \bibtype{conference},
\bibtype{inproceedings}, \bibtype{inbook}, \bibtype{letter},
\bibtype{suppbook}, \bibtype{suppcollection}

\typeitem{letter} No aliases
\typeitem{misc} No aliases
\typeitem{online} No aliases
\typeitem{patent} No aliases

\typeitem{reference}

Aliases: \bibtype{inreference}

\typeitem{review} No aliases

\typeitem{thesis}

Aliases: \bibtype{mastersthesis}, \bibtype{phdthesis},
\bibtype{unpublished}

\end{typelist}

\noindent For the most part, you may assign every work to the basic
entry types listed above. A PhD thesis, for example, may have the
entry type \bibtype{thesis} or \bibtype{phdthesis}; the output is the
same. An exception applies to books in collections: Every book in a
one-volume or multivolume collection needs the \bibtype{inbook} or
\bibtype{bookinbook} entry type.\footnote{Using \bibtype{inbook} for
books in collections departs somewhat from the specifications of
\BibTeX\ and \biblatex, though it seems harmless enough.} You may use
either \bibtype{inbook} or \bibtype{bookinbook}. If an entry belongs
to a type other than the ones listed above, the style processes it as
a book.

One comment about \bibtype{reference} and \bibtype{inreference}
entries: You may cross-reference \bibtype{inreference} entries to
\bibtype{reference} entries, as with articles in books, but you can
get the same output using one or the other entry type alone. Take an
example from \textit{CMS} \ref{14.232}:

\begin{citeonly}
\item \cite{salvation1980}
\end{citeonly}

A bibliography database could have an \bibtype{inreference} entry for
the article cross-referenced to a \bibtype{reference} entry for
\textit{Encyclopaedia Britannica}:

\begin{verbatim}
   @InReference{salvation1980,
     title = {salvation},
     crossref = {britannica1980}
   }
   @Reference{britannica1980,
     organization = {{\emph{Encyclopaedia Britannica}}},
     edition = {15},
     year = {1980}
   }
\end{verbatim}

\noindent This approach makes sense if you plan to cite more than one
article from the source.\footnote{Incidentally, reference work's don't
always have titles in italics. As a result, you need to handle it in
your bibliography database.} But you could also have a single entry of
either type with the same data, like this:

\begin{verbatim}
   @Reference{salvation1980,
     organization = {{\emph{Encyclopaedia Britannica}}},
     edition = {15},
     title = {salvation},
     year = {1980}
   }
\end{verbatim}

For unusually complicated ci\-ta\-tions---or those just not supported
by the style---consider using the \bibtype{misc} entry type. The style
formats these entries with a small number of fields but in way that
makes it a fallback for almost anything. The example below is from
\textit{CMS} 14.264:

\begin{citebib}
\item \cite{roosevelt1959}
\end{citebib}

\noindent The database entry for this work contains most of the format
in \bibfield{usera} (for notes) and \bibfield{userb} (for
bibliographies). In \bibfield{title}, the style needs manual
formatting, since works of this type may have titles in italics or
quotation marks. Some trickery with the year helps with formatting the
reference list entry.

\begin{verbatim}
   @Misc{roosevelt1959,
     author = {Roosevelt, Eleanor},
     title = {\mkbibquote{Is America Facing World
              Leadership?}},
     usera = {convocation speech, Ball State Teacher's
              College, May 6, \thefield{year}, radio
              broadcast, reel-to-reel tape, MPEG copy,
              1:12:49},
     userb = {Convocation Speech. Ball State Teacher's
              College. May 6, \thefield{year}. Radio
              broadcast. Reel-to-reel tape. MPEG copy.
              1:12:49},
     url = {http://libx.bsu.edu/cdm4/singleitem/collection/
            ElRoos/id/1},
     year = {1959}
   }
\end{verbatim}

\noindent No other example in this document relies on \bibtype{misc}.

\section{Editors, Translators, and Compilers}
\label{edtrans}

The style offers a wide range of different classifications of editors
and some flexibility over the handling of translators and compilers.
This section explains how to use these features.

\subsection{Basic Placement of Editors, Translators, and Compilers}
\label{edtranspos}

Normally, the style prints editors' names first. However, if
translators are listed first on a work's title page (or in some other
relevant place), you may want to reverse the order and print the
translators' names first. For that, use the entry option
\opt{transfirst} (see Section \ref{entryops}). Compare:

\begin{citebib}
\item \cite{doe2010a}
\item \cite{doe2010b}
\end{citebib}

If a work has no author, but has an editor, the style will print the
name of the editor in the author's position, as in this example from
\textit{CMS} \ref{14.103}:

\begin{citebib}
\item \cite[100]{egan2014}
\end{citebib}

\noindent A similar switch happens if a work has no author but has a
translator:

\begin{citebib}
\item \cite[34]{silverstein1974}
\end{citebib}

Again, the style prints editors' names first, so if a work has no
author but has editors and translators, the style will print the
editors' names in the author's position:

\begin{citebib}
\item \cite{smith2002a}
\end{citebib}

\noindent Use \opt{transfirst} to reverse them:

\begin{citebib}
\item \cite{smith2002b}
\end{citebib}

Similarly, if a work's editors and translators are identical, the
style will print the editors' role first, as in, ``Edited and
translated by\ldots'' Use \opt{transfirst} to reverse them, as in this
example from \textit{CMS} \ref{14.104}:

\begin{citebib}
\item \cite{menchu1999}
\end{citebib}

Since the style treats a compiler as a kind of editor, the comments
above apply to compilers, too: If a work has compilers and
translators, the style will print compilers' names first, unless you
use \opt{transfirst}. For more on compilers, see Section
\ref{edtransnames}.

\subsection{Types of Editors, Translators, and Compilers}
\label{edtransnames}

The style recognizes five kinds of editors, those of a
\bibfield{title}, \bibfield{booktitle}, \bibfield{maintitle},
\bibfield{issuetitle}, or \bibfield{series}. The style recognizes two
kinds of translators, those of a \bibfield{title} or
\bibfield{booktitle}. Although the style treats compilers as a kind of
editor, it recognizes just two types of them, those of a
\bibfield{title} or \bibfield{booktitle}.\footnote{As will become
clear soon, indicate a compiler by giving the value \textit{compiler}
to the appropriate \bibfield{editortype}.}

As confusing as all this may seem, the assignments of editors and
translators are often automatic. The style's default is to associate
\bibfield{editor} and \bibfield{translator} with the lowest level
title within the scope of an entry. For most sources, then, no more
work is necessary than just adding \bibfield{editor} and
\bibfield{translator} fields to a database entry. For example, this
entry shows data for \ref{14.104}:

\begin{verbatim}
   @Book{menchu1999,
     options = {transfirst},
     author = {Menchú, Rigoberta},
     title = {Crossing Borders},
     editor = {Wright, Ann},
     translator = {Wright, Ann},
     address = {New York},
     publisher = {Verso},
     year = {1999}
   }
\end{verbatim}

\noindent When the style processes this entry, it identifies Ann
Wright as the book's translator and editor.\footnote{The entry option
\opt{transfirst} ensures that her role as translator is mentioned
first. See Section \ref{entryops}.}

Cross-referencing introduces a bit more complexity, but the principle
remains the same: Within the scope of an entry, the style associates
\bibfield{editor} and \bibfield{translator} with the lowest level
title. In this example, which shows an essay cross-referenced to a
collection, \bibfield{editor} appears within the collection's database
entry:

\begin{verbatim}
   @InCollection{thoreau2007.7,
     title = {Walking},
     pages = {185--222},
     crossref = {thoreau2007}
   }
   @Collection{thoreau2007,
     author = {Thoreau, Henry David},
     title = {Excursions},
     editor = {Moldenhauer, Joseph J.},
     series = {The Writings of Henry D. Thoreau},
     publisher = {Princeton University Press},
     address = {Princeton, NJ},
     year = {2009}
   }
\end{verbatim}

\begin{citebib}
\item \cite{thoreau2007.7}
\end{citebib}

\noindent As a result, the style identifies Joseph Moldenhauer as the
editor of \textit{Excursions}. If you moved \bibfield{editor} from
\bibtype{collection} to \bibtype{incollection}, the
style would identify him as the editor of ``Walking.'' If instead you
want him as the editor of the series, leave \bibfield{editor} in
\bibtype{collection} but add a line to the entry:

\begin{verbatim}
   editortype = {series},
\end{verbatim}

Other values of \bibfield{editortype} are \textit{maintitle},
\textit{issuetitle}, and \textit{compiler}. So, to add a compiler to
an entry, add the compiler's name to \bibfield{editor}, then add
\bibfield{editortype} with the value \textit{compiler}.\footnote{Of
course, if \bibfield{editor} is occupied, do the same for
\bibfield{editora} and \bibfield{editoratype}, and so on.}

\begin{verbatim}
   @InCollection{orwell2009.17,
     title = {Politics and the English Language},
     pages = {270--86},
     crossref = {orwell2009}
   }
   @Collection{orwell2009,
     author = {Orwell, George},
     title = {All Art is Propaganda},
     subtitle = {Critical Essays},
     editor = {Packer, George},
     editortype = {compiler},
     address = {Boston},
     publisher = {Mariner Books},
     year = {2009}
   }
\end{verbatim}

\begin{citebib}
\item \cite{orwell2009.17}
\end{citebib}

As already mentioned, this way of handling editors and translators
allows you to assign them to particular works in a collection. This is
most often useful for citing translators, as in the following:

\begin{citebib}
\item \cite{petrarca1948}
\end{citebib}

\noindent Hans Nachod translated ``The Ascent of Mont Ventoux,'' among
other works in the collection, but not \textit{every} work in the
entire collection. Thus, \bibfield{translator} must fall within the
scope of \bibtype{incollection}:

\begin{verbatim}
   @InCollection{petrarca1948,
     author = {Petrarca, Francesco},
     title = {The Ascent of Mont Ventoux},
     translator = {Nachod, Hans},
     pages = {36--46},
     crossref = {cassirer1948}
   }
   @Collection{cassirer1948,
     title = {The Renaissance Philosophy of Man},
     editor = {Cassirer, Ernst and Kristeller, Paul Oskar and
               Randall, Jr., John Herman},
     address = {Chicago},
     publisher = ucp,
     year = {1948}
   }
\end{verbatim}

\noindent By the same token, since \bibfield{editor} falls within
\bibtype{collection}, the style assigns it to \textit{The Renaissance
Philosophy of Man}.

One complication remains: There are other name lists for editors than
\bibfield{editor}. The style also uses \bibfield{editora},
\bibfield{editorb}, and \bibfield{editorc}.\footnote{The style also
uses \bibfield{translatora}, though only for internal purposes.}

Designate editors much like you designate titles. Reserve
\bibfield{editor} for the lowest level title in a work that you wish
to cite, say, an essay in a collection or a single volume in a
multivolume work. The next level up, as it were, is for
\bibfield{editora}, followed by \bibfield{editorb}, and so on.
Remember to include the appropriate \bibfield{type} field to indicate
an editor's role. These fields are \bibfield{editortype} (for
\bibfield{editor}), \bibfield{editoratype} (for \bibfield{editora}),
\bibfield{editorbtype} (for \bibfield{editorb}), and
\bibfield{editorctype} (for \bibfield{editorc}). For example:

\begin{citebib}
\item \cite{smith1981}
\end{citebib}

\noindent Since the work above has two sets of editors, the database
entry needs to use \bibfield{editor} and \bibfield{editora}:

\begin{verbatim}
   @Book{smith1981,
     author = {Smith, Adam},
     title = {An Inquiry into the Nature and Causes of the
              Wealth of Nations},
     shorttitle = {The Wealth of Nations},
     editor = {Todd, W. B.},
     volumes = {2},
     volume = {2},
     maintitle = {The Glasgow Edition of the Works and
                  Correspondence of Adam Smith},
     editora = {Campbell, R. H. and Skinner, A. S.},
     editoratype = {maintitle},
     address = {Indianapolis},
     publisher = {Liberty Fund},
     year = {1981},
     origlocation = {Oxford},
     origpublisher = {Oxford University Press},
     origdate = {1976}
   }
\end{verbatim}

\section{Multivolume Works}
\label{multivolume}

Typical examples of \bibtype{collection} and \bibtype{incollection}
entries appear in Section \ref{notes}. But not all such works are
typical. This section addresses some of them.

\subsection{Collections as Single Works}

Although its discussion is a bit obscure, \textit{CMS} treats some
multivolume works as a single work---but only, it seems, if every
volume of the collection has the same title and publication date.
Below are some examples from \ref{14.118}:

\begin{citeonly}
\item \cite[243]{byrne1981.4}
\item \cite*[32--33]{james1963.5}
\item \cite[245]{byrne1981.4}
\item \cite*[34]{james1963.5}
\end{citeonly}

In three of the four notes, the volume numbers and pages are separated
by a colon. The exception is the second note, where the volume number
appears after the editor's name. What makes it different?

Apparently, \textit{The Lisle Letters} counts as a single work because
every volume in the collection has the same title and publication
date. Not so \textit{The Complete Tales of Henry James}. The volumes
of that collection have different titles and publication dates.

How exactly that justifies the difference in format is less clear, but
Windy City accommodates it. Citations to other volumes of \textit{The
Lisle Letters} receive short first citations, as though they were all
citations of the same work, while other volumes of \textit{Henry
James} receive long first citations, as though they were unique works,
not parts of a single work. Let's extend the previous examples:

\begin{citeonly}
\item \cite[243]{byrne1981.4}
\item \cite*[32--33]{james1963.5}
\item \cite[245]{byrne1981.4}
\item \cite*[34]{james1963.5}
\item \cite[91]{byrne1981.5}
\item \cite*[67]{james1962.4}
\item \cite*[106]{james1962.4}
\end{citeonly}

This treatment of multivolume works is consistent across formats.
Switching to \opt{short}, \opt{ibid}, or \opt{firstshort} will change
the appearance of citations but maintain the treatment of \textit{The
Lisle Letters} as a single work and the volumes of \textit{Henry
James} as distinct works.

How does Windy City do it? For \textit{The Lisle Letters}, the
bibliography database contains a \bibtype{collection} entry with
information about the whole collection and separate \bibtype{inbook}
entries for each volume. (This use of \bibtype{inbook} is perhaps a
bit nonstandard but makes sense if you regard the collection as a
single work consisting of books.) The \bibtype{inbook} entries
cross-reference the \bibtype{collection} entry and have the same title
and publication date as the \bibtype{collection}. That's enough for
the style to format them as \textit{CMS} recommends. Meanwhile, the
volumes of \textit{Henry James} have separate \bibtype{book} entries,
with no cross-referencing. Regardless, the publication dates and
titles of the volumes differ, so cross-referencing wouldn't make the
collection into a single work.

It may help to reproduce some entries from this document's
bibliography database. Here are two entries for \textit{The Lisle
Letters}, one for the fourth volume and the one for the whole
collection:

\begin{verbatim}
   @InBook{byrne1981.4,
     title = {The Lisle Letters},
     volume = {4},
     year = {1981},
     crossref = {byrne1981}
   }
   @Collection{byrne1981,
     editor = {Byrne, Muriel St. Clare},
     title = {The Lisle Letters},
     volumes = {6},
     address = {Chicago},
     publisher = ucp,
     year = {1981}
   }
\end{verbatim}

The \bibtype{inbook} entry, \bibfield{byrne1981.4}, contains the
minimum information that Windy City needs to format its citation
correctly. Adding information, such as the publisher, address, and
editor would have no effect on the output.

By contrast, here is the \bibtype{book} entry for a volume of
\textit{Henry James}, along with one for the collection:

\begin{verbatim}
   @Book{james1963.5,
     author = {James, Henry},
     title = {The Complete Tales of Henry James},
     shorttitle = {Complete Tales of Henry James},
     editor = {Edel, Leon},
     volume = {5},
     volumetitle = {1883–1884},
     address = {London},
     publisher = {Rupert Hart-Davis},
     year = {1963}
   }
   @Collection{james1962,
     author = {James, Henry},
     title = {The Complete Tales of Henry James},
     shorttitle = {Complete Tales of Henry James},
     editor = {Edel, Leon},
     volumes = {12},
     address = {London},
     publisher = {Rupert Hart-Davis},
     date = {1962/1964}
   }
\end{verbatim}

Cross-referencing in this case is inadvisable, since not all data for
the collection as a whole is true of particular volumes.

\subsection{Options for Difficult Cases}

Occasionally, getting acceptable output for a multivolume work
requires trial-and-error. Consider a difficult example:

\begin{citebib}
\item \cite{kierkegaard1987}
\end{citebib}

\noindent This edition of \textit{Either/Or} is a two-volume work,
each volume of which is a volume of \textit{Kierkegaard's Writings}.
In the bibliography database, \bibfield{volume} contains ``3 and 4,''
though ordinarily it would contain a single number for a single
volume. The style responds by preceding ``3 and 4'' with the
abbreviation \textit{vols.}, rather than the usual, \textit{vol}.
Meanwhile, \bibfield{volumes} contains \liningnums{2}, indicating the
two volumes of \textit{Either/Or}, and \bibfield{maintitle} contains
\textit{Kierkegaard's Writings}. No cross-referencing occurs with the
entry.\footnote{You could drop \bibfield{volume} for simpler output.
But there are occasions when you need it along with
\bibfield{volumes}, such as when a two-volume work comprises a single
volume of a multivolume collection. This is the case with \textit{The
Wealth of Nations} cited in Section \ref{edtransnames}.}

The greatest difficulty, though, is how to cite particular works from
\textit{Either/Or}. A citation should include the volume of the book.
But each volume is also a volume of \textit{Kierkegaard's Writings},
which makes each volume a volume twice over. Plus, each volume is a
collection, which means that each volume is a collection (consisting
of essays) within a collection (the two-volume book) within a
collection (of \textit{Kierkegaard's Writings}).

The title pages of \textit{Either/Or} might suggest a solution. They
refer to the volumes as \textit{Part I} and \textit{Part II}. But
using \bibfield{part} is incorrect, since, for books, its function is
to indicate parts of a volume, not volumes of a collection. What
should you do?

One solution is to keep a separate entry in the bibliography database
for each volume, without cross-referencing the original entry, and to
use \bibfield{note} for holding ``Part I'' and ``Part II.'' Citing
particular works from each volume is then a straightforward matter of
making \bibtype{incollection} entries for those works and
cross-referencing them to the new \bibtype{collection} entries. The
output is at least adequate, though the placement of \textit{Part I}
may seem odd:

\begin{citebib}
\item \cite[290]{kierkegaard1987.1.8a}
\end{citebib}

\noindent Alternatively, you could cross-reference each volume of
\textit{Either/Or} to the collection, then cross-reference
individual works to their respective volume:

\begin{verbatim}
   @InCollection{kierkegaard1987.1.8b,
     title = {Rotation of Crops},
     pages = {281–300},
     crossref = {kierkegaard1987.1b}
   }
   @InBook{kierkegaard1987.1b,
     title = {Either/Or},
     volume = {1},
     crossref = {kierkegaard1987b}
   }
   @Collection{kierkegaard1987b,
     author = {Kierkegaard, Søren},
     title = {Either/Or},
     editor = {Hong, Howard V. and Hong, Edna H.},
     translator = {Hong, Howard V. and Hong, Edna H.},
     address = {Princeton, NJ},
     publisher = {Princeton University Press},
     year = {1987}
   }
\end{verbatim}

\noindent The entries above drop the \bibfield{note} and
\bibfield{maintitle} and use \bibfield{volume} for the volume of
\textit{Either/Or}. The resulting output is fairly clean:

\begin{citebib}
\item \cite[290]{kierkegaard1987.1.8b}
\end{citebib}

But what if you want to include the \bibfield{maintitle} with its
volume number? Try adding a \bibfield{maintitle} to
\bibtype{collection} along with a new field, \bibfield{volumea}, for
the volume of \textit{Kierkegaard's Writings}. The result:

\begin{citebib}
\item \cite[290]{kierkegaard1987.1.8c}
\end{citebib}

Windy City adds an entry to the bibliography for every work that you
cite from a collection. Generally, the style will add a separate entry
for the collection after you cite two or more works from
it.\footnote{This isn't always true. In some cases, depending on the
parameters of \cmd{printbibliography}, and how the document is divided
into reference sections and segments, bibliographies won't
automatically include a collection that hasn't been explicitly cited.}
If you prefer a different threshold, load \biblatex with a different
value for the preamble option \opt{mincrossrefs}. A value of
\liningnums{1}, for example, should add an entry for the collection if
you cite from it at least once.

\section{Examples from \emph{CMS} Chap. 14, ``Notes and
Bibliography''}
\label{notes}

Examples in this section reproduce those in \textit{CMS} Chapter 14.
To help with cross-checking, subsection numbers and headings are from
\textit{CMS}.

\subsection{Basic Format, With Examples and Variations}
\setcounter{subsection}{14}

\setcounter{subsubsection}{22}
\subsubsection{Notes and bibliography—examples and variations}
% 14.23 Notes and bibliography—examples and variations
\label{14.23}

\begin{citebib}
\item \cite[87-88]{strayed2012}
\item \cite[261, 265]{strayed2012}
\item \cite[32]{daum2015}
\item \cite[134--35]{daum2015}
\item \cite[188]{grazer2015}
\item \cite[190]{grazer2015}
\item \cite[242--55]{garcia1988}
\item \cite[33]{garcia1988}
\item \cite[310]{gould1984a}
\item \cite[309]{gould1984a}
\item \cite[484--85]{bagley2015}
\item \cite[501]{bagley2015}
\item \cite[311]{liu2015}
\item \cite[312]{liu2015}
\end{citebib}

\setcounter{subsection}{1}
\subsection{Notes}
\setcounter{subsection}{14}

\setcounter{subsubsection}{29}
\subsubsection{Basic structure of the short form}
% 14.30: Basic structure of the short form
\label{14.30}

\begin{citebib}
\item \cite[24--25]{morley1995}
\item \cite{schwartz1992}
\item \cite{kaiser1964}
\item \cite[43]{morley1995}
\item \cite[138]{schwartz1992}
\item \cite[189--90]{kaiser1964}
\end{citebib}

\setcounter{subsubsection}{33}
\subsubsection{Shortened citations versus ``ibid''}
% 14.34:
\label{14.34}

See Section \ref{short} for a discussion of how to enable the short
format and the use of \textit{ibid.} First, the short format:

\begin{citeonly}
\AtNextCitekey{\toggletrue{short}\toggletrue{firstshort}}
\item \cite[3]{morrison2004a}
\AtNextCitekey{\toggletrue{short}}
\item \cite[18]{morrison2004a}
\AtNextCitekey{\toggletrue{short}}
\item \cite[18]{morrison2004a}
\AtNextCitekey{\toggletrue{short}}
\item \cite[24--26]{morrison2004a}
\AtNextCitekey{\toggletrue{short}\toggletrue{firstshort}}
\item \cite[401-2]{morrison2004b}
\AtNextCitekey{\toggletrue{short}}
\item \cite[433]{morrison2004b}
\AtNextCitekey{\toggletrue{short}\toggletrue{firstshort}}
\item \cite[37--38]{diaz2008}
\AtNextCitekey{\toggletrue{short}}
\item \cite[403]{morrison2004b}
\AtNextCitekey{\toggletrue{short}}
\item \cite[152]{diaz2008}
\AtNextCitekey{\toggletrue{short}}
\item \cite[201-2]{diaz2008}
\AtNextMultiCite{\toggletrue{short}}
\item \cites[240]{morrison2004b}[32]{morrison2004a}
\AtNextCitekey{\toggletrue{short}} \item \cite[33]{morrison2004a}
\end{citeonly}

\noindent With \textit{ibid.}:

\begin{citeonly}
\AtNextCitekey{\toggletrue{short}\toggletrue{firstshort}\toggletrue{ibid}}
\item \cite[3]{morrison2004a}
\AtNextCitekey{\toggletrue{short}\toggletrue{ibid}}
\item \cite[18]{morrison2004a}
\AtNextCitekey{\toggletrue{short}\toggletrue{ibid}}
\item \cite[18]{morrison2004a}
\AtNextCitekey{\toggletrue{short}\toggletrue{ibid}}
\item \cite[24--26]{morrison2004a}
\AtNextCitekey{\toggletrue{short}\toggletrue{firstshort}\toggletrue{ibid}}
\item \cite[401-2]{morrison2004b}
\AtNextCitekey{\toggletrue{short}\toggletrue{ibid}}
\item \cite[433]{morrison2004b}
\AtNextCitekey{\toggletrue{short}\toggletrue{firstshort}\toggletrue{ibid}}
\item \cite[37--38]{diaz2008}
\AtNextCitekey{\toggletrue{short}\toggletrue{ibid}}
\item \cite[403]{morrison2004b}
\AtNextCitekey{\toggletrue{short}\toggletrue{ibid}}
\item \cite[152]{diaz2008}
\AtNextCitekey{\toggletrue{short}\toggletrue{ibid}}
\item \cite[201-2]{diaz2008}
\AtNextMultiCite{\toggletrue{short}\toggletrue{firstshort}}
\item \cites[240]{morrison2004b}[32]{morrison2004a}
\AtNextCitekey{\toggletrue{short}\toggletrue{ibid}}
\item \cite[33]{morrison2004a}
\end{citeonly}

\setcounter{subsubsection}{58}
\subsubsection{Abbreviations for frequently cited works}
% 14.59 Abbreviations for frequently cited works
\label{14.59}

The only way for the style to match \textit{CMS} on these citations is
by citing a \bibtype{collection} entry with the volume and page
numbers in the \bibfield{postnote} field, as in ``1:126.''

You may override the default announcement of a \bibfield{shorthand},
and format it however you like, by adding your preferred content to
\bibfield{shorthandintro}.

\begin{citebib}
% Better to use furet199 in 14.99, without shorthand:
%\item \cite[368]{furet1999}
% FIX: Explain how you get these examples!
\item \cite[1:126]{shurtleff1853}
\item \cite[2:330]{shurtleff1853}
\end{citebib}

\setcounter{subsection}{2}
\subsection{Author's Name}
\setcounter{subsection}{14}

\setcounter{subsubsection}{74}
\subsubsection{One author}
% 14.75 One author

\begin{citebib}
\item \cite[33]{shields2013}
\item \cite[677]{chun2015}
\item \cite[5]{mccune2014}
\item \cite[100--101]{shields2013}
\item \cite[681]{chun2015}
\item \cite[105--11]{mccune2014}
\end{citebib}

\subsubsection{Two or more authors (or editors)}
% 14.76 Two or more authors (or editors)
\label{14.76}

\begin{citebib}
\item \cite[xvi]{sorrells2015}
\item \cite[20--21]{levitt2005}
\item \cite[422]{umbers2015}
\item \cite[xx-xxi]{sorrells2015}
\item \cite[158]{gmuca2015}
\item \cite[160]{gmuca2015}
\end{citebib}

\subsubsection{Two or more authors (or editors) with same family name}
% 14.77 Two or more authors (or editors) with same family name

\begin{citebib}
\item \cite[14]{kendris2010}
\item \cite[27--28]{kendris2010}
\end{citebib}

\subsubsection{Author's name in title}
% 14.78 Author's name in title
\label{14.78}

\begin{citebib}
\item \cite*[233]{franklin1868}
\item \cite*[234]{franklin1868}
\end{citebib}

\subsubsection{No listed author (anonymous works)}
% 14.79 No listed author (anonymous works)
\label{14.79}

See Section \ref{entryops} on the \opt{anonauth} and \opt{anonauthq}
entry options.

\begin{citebib}
\item \cite{anon1610}
\item \cite{anon1547}
\item \cite{horsley1796}
\item \cite{hawkes1834}
\end{citebib}

\subsubsection{Pseudonyms}
% 14.80 Pseudonyms

To print the author's real name in brackets after a pseudonym, use the
\bibfield{nameaddon} field. If the real name is unknown, and you want
to indicate that a name is a pseudonym, put 'pseud.' in
\bibfield{nameaddon}.

\begin{citebib}
\item \cite{carre1982}
\item \cite{stendhal1925}
\end{citebib}

\setcounter{subsubsection}{82}
\subsubsection{Authors known by a given name}
% 14.83 Authors known by a given name

\begin{citebib}
\item \cite{elizabeth2000}
\end{citebib}

\subsubsection{Organization as author}
% 14.84 Organization as author

If an organization is the work's author, remember to add an extra pair
of brackets around the name of the organization in your bibliography
database.

\begin{citebib}
\AtNextCitekey{\clearfield{shorthand}}
\item \cite{chicago2017}
\item \cite{iso1997}
\end{citebib}

\setcounter{subsection}{3}
\subsection{Title of Work}
\setcounter{subsection}{14}

\setcounter{subsubsection}{88}
\subsubsection{Subtitles in cited works and the use of the colon}
% 14.89 Subtitles in cited works and the use of the colon

\begin{citebib}
\item \cite{gladwell2013}
\end{citebib}

\subsubsection{Two subtitles in a cited work}
% 14.90 Two subtitles in a cited work

\begin{citebib}
\item \cite{sereny1999}
\end{citebib}

\setcounter{subsubsection}{91}
\subsubsection{``And other stories'' and such}
% 14.92 ``And other stories'' and such

\begin{citebib}
\item \cite[104]{maclean1976}
\end{citebib}

\subsubsection{Dates in titles of cited works}
% 14.93 Dates in titles of cited works

\begin{citebib}
\item \cite{beiser2014}
\end{citebib}

\subsubsection{Quoted titles and other terms within cited titles of works}
% 14.94 Quoted titles and other terms within cited titles of works

\begin{citebib}
\item \cite{levitt2005}
\item \cite{mchugh1980}
\end{citebib}

\subsubsection{Italicized titles and other terms within cited titles of works}
% 14.95 Italicized titles and other terms within cited titles of works

\begin{citebib}
\item \cite{vanwagenen1973}
\end{citebib}

\subsubsection{Question marks or exclamation points in titles of cited works}
% 14.96 Question marks or exclamation points in titles of cited works

\begin{citebib}
\item \cite[63]{berra2002}
\item \cite[183]{oram2007}
\item \cite[778]{tessler2014}
\item \cite[336]{batson1990}
\item \cite[55--56]{berra2002}
\item \cite[184]{oram2007}
\item \cite[780]{tessler2014}
\item \cite[337]{batson1990}
\end{citebib}

\setcounter{subsubsection}{98}
\subsubsection{Translated titles of cited works}
% 14.99 Translated titles of cited works

\begin{citebib}
% Gives a comma before postnote, not a semicolon as in CMS:
%\item \cite[includes a summary in German]{wereszycki1977}
\item \cite[272]{kern1938}
\item \cite{pirumova1977b}
\item \cite{furet1999}
\end{citebib}

\setcounter{subsection}{4}
\subsection{Books}
\setcounter{subsection}{14}

\setcounter{subsubsection}{100}
\subsubsection{Form of author’s name and title of book in source citations}
% 14.101 Form of author’s name and title of book in source citations

\begin{citebib}
\item \cite[79--80]{gawande2014}
\item \cite[191]{gawande2014}
\end{citebib}

\setcounter{subsubsection}{102}
\subsubsection{Editor in place of author}
% 14.103 Editor in place of author
\label{14.103}

\begin{citebib}
\item \cite[100]{egan2014}
\item \cite[33]{schechter2011}
\item \cite[34]{silverstein1974}
\item \cite[301--2]{egan2014}
\item \cite[54--56]{schechter2011}
\item \cite[38]{silverstein1974}
\end{citebib}

\subsubsection{Editor or translator in addition to author}
% 14.104 Editor or translator in addition to author
\label{14.104}

\begin{citebib}
\item \cite{bonnefoy1995}
\item \cite{menchu1999}
\item \cite{adorno1999}
% FIX: Requires a new option!
%\item \cite{pound1953}
\end{citebib}

\subsubsection{Other contributors listed on the title page}
% 14.105 Other contributors listed on the title page
\label{14.105}

\begin{citebib}
\item \cite{chaucer1966}
\item \cite{cullen1961}
\item \cite{hayek1994}
\item \cite{prather1998}
\item \cite{williams1990}
\end{citebib}

\subsubsection{Chapter in a single-author book}
% 14.106 Chapter in a single-author book

\begin{citebib}
\item \cite[211]{brower2015.8}
\item \cite{samples2006.7}
\item \cite[30-31]{samples2006.7}
\end{citebib}

\subsubsection{Contribution to a multiauthor book}
% 14.107 Contribution to a multiauthor book

\begin{citebib}
\item \cite[325]{miller2014}
\item \cite{ellet1968}
\end{citebib}

\subsubsection{Several contributions to the same multiauthor book}
% 14.108 Several contributions to the same multiauthor book

\begin{citebib}
\item \cite[84--87]{keating1968}
\item \cite[362--70]{lippincott1968}
\item \cite{draper1987}
\item \cite{harrington1987}
\item \cite{zukowsky1987}
\end{citebib}

\subsubsection{Book-length work within a book}
% 14.109 Book-length work within a book

\begin{citebib}
\item \cite{bernard1990a}
\item \cite{updike1995a}
\end{citebib}

\subsubsection{Introductions, prefaces, afterwords, and the like}
% 14.110 Introductions, prefaces, afterwords, and the like

\begin{citebib}
\item \cite{morrison2004b.f}
\item \cite{mansfield2000}
\end{citebib}

\subsubsection{Letters in published collections}
% 14.111 Letters in published collections

\begin{citebib}
\item \cite[133--34]{adams1867}
\item \cite{jackson1676}
\end{citebib}

\setcounter{subsubsection}{112}
\subsubsection{Editions other than the first}
% 14.113 Editions other than the first

\begin{citebib}
\item \cite[401--2]{einsohn2011}
\item \cite[101]{boudett2013}
\item \cite{strunk2000}
\end{citebib}

\subsubsection{Reprint editions and modern editions}
% 14.114 Reprint editions and modern editions

% You can have at most one \bibfield{origdate} per entry. So, if the
% citation is to a work in a collection, say, an article or book in an
% anthology, the style assumes that \bibfield{origdate} is for the
% collection, not for the individual work.

\begin{citebib}
\item \cite[152--53]{barzun1994}
\item \cite{bahadur2014}
\item \cite{emerson1985}
\item \cite{schweitzer1966}
\end{citebib}

\subsubsection{Microform editions}
% 14.115 Microform editions

The citation of Farwell comes close to \textit{CMS} but isn't an exact
match. The problem is with the field \bibfield{howpublished}, which
seems like the best choice to contain ``microfiche'' but which, as
must happen in other cases, prints after the \bibfield{postnote}, ``p.
67, 3C12.''

\begin{citebib}
\item \cite[p. 67, 3C12]{farwell1997}
\item \cite{tauber1958}
\end{citebib}

\setcounter{subsubsection}{116}
\subsubsection{Citing a multivolume work as a whole}
% 14.117 Citing a multivolume work as a whole

\begin{citebib}
\item \cite{aristotle1983}
\item \cite{byrne1981}
\item \cite{james1962}
\end{citebib}

\subsubsection{Citing a particular volume in a note}
% 14.118 Citing a particular volume in a note
\label{14.118}

\begin{citebib}
\item \cite[243]{byrne1981.4}
\item \cite*[32--33]{james1963.5}
\item \cite[245]{byrne1981.4}
\item \cite*[34]{james1963.5}
\end{citebib}

\subsubsection{Citing a particular volume in a bibliography}
% 14.119 Citing a particular volume in a bibliography

\begin{citebib}
\item \cite{armstrong1992}
\end{citebib}

%\subsubsection{Chapters and other parts of individual volumes}
% 14.120 Chapters and other parts of individual volumes

%\begin{citebib}
% Error in book: In bib entry, 'ed.' should be 'edited by'.
% FIX: Has CMS moved away from the format that would make the volume
% and page reference here '3:181--200' instead of '181--200. Vol. 3',
% as in the book?
%\item \cite[180]{chen2010.3}
%\end{citebib}

\setcounter{subsubsection}{120}
\subsubsection{One volume in two or more books}
% 14.121 One volume in two or more books

\begin{citebib}
\item \cite[351]{lach1977}
\item \cite{harley1994}
\end{citebib}

\subsubsection{Authors and editors of multivolume works}
% 14.122 Authors and editors of multivolume works

\begin{citebib}
\item \cite{barrows1959}
\item \cite{donne1995}
\end{citebib}

\subsubsection{Series titles, numbers, and editors}
% 14.123 Series titles, numbers, and editors
\label{14.123}

\begin{citebib}
\item \cite{lei2014}
\item \cite{mazrim2011}
\item \cite{wauchope1950}
\item \cite{allen2009}
\end{citebib}

\subsubsection{Series or multivolume work?}
% 14.124 Series or multivolume work?

\begin{citebib}
\item \cite{boyer1986}
\item \cite{cochrane1987}
\end{citebib}

\setcounter{subsubsection}{125}
\subsubsection{``Old series'' and ``new series''}
% 14.126 ``Old series'' and ``new series''
\label{14.126}

\begin{citebib}
\item \cite{boxer1953}
\item \cite{palmatary1950}
\end{citebib}

\subsubsection{Place, publisher, and date}
% 14.127 Place, publisher, and date

\begin{citebib}
\item \cite{woolf1927}
\end{citebib}

\subsubsection{Place and date only, for books published before 1900}
% 14.128 Place and date only, for books published before 1900

\begin{citebib}
\item \cite{goldsmith1766}
\item \cite{cervantes1605}
\end{citebib}

\setcounter{subsubsection}{131}
\subsubsection{No place of publication}
% 14.132 No place of publication

\begin{citebib}
\item \cite{windsor1910}
\item \cite{vliet1890}
\end{citebib}

\setcounter{subsubsection}{136}
\subsubsection{Self-published or privately published books}
% 14.137 Self-published or privately published books

\begin{citebib}
\item \cite{karavaev2015}
\item \cite{shumaker2014}
\end{citebib}

\setcounter{subsubsection}{139}
\subsubsection{Copublication}
% 14.140 Copublication

\begin{citebib}
\item \cite{strauss1962}
\end{citebib}

\subsubsection{Distributed books}
% 14.141 Distributed books

\begin{citebib}
\item \cite{willke2007}
\end{citebib}

\setcounter{subsubsection}{143}
\subsubsection{Multivolume works published over more than one year}
% 14.144 Multivolume works published over more than one year

\begin{citebib}
\item \cite[329]{hayek1988}
\item \cite{tillich1951}
\end{citebib}

\subsubsection{No date of publication}
% 14.145 No date of publication

\begin{citebib}
\item \cite{boston}
\item \cite{edinburgh1750}
\item \cite{edinburgh}
\end{citebib}

\subsubsection{Forthcoming publications}
% 14.146 Forthcoming publications

\begin{citebib}
\item \cite{author}
\item \cite[345--46]{writer}
\item \cite{contributor}
\end{citebib}

\setcounter{subsubsection}{158}
\subsubsection{Books requiring a specific application or device (e-books)}
% 14.159 Books requiring a specific application or device (e-books)

\begin{citebib}
\item \cite{borel2015}
\end{citebib}

\setcounter{subsubsection}{160}
\subsubsection{Books consulted online}
% 14.161 Books consulted online

\begin{citebib}
\item \cite[59]{lystra2004}
\item \cite[60--61]{lystra2004}
\end{citebib}

\subsubsection{Freely available electronic editions of older works}
% 14.162 Freely available electronic editions of older works

\begin{citebib}
\item \cite{james2008}
\end{citebib}

\subsubsection{Books on CD-ROM and other fixed media}
% 14.163 Books on CD-ROM and other fixed media

\begin{citebib}
\item \cite[1.4]{chicago2003}
\end{citebib}

\setcounter{subsection}{5}
\subsection{Periodicals}
\setcounter{subsection}{14}

\setcounter{subsubsection}{170}
\subsubsection{Journal volume, issue, and date}
% 14.171 Journal volume, issue, and date

% Errors in book: The note to harper2015 includes the month of
% publication, which the bibliography example omits. The example
% bibliography also gets Lock's name wrong.

\begin{citebib}
\item \cite[155]{lock2015}
\item \cite[651]{wesoky2015}
\item \cite[645]{harper2014}
\item \cite[60]{wilder2013}
\item \cite[52]{beattie1974}
\end{citebib}

\subsubsection{Forthcoming journal articles}
% 14.172 Forthcoming journal articles

\begin{citebib}
\item \cite{authora}
\end{citebib}

\setcounter{subsubsection}{173}
\subsubsection{Journal page references}
% 14.174 Journal page references

\begin{citebib}
\item \cite{gold2015}
\item \cite[2--3]{paudyal2015}
\end{citebib}

\subsubsection{Journal articles consulted online}
% 14.175 Journal articles consulted online

\begin{citebib}
\item \cite[268]{whitney1929}
\item \cite[260--61]{schoenfield2016}
\end{citebib}

\subsubsection{Access dates for journal articles}
% 14.176 Access dates for journal articles

\begin{citebib}
\item \cite[81]{narr2015}
\item \cite[88--89]{narr2015}
\end{citebib}

\setcounter{subsubsection}{177}
\subsubsection{Journal special issues}
% 14.178 Journal special issues

\begin{citebib}
\item \cite[351--81]{tezuka2013}
\end{citebib}

\setcounter{subsubsection}{179}
\subsubsection{Articles published in installments}
% 14.180 Articles published in installments

\begin{citebib}
\item \cite[312]{brown1978}
\end{citebib}

\setcounter{subsubsection}{181}
\subsubsection{Place where journal is published}
% 14.182 Place where journal is published

\begin{citebib}
\item \cite[65--70]{luu1999}
\item \cite{garrett1975}
\end{citebib}

\subsubsection{Translated or edited article}
% 14.183 Translated or edited article

\begin{citebib}
\item \cite{authorb}
\item \cite{authorc}
\end{citebib}

\subsubsection{New series for journal volumes}
% 14.184 New series for journal volumes
\label{14.184}

\begin{citebib}
\item \cite[414]{sewall1896}
\item \cite{moraes1950}
\end{citebib}

\subsubsection{Short titles for articles}
% 14.185 Short titles for articles

\begin{citebib}
\item \cite[223]{rosenblum2015}
\end{citebib}

\subsubsection{Abstracts}
% 14.186 Abstracts

\begin{citebib}
\item \cite{matute2015}
\end{citebib}

\setcounter{subsubsection}{187}
\subsubsection{Basic citation format for magazine articles}
% 14.188 Basic citation format for magazine articles

\begin{citebib}
\item \cite[48]{saulnier2008}
\item \cite[59]{lepore2015}
\end{citebib}

\subsubsection{Magazine articles consulted online}
% 14.189 Magazine articles consulted online

\begin{citebib}
\item \cite{vick2015}
\item \cite[5]{hanemann1926}
\end{citebib}

\subsubsection{Magazine departments}
% 14.190 Magazine departments

\begin{citebib}
\item \cite{marx2015}
\item \cite{wallraff2008}
\item \cite{gourmet2000}
\end{citebib}

\subsubsection{Basic citation format for newspaper articles}
% 14.191 Basic citation format for newspaper articles

\begin{citebib}
\item \cite{editorial1990}
\item \cite{royko1992}
\item \cite{forester2000}
\item \cite{samenow2016}
\end{citebib}

\setcounter{subsubsection}{194}
\subsubsection{Regular columns or features}
% 14.195 Regular columns or features

\begin{citebib}
\item \cite{jaffe2015}
\end{citebib}

\setcounter{subsubsection}{196}
\subsubsection{Weekend supplements, magazines, and the like}
% 14.197 Weekend supplements, magazines, and the like

\begin{citebib}
\item \cite[48]{ghansah2015}
\end{citebib}

\setcounter{subsubsection}{198}
\subsubsection{Unsigned newspaper articles}
% 14.199 Unsigned newspaper articles

\begin{citebib}
\item \cite{nytimes2002}
\end{citebib}

\subsubsection{News services and news releases}
% 14.200 News services and news releases

\begin{citebib}
\item \cite{ap2015}
\end{citebib}

\setcounter{subsubsection}{201}
\subsubsection{Book reviews}
% 14.202 Book reviews

\begin{citebib}
\item \cite[B13--B14]{ratliff1999}
\item \cite{brehm2015}
\end{citebib}

\setcounter{subsubsection}{203}
\subsubsection{Unsigned reviews}
% 14.204 Unsigned reviews

This example fails to match \textit{CMS}. The date should appear
immediately after the newspaper's title and not in parentheses.
Presumably, \textit{CMS} puts the date first because the date is a
more important part of a magazine's or newspaper's citation.
Nevertheless, the format below is consistent with many other types of
articles.

For reference lists, unsigned reviews have no plausible place for the
publication year but where the default format would put it. The
easiest solution is to give unsigned reviews the same format in
reference lists as in bibliographies.

\begin{citebib}
\item \cite{zeitung1828}
\end{citebib}

\setcounter{subsection}{6}
\subsection{Papers, Contracts, and Reports}
\setcounter{subsection}{14}

\setcounter{subsubsection}{214}
\subsubsection{Theses and dissertations}
% 14.215 Theses and dissertations

\begin{citebib}
\item \cite[59]{vedrashko2006}
\item \cite{choi2008}
\end{citebib}

\setcounter{subsubsection}{216}
\subsubsection{Lectures and papers or posters presented at meetings}
% 14.217 Lectures and papers or posters presented at meetings

\begin{citebib}
\item \cite{hong2015}
\end{citebib}

\subsubsection{Working papers and the like}
% 14.218 Working papers and the like

\begin{citebib}
\item \cite{lucki1980}
\end{citebib}

\setcounter{subsubsection}{219}
\subsubsection{Pamphlets, reports, and the like}
% 14.220 Pamphlets, reports, and the like

\begin{citebib}
\item \cite{lifestyles1996}
\item \cite{mcdonalds2014}
\item \cite[¶2,620]{standardtax1996}
\end{citebib}

\setcounter{subsection}{7}
\subsection{Special Types of References}
\setcounter{subsection}{14}

\setcounter{subsubsection}{231}
\subsubsection{Reference works consulted in physical formats}
% 14.232 Reference works consulted in physical formats
\label{14.232}

Some reference works show full publication information in the same way
as books. Use the \bibtype{book} entry type for them. The first three
citations below are different. They need the \bibtype{reference} or
\bibtype{inreference} entry type. See Section \ref{entrytypes} for
more information. Following the suggestion in \textit{CMS} 14.232,
\bibtype{reference} and \bibtype{inreference} works don't appear in
bibliographies and reference lists.

\begin{citebib}
\item \cite{salvation1980}
\item \cite{hootananny2009}
\item \cite{dab1937}
\item \cite[s.vv. \mkbibquote{police ranks}, \mkbibquote{postal addresses}]{timestyle2003}
\item \cite[6.8.2]{mla2008}
\end{citebib}

\subsubsection{Reference works consulted online}
% 14.233 Reference works consulted online

Like some of the reference works in the previous section, the ones
below need the \bibtype{reference} or \bibtype{inreference} entry
type. As odd as it may seem, but consistent with \textit{CMS}, they,
too, aren't included in bibliographies and reference lists. See
Section \ref{entrytypes} for more information.

\begin{citeonly}
\item \cite{toscanini2016}
\item \cite{cairns2016}
\item \cite{wikipedia2016}
\item \cite{merriam2016}
\end{citeonly}

\subsubsection{Citing individual reference entries by author}
% 14.234 Citing individual reference entries by author

Reference works like the following have the same format as articles in
collections. Otherwise, Windy City doesn't quite format them
correctly.

\begin{citebib}
\item \cite{isaacson2005}
\end{citebib}

\setcounter{subsubsection}{245}
\subsubsection{Citing specific editions of classical references}
% 14.246 Citing specific editions of classical references

\begin{citebib}
\item \cite{epictetus1916}
\end{citebib}

\setcounter{subsubsection}{250}
\subsubsection{Modern editions of the classics}
% 14.251 Modern editions of the classics

\begin{citebib}
\item \cite{aristotle1983}
\item \cite{maimonides1965}
\end{citebib}

\setcounter{subsubsection}{257}
\subsubsection{Patents}
% 14.258 Patents

\begin{citebib}
\item \cite{iizuka1986}
\end{citebib}

%\setcounter{subsubsection}{259}
%\subsubsection{Citations taken from secondary sources}
% 14.260 Citations taken from secondary sources

%\begin{citebib}
% FIX: can't override period at the end of first \cite
%\item \cite[269]{zukofsky1931}, quoted in \cite[78]{costello1981}
%\end{citebib}

\section{Examples from \emph{CMS} Chap. 15, ``Author-Date
References''}
\label{paren}

Examples in this section reproduce those in \textit{CMS} Chapter 15.
To help with cross-checking, subsection numbers and headings are from
\textit{CMS}. Since parenthetical citations are relatively simple, and
since the format of references lists is derivative of the default, the
examples below are more selective than those in the previous section.

\subsection{Basic Format, With Examples and Variations}
\setcounter{subsection}{15}

\setcounter{subsubsection}{8}
\subsubsection{Author-date references—examples and variations}
% 15.9 Author-date references—examples and variations

\begin{citeref}
\item \parencite[87-88]{strayed2012}
\item \parencite[261, 265]{strayed2012}
\item \parencite[32]{daum2015}
\item \parencite[188]{grazer2015}
\item \parencite[242--55]{garcia1988}
\item \parencite[310]{gould1984a}
\item \parencite[484--85]{bagley2015}
\item \parencite[312]{liu2015}
\end{citeref}

\setcounter{subsubsection}{13}
\subsubsection{Placement of dates in reference list entries}
% 15.14 Placement of dates in reference list entries

\begin{citeref}
\item \parencite{pager2015}
\item \parencite{unger2014}
\end{citeref}

\setcounter{subsubsection}{19}
\subsubsection{Reference list entries with same author(s), same year}
% 15.20 Reference list entries with same author(s), same year

\begin{citeref}
\item \parencite[218]{fogel2004b}
\item \parencite[45--46]{fogel2004a}
\end{citeref}

\setcounter{subsection}{1}
\subsection{Author-Date References: Special Cases}
\setcounter{subsection}{15}

\setcounter{subsubsection}{33}
\subsubsection{Author-date format for anonymous works (no listed author)}
% 15.34 Author-date format for anonymous works (no listed author)

See Section \ref{entryops} on the \opt{anonauth} and \opt{anonauthq}
entry options.

\begin{citeref}
\item \parencite{anon1610}
\item \parencite{anon1547}
\item \parencite{horsley1796}
\item \parencite{hawkes1834}
\end{citeref}

\subsubsection{Pseudonyms in author-date references}
% 15.35 Pseudonyms in author-date references

\begin{citeref}
\item \parencite{stendhal1925}
\end{citeref}

\subsubsection{Editor in place of author in text citations}
% 15.36 Editor in place of author in text citations

\begin{citeref}
\item \parencite{silverstein1974}
\item \parencite{soltes1999}
\end{citeref}

\subsubsection{Organization as author in author-date references}
% 15.37 Organization as author in author-date references

\begin{citeref}
% Error in book: '4:1997' shouldn't be part of the citation
\item \parencite{iso1997.ref}
\end{citeref}

\setcounter{subsubsection}{39}
\subsubsection{Reprint editions and modern editions—more than one date}
% 15.40 Reprint editions and modern editions—more than one date

\begin{citeref}
\item \parencite{austen2003}
\item \parencite{maitland1998}
\end{citeref}

\subsubsection{Multivolume works published over more than one year}
% 15.41 Multivolume works published over more than one year

\begin{citeref}
\item \parencite[1:133]{tillich1951}
\item \parencite[vol. 2]{tillich1951}
\item \parencite[329]{hayek1988}
\end{citeref}

\subsubsection{Cross-references to multiauthor books in reference lists}
% 15.42 Cross-references to multiauthor books in reference lists

\begin{citebib}
\item \cite{draper1987}
\item \cite{harrington1987}
\item \cite{zukowsky1987}
\end{citebib}

\setcounter{subsubsection}{43}
\subsubsection{No date of publication in author-date references}
% 15.44 No date of publication in author-date references

\begin{citeref}
\item \parencite{nano1750}
\item \parencite{nano}
\end{citeref}

\subsubsection{``Forthcoming'' in author-date references}
% 15.45 ``Forthcoming'' in author-date references

\begin{citeref}
\item \parencite{faraday}
\end{citeref}

\setcounter{subsubsection}{46}
\subsubsection{Parentheses or comma with issue number}
% 15.47 Parentheses or comma with issue number

\begin{citeref}
\item \parencite{glass2014}
\item \parencite{meyerovitch1959}
\end{citeref}

\subsubsection{Colon with volume number}
% 15.48 Colon with volume number

\begin{citeref}
\item \parencite{gunderson2015}
\end{citeref}

\subsubsection{Newspapers and magazines in reference lists}
% 15.49 Newspapers and magazines in reference lists

\begin{citeref}
\item \parencite{nytimes2002}
\end{citeref}

\setcounter{subsubsection}{54}
\subsubsection{Patents or other documents cited by more than one date}
% 15.55 Patents or other documents cited by more than one date

\begin{citeref}
\item \parencite{iizuka1986}
\end{citeref}

\defbibnote{sh}{This section shows the output of
\cmd{printshorthands}. By default, works from this list also appear in
bibliographies and reference lists. To exclude them, use the preamble
option \opt{nolos} (see Section \ref{preops}). Note that the style
does not automatically italicize a \bibfield{shorthand}. Whether you
should italicize it depends on whether you should italicize the work's
title (14.60).\\}%

\defbibnote{bib}{This section shows the default output of
\cmd{printbibliography}. In the next section, the same works appear in
the author-date format.\\}%

\defbibnote{ref}{This section shows the output of
\cmd{printbibliography} for a reference list (see \opt{reflist} in
Section \ref{preops}). The works below are the same as those in the
previous section but in the author-date format.\\}%

\printshorthands[prenote=sh]
\refstepcounter{sh}\label{sh}
\printbibliography[notkeyword=notinbib,prenote=bib]
\refstepcounter{bib}\label{bib}
\printbibliography[%
  env=reflist,
  heading=references,
  notkeyword=notinref,
  prenote=ref]
\refstepcounter{ref}\label{ref}
\end{document}
