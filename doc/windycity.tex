% Last modified: Sat 28 Nov 2020 04:15:08 PM CST
\documentclass[11pt,letterpaper,oneside]{article}
\usepackage{windycity}

\begin{document}
\title{Windy City}
\subtitle{A Chicago Style for \texttt{\biblatex}}
\author{Brian Chase}
\email{brianmichaelchase@gmail.com}
\website{https://github.com/brianchase/windycity}
\version{2020-11-10}
\maketitle
\begingroup
\hypersetup{linkcolor=black}
\tableofcontents\markboth{Contents}{Contents}
\endgroup

\section{Introduction}

\nfootnote{Copyright \textcopyright\ 2014--2020 Brian Chase. Under the
terms of the \LaTeX\ Project Public License, version 1.3, permission
is granted to copy, distribute, or modify this software. See
\url{http://www.ctan.org/tex-archive/macros/latex/base/lppl.txt} or
\url{https://www.latex-project.org/lppl/}.}

Windy City is a style for \biblatex that formats notes,
bibliographies, parenthetical citations, and reference lists according
to the 17th edition of \textit{The Chicago Manual of Style}
(\textit{CMOS}).\footnote{\cite{chicago2017}} It accurately handles a
wide range of citations in different formats and includes a set of
options and commands to support special circumstances. It also has
extensive support for citing and arranging different kinds of editors
and translators within a single citation. These features make Windy
City especially suitable for academic work.

The following sections assume familiarity with \textit{CMOS} and
\biblatex. Section \ref{overview} gives a brief overview of the
style's features. Section \ref{edtrans} discusses the assignment and
placement of editors, translators, and the like. Section
\ref{collections} discusses several issues with collections, including
options for formatting citations of individual volumes. Sections
\ref{notes} and \ref{paren} reproduce examples from \textit{CMOS}
chapters 14 and 15, respectively, with occasional commentary and
references to other sections.

Windy City requires \biblatex version 3.13 or later.

\section{Overview}
\label{overview}

This section covers basic information about Windy City. If you're
completely new to \biblatex, you should probably glance at its
documentation. For the impatient, examples in sections \ref{standard},
\ref{short}, \ref{notes}, and \ref{paren} might be of more immediate
interest.

\subsection{Getting Started}

If you already know how to use \biblatex, getting started with Windy
City is easy. The first task is to confirm that \biblatex and Windy
City are installed properly on your system. Since both are included in
some distributions of \LaTeX, you might be able to skip this step.

Either way, please consider downloading the most recent release of
Windy City from \begingroup \hypersetup{urlcolor=blue}
\href{https://www.ctan.org/pkg/windycity}{its home on CTAN}. More
recent but potentially less reliable updates are available from Windy
City's \href{https://github.com/brianchase/windycity}{repository on
GitHub}.\endgroup

Windy City consists of four files:

\begin{itemize}[before=\small]
\item \file{windycity.bbx}
\item \file{windycity.cbx}
\item \file{windycity.dbx}
\item \file{american-windycity.lbx}
\end{itemize}

If you need to install Windy City on your system, you have several
options. For a one-off compilation, say, to give Windy City a trial
run on a single document, you could copy Windy City's files to the
document's root directory. Beyond that, your best option is to install
it in your local \path{texmf}. Copy the files to a directory of your
choosing, then update your \path{texmf} file name database.

To compile a document with Windy City, tell \biblatex to load it with
the load-time option \opt{style}:

\begin{verbatim}
   \usepackage[style=windycity]{biblatex}
\end{verbatim}

\noindent Typically, this goes in a document's preamble or in one of
its style files.

The localization file \file{american-windycity.lbx} is responsible for
Windy City's American-style punctuation and dates and many of the
bibliography strings that print in citations, such as \textit{edited
by}. Windy City loads this file if \biblatex determines that your
document's language is English---either due to settings in \babel or
\polyglossia or because neither \babel nor \polyglossia has been
loaded. You can prevent Windy City from loading
\file{american-windycity.lbx} by commenting the following line in
\file{windycity.bbx}:

\begin{verbatim}
   \DeclareLanguageMapping{english}{american-windycity}
\end{verbatim}

If you wish to use Windy City with a language other than English, set
it accordingly with \babel or \polyglossia before loading \biblatex.
Windy City will then try to load \file{<language>-windycity.lbx} and,
if it's available, use it to override any other localization files
that were loaded. This allows you to make your own localization files
for Windy City without needing to edit other files. For example, if
you load \babel with option \sty{german}, Windy City will try to load
\file{german-windycity.lbx}. If it's available, it will override any
other localization files that were loaded, including \biblatex's own
\file{german.lbx}.

For some entries in your bibliography database, you may need to add
fields or make other adjustments to get the right output. But since
Windy City relies as much as possible on standard \BibTeX\ fields, and
secondarily on \biblatex fields, you may not need to make major
changes. The examples in this document and its accompanying
bibliography database, \file{windycity.bib}, should serve as a guide
for how to manage your input for nearly every circumstance that the
style is meant to handle.

\subsection{Standard Citations}
\label{standard}

For a first set of examples, consider this passage from \textit{CMOS}
\ref{14.30}:

\begin{citeonly}
\item \cite[24--25]{morley1995}
\item \cite{schwartz1992}
\item \cite{kaiser1964}
\item \cite[43]{morley1995}
\item \cite[138]{schwartz1992}
\item \cite[189--90]{kaiser1964}
\end{citeonly}

A work's first citation is similar to its entry in the bibliography.
It includes all or most of its bibliographic information. Subsequent
citations are shorter, usually consisting of a short form of the
author's name and a short form of the work's title.

Windy City supports variations on this format. For information on
short citations, including the use of \textit{ibid.}, see section
\ref{short}. For options to skip parts of citations, change the order
of editors and translators, and more, see sections \ref{preamble} and
\ref{entry}. For parenthetical citations, see section \ref{paren}.

The block below shows Windy City's default bibliography for the
previously cited works:

\begin{bibonly}
\nocite{kaiser1964,morley1995,schwartz1992}
\end{bibonly}

You may also print a bibliography in the author-date format (what
\textit{CMOS} calls a reference list). The key difference is the
placement of the publication date after the author's name:

\begin{refonly}
\nocite{kaiser1964,morley1995,schwartz1992}
\end{refonly}

To make \cmd{printbibliography} use the author-date format, load
\biblatex with Windy City's preamble option \opt{reflist}:

\begin{verbatim}
   \usepackage[reflist,style=windycity]{biblatex}
\end{verbatim}

\noindent Alternatively:

\begin{verbatim}
   \usepackage[reflist=true,style=windycity]{biblatex}
\end{verbatim}

To use the author-date format on a case-by-case basis, run
\cmd{printbibliography} with an appropriate \opt{env} option. With
Windy City, a so-called ``bib environment'' must set the style's
internal \opt{reflist} toggle to \opt{true}. Windy City's own such
environment is called \opt{reflist}. Use it as follows:

\begin{verbatim}
   \printbibliography[env=reflist]
\end{verbatim}

Unfortunately, while the \opt{env} option allows for differently
formatted bibliographies within the same document, the reference lists
may have problems with sorting. (You'll notice some in \ref{ref}, at
the end of this document.) For best results, use the \opt{reflist}
preamble option.

As you proceed through the this guide, note that all examples of
citations and bibliographies are outputs of the style from commands
that you can inspect in the document's source, \file{windycity.tex},
and in its style file, \file{windycity.sty}. Almost all citations are
from \cmd{cite} or \cmd{parencite}. A few are from more specialized
commands, such as \cmd{cite*} or \cmd{cites}. All example
bibliographies are outputs of the style from \cmd{printbibliography}.
All bibliographic data reside in \file{windycity.bib}.

\subsection{Short Citations}
\label{short}

Standard citations may take a variety of shorter forms. Windy City
offers several preamble options and other means for reproducing them.

Let's start with another passage of default output:

\begin{citeonly}
\item \cite[3]{morrison2004a}
\item \cite[18]{morrison2004a}
\item \cite[18]{morrison2004a}
\item \cite[24--26]{morrison2004a}
\item \cite[401-2]{morrison2004b}
\item \cite[433]{morrison2004b}
\item \cite[37--38]{diaz2008}
\item \cite[403]{morrison2004b}
\item \cite[152]{diaz2008}
\item \cite[201-2]{diaz2008}
\item \cites[240]{morrison2004b}[32]{morrison2004a}
\item \cite[33]{morrison2004a}
\end{citeonly}

A shorter form of this passage appears in \textit{CMOS}
\ref{14.34}:\footnote{Switching forms within a document isn't a
feature of the style. For demonstration purposes, though, it's
possible.}

% The use of 'firstshort' below might be confusing. It's necessary to
% simulate the effect of 'short' within a document that otherwise
% gives default output.

\begin{citeonly}
\AtNextCitekey{\toggletrue{short}\toggletrue{firstshort}}
\item \cite[3]{morrison2004a}
\AtNextCitekey{\toggletrue{short}}
\item \cite[18]{morrison2004a}
\AtNextCitekey{\toggletrue{short}}
\item \cite[18]{morrison2004a}
\AtNextCitekey{\toggletrue{short}}
\item \cite[24--26]{morrison2004a}
\AtNextCitekey{\toggletrue{short}\toggletrue{firstshort}}
\item \cite[401-2]{morrison2004b}
\AtNextCitekey{\toggletrue{short}}
\item \cite[433]{morrison2004b}
\AtNextCitekey{\toggletrue{short}\toggletrue{firstshort}}
\item \cite[37--38]{diaz2008}
\AtNextCitekey{\toggletrue{short}}
\item \cite[403]{morrison2004b}
\AtNextCitekey{\toggletrue{short}}
\item \cite[152]{diaz2008}
\AtNextCitekey{\toggletrue{short}}
\item \cite[201-2]{diaz2008}
\AtNextMultiCite{\toggletrue{short}}
\item \cites[240]{morrison2004b}[32]{morrison2004a}
\AtNextCitekey{\toggletrue{short}}
\item \cite[33]{morrison2004a}
\end{citeonly}

In this version, a work's first citation gives short names and titles
and omits all other publication information. Consecutive citations of
a work may omit the title or, as in the eleventh note, the author's
name. For citations in this form, use the preamble option \opt{short}.
See section \ref{preamble} for more information.

\textit{CMOS} \ref{14.34} also shows how to render the passage with
\textit{ibid.} Unlike previous editions of \textit{CMOS}, the 17th
edition discourages its use. Windy City makes it available with the
preamble option \opt{ibid} (see section \ref{preamble}). Options
\opt{short} and \opt{ibid} together give the following:\footnote{As
explained in section \ref{preamble}, Windy City won't print
\textit{ibid.} in reference to a citation on a previous page. A page
break here may affect the output.}

\begin{citeonly}
\AtNextCitekey{\toggletrue{short}\toggletrue{firstshort}\toggletrue{ibid}}
\item \cite[3]{morrison2004a}
\AtNextCitekey{\toggletrue{short}\toggletrue{ibid}}
\item \cite[18]{morrison2004a}
\AtNextCitekey{\toggletrue{short}\toggletrue{ibid}}
\item \cite[18]{morrison2004a}
\AtNextCitekey{\toggletrue{short}\toggletrue{ibid}}
\item \cite[24--26]{morrison2004a}
\AtNextCitekey{\toggletrue{short}\toggletrue{firstshort}\toggletrue{ibid}}
\item \cite[401-2]{morrison2004b}
\AtNextCitekey{\toggletrue{short}\toggletrue{ibid}}
\item \cite[433]{morrison2004b}
\AtNextCitekey{\toggletrue{short}\toggletrue{firstshort}\toggletrue{ibid}}
\item \cite[37--38]{diaz2008}
\AtNextCitekey{\toggletrue{short}\toggletrue{ibid}}
\item \cite[403]{morrison2004b}
\AtNextCitekey{\toggletrue{short}\toggletrue{ibid}}
\item \cite[152]{diaz2008}
\AtNextCitekey{\toggletrue{short}\toggletrue{ibid}}
\item \cite[201-2]{diaz2008}
\AtNextMultiCite{\toggletrue{short}\toggletrue{firstshort}}
\item \cites[240]{morrison2004b}[32]{morrison2004a}
\AtNextCitekey{\toggletrue{short}\toggletrue{ibid}}
\item \cite[33]{morrison2004a}
\end{citeonly}

Other ways to make citations more concise: For a compromise between
standard and short forms, try the  preamble option \opt{firstshort}.
It swaps long first citations for short ones but otherwise follows the
standard (see section \ref{preamble}). Also with standard citations,
the preamble option \opt{idemtracker} shortens the author's name of a
work's first citation if the previous citation is of the same author
(see section \ref{preamble}). The entry option \opt{noauth} omits the
author's name altogether (see section \ref{entry}). And the
\bibfield{shorthand} field allows you to set an abbreviation to stand
in place of the author's name, the work's title, and other elements of
a citation (see sections \ref{otherfields} and \ref{14.59}).

\subsection{Preamble Options}
\label{preamble}

A preamble option is an argument for the \cmd{usepackage} macro that
loads \biblatex. Preamble options affect the format of notes,
bibliographies, and reference lists. Some features of the style
require them.

All options described below are \opt{false} by default. Set them to
\opt{true} by passing the name of the option to \biblatex, with or
without an additional \opt{=true}. In other words, using
\opt{annotate} as an example, the following are equivalent:

\begin{verbatim}
   \usepackage[annotate,style=windycity]{biblatex}
   \usepackage[annotate=true,style=windycity]{biblatex}
\end{verbatim}

Bear in mind that Windy City uses many preamble options native to
\biblatex, a few of which you may want to change. These options are
set in \file{windycity.bbx}. In particular, the style sets
\opt{idemtracker} to \opt{false}. If you set it to \opt{true} (or to
some value that implies \opt{true}), Windy City will detect when the
first citation of a work follows another citation of a work by the
same author and shorten the author's name. Recall from the previous
section the citation of Toni Morrison's \textit{Song of Solomon}
immediately after a citation of her \textit{Beloved}. In a context
like that, do you really need to remind readers of the author's full
name? If you think not, change \opt{idemtracker} to an appropriate
value (see section 3.1.2.3 of \biblatex's user
guide).\footnote{\textit{CMOS} seems to have no policy on this point.
In the 16th edition, however, Figure 14.3 shows consecutive citations
of works by the same author, both of which give the author's full
name.}

\begin{optionlist}

\optitem[false]{annotate}{\opt{true}, \opt{false}}

\noindent This option is for printing annotated bibliographies.
Annotations print in block paragraphs below entries. To change
the spacing between entries and annotations, change the value of
\cmd{bibitemsep}. Save an annotation in the \bibfield{annotation}
field of a work's bibliography database entry.

\optitem[false]{collsonly}{\opt{true}, \opt{false}}

\noindent Citing individual works of a collection adds entries for
those works to the bibliography. To exclude them and print only an
entry for the whole collection, use \opt{collsonly}. It has no effect
on many \bibtype{incollection} entries, such as articles in books, but
it does filter out chapters of books, books in books, and volumes of
collections. For discussion of multivolume works, see section
\ref{multivolume}.

\optitem[false]{firstshort}{\opt{true}, \opt{false}}

\noindent Use this option to shorten a work's first citation. The
resulting output consists mainly of the author's name and the work's
title. According to \textit{CMOS}, this approach is optional for
documents with complete bibliographies. (See \textit{CMOS}
\ref{14.23}, also 14.29--14.36.) You may use \opt{firstshort} in
conjunction with \opt{ibid}. However, it adds nothing to \opt{short},
which has the same effect on first citations.

\optitem[false]{ibid}{\opt{true}, \opt{false}}

\noindent This option controls whether consecutive citations of a work
on the same page receive an \textit{ibid}. The qualification ``on the
same page'' means that \textit{ibid.} always refers to a work cited on
the current page without an \textit{ibid.} The latter isn't a
requirement of \textit{CMOS} but seems reasonable, since it prevents
readers from having to look at another page to find the referent of an
\textit{ibid}. For examples of its output, see section \ref{short} and
\ref{14.34}. As of the 17th edition, \textit{CMOS} discourages the use
of \textit{ibid.} (see \ref{14.34}).

\optitem[false]{isbn}{\opt{true}, \opt{false}}

\noindent Use this option to print ISBNs in bibliographies. A work's
ISBN goes in the \bibfield{isbn} field of its bibliography database
entry. With this option, the style prints ISBNs at the end of every
entry in the bibliography, before annotations. To print the ISBN of a
particular work, use the \opt{isbn} entry option.

\optitem[false]{issn}{\opt{true}, \opt{false}}

\noindent Similar to \opt{isbn} but for ISSNs.

\optitem[false]{library}{\opt{true}, \opt{false}}

\noindent Like \opt{isbn} and \opt{issn}, this option prints the
\bibfield{library} field of every work in the bibliography. Use it to
print information about libraries, call numbers, and the like. If you
use it with the \opt{isbn} and \opt{annotation} options, it prints
after the former but before the latter. To print this information for
selected works, use the \opt{library} entry option.

\optitem[false]{nolos}{\opt{true}, \opt{false}}

\noindent By default, every work with a \bibfield{shorthand} receives
an entry in the bibliography. If you wish to exclude them, say, to
avoid duplication with the output of \cmd{printshorthands}, use
\opt{nolos}. Since \opt{collsonly} also excludes works from the
bibliography, their results may overlap.

\optitem[false]{nopages}{\opt{true}, \opt{false}}

\noindent On the first citation of \bibtype{article} or
\bibtype{review} entries (and their aliases), Windy City prints the
\bibfield{pages} field if the \bibfield{postnote} field is blank. This
lets you cite the entirety of a work without having to duplicate the
content of the \bibfield{pages} field in the \bibfield{postnote}. To
override this feature, use the \opt{nopages} option.

\optitem[false]{reflist}{\opt{true}, \opt{false}}

\noindent Use this option to print a bibliography in the author-date
format (what \textit{CMOS} calls a reference list). If you use
parenthetical citations, consider using \opt{reflist} to maintain
consistency with \textit{CMOS}. Again, another way to print a
reference list is to pass \opt{env=reflist} to
\cmd{printbibliography}. See section \ref{overview} for more
information.

\optitem[false]{short}{\opt{true}, \opt{false}}

\noindent As shown in section \ref{short}, this option prints short
citations (see \textit{CMOS} \ref{14.34}). The use of \opt{short} has
one feature in common with \opt{ibid}: Just as \textit{ibid.} appears
only for consecutive citations of a work on the same page, and so
never refers to a citation on a previous page, \opt{short} affects
consecutive citations of a work on the same page, never in reference
to a citation on a previous page. As with \textit{ibid.}, this feature
isn't required by \textit{CMOS}, but it prevents readers from having
to look at another page to find the title of a citation.

In contexts where \opt{short} would drop a title from a citation, but
where no name occupies the author's position, it prints the work's
\bibfield{labeltitle}. This can be a short form of the title, either
the title minus the subtitle or the content of the
\bibfield{shorttitle} field, if available.

As noted earlier, \opt{short} has the same effect on first citations
as \opt{firstshort}. But recall from section \ref{short} that you can
combine \opt{short} and \opt{ibid} for more concise output.

\optitem[false]{swapvol}{\opt{true}, \opt{false}}

\noindent In bibliographies and long citations, works in collections
may place publication information for the volume before that of the
collection or \textit{vice versa}. Windy City gives priority to the
volume. To reverse this for all relevant citations, use \opt{swapvol}.
For more information, see section \ref{collorder}.

\end{optionlist}

\subsection{Entry Options}
\label{entry}

An entry option is a value for the \bibfield{options} field of a
work's database entry. It affects the format of that particular work.
For options that affect the format of every work, see section
\ref{preamble}.

\begin{optionlist}

\optitem[false]{isbn}{\opt{true}, \opt{false}}

\noindent Use this option to print the ISBN of a particular work in a
bibliography. The ISBN appears at the end of the work's entry (if
applicable, before an annotation). To print ISBNs of every work in the
bibliography, see the \bibfield{isbn} preamble option.

\optitem[false]{issn}{\opt{true}, \opt{false}}

\noindent Similar to \opt{isbn} but for ISSNs.

\optitem[false]{library}{\opt{true}, \opt{false}}

\noindent This option prints the \bibfield{library} field of a work's
bibliography database entry. This information prints at the end of the
work's entry (if applicable, after an ISBN and before an annotation).
To print the \bibfield{library} field of every work in the
bibliography, use the \opt{library} preamble option.

\optitem[false]{listvols}{\opt{true}, \opt{false}}

\noindent \textit{CMOS} offers some flexibility over whether a long
citation gives a work's total number of volumes (see \textit{CMOS}
\ref{14.118}). Windy City replicates this in a roundabout way: By
default, it prints the \bibfield{volumes} field in certain long
citations only if the \bibfield{postnote} field is empty. This does a
better job of matching examples in \textit{CMOS} than a simpler policy
would. For the rest, use \opt{listvols}. It prints the
\bibfield{volumes} field in those citations no matter what the
\bibfield{postnote} contains. For more information, see section
\ref{collorder}.

\optitem[false]{noauth}{\opt{true}, \opt{false}}

\noindent This option tells the style to bypass the author's position
of a work in notes and bibliographies. Citations will begin with the
title's position. Below is an example from \textit{CMOS} \ref{14.105}:

\begin{citebib}
\item \cite{chaucer1966}
\end{citebib}

\noindent To bypass the author's position in a single note, without
affecting the bibliography, use starred versions of citation commands,
such as \cmd{cite*}, \cmd{footcite*}, and \cmd{parencite*}.

\optitem[false]{skipdate}{\opt{true}, \opt{false}}

\noindent On rare occasions, you may need an entry in a reference list
to skip the \textit{date}. See section \ref{entrytypes} for examples.

\optitem[false]{swapauth}{\opt{true}, \opt{false}}

\noindent To swap the places of a book's author with an editor or
translator, use \opt{swapauth}. This option works for \bibtype{book},
\bibtype{bookinbook}, \bibtype{collection}, \bibtype{inbook}, and
\bibtype{incollection} entry types. For more information, see section
\ref{edtranspos}.

\optitem[false]{swaptrans}{\opt{true}, \opt{false}}

\noindent According to \textit{CMOS}, if a work has an editor and a
translator, their names should appear in citations in the order in
which they appear on the work's title page (see \textit{CMOS}
\ref{14.104}). By default, the style lists editors first. Entries with
the option \bibfield{swaptrans} reverse this order: Their translators
print first. If a work's translators and editors are the same,
\bibfield{swaptrans} reverses the order of their roles, say, from
\textit{edited and translated by} to \textit{translated and edited
by}. The same goes for compilers, revisers, and updaters, which the
style treats as specialized editors. For more information, see section
\ref{edtranspos}.

\optitem[false]{swapvol}{\opt{true}, \opt{false}}

\noindent As an entry option, \opt{swapvol} does on a case-by-case
basis what the \opt{swapvol} preamble option does globally: When set
to true, it changes the format of a work in a collection so that, in
bibliographies and long citations, publication information for the
collection precedes that of the volume. For more information, see
section \ref{collorder}.

\end{optionlist}

\subsection{Citation Commands}

Windy City supports most of the citation commands familiar from
\biblatex, plus a small set of custom commands for tricky situations.

\subsubsection{Standard Citation Commands}
\label{std.cmd}

The following list, though not exhaustive, shows the most important
citation commands from \biblatex that Windy City supports:

\begin{ltxsyntax}
\cmditem{autocite}[prenote][postnote]{key}
\cmditem{autocites}(multiprenote)(multipostnote)[prenote][postnote]{key}|...|[prenote][postnote]{key}
\cmditem{cite}[prenote][postnote]{key}
\cmditem{cite*}[prenote][postnote]{key}
\cmditem{cites}(multiprenote)(multipostnote)[prenote][postnote]{key}|...|[prenote][postnote]{key}
\cmditem{cites*}(multiprenote)(multipostnote)[prenote][postnote]{key}|...|[prenote][postnote]{key}
\cmditem{footcite}[prenote][postnote]{key}
\cmditem{footcite*}[prenote][postnote]{key}
\cmditem{footcites}(multiprenote)(multipostnote)[prenote][postnote]{key}|...|[prenote][postnote]{key}
\cmditem{footcites*}(multiprenote)(multipostnote)[prenote][postnote]{key}|...|[prenote][postnote]{key}
\cmditem{footfullcite}[prenote][postnote]{key}
\cmditem{footfullcite*}[prenote][postnote]{key}
\cmditem{fullcite}[prenote][postnote]{key}
\cmditem{fullcite*}[prenote][postnote]{key}
\cmditem{nocite}{key}
\cmditem*{nocite}|\{*\}|
\cmditem{parencite}[prenote][postnote]{key}
\cmditem{parencite*}[prenote][postnote]{key}
\cmditem{parencites}(multiprenote)(multipostnote)[prenote][postnote]{key}|...|[prenote][postnote]{key}
\cmditem{smartcite}[prenote][postnote]{key}
\cmditem{smartcite*}[prenote][postnote]{key}
\cmditem{smartcites}(multiprenote)(multipostnote)[prenote][postnote]{key}|...|[prenote][postnote]{key}
\cmditem{smartcites*}(multiprenote)(multipostnote)[prenote][postnote]{key}|...|[prenote][postnote]{key}
\cmditem{textcite}[prenote][postnote]{key}
\cmditem{textcites}(multiprenote)(multipostnote)[prenote][postnote]{key}|...|[prenote][postnote]{key}
\cmditem{textcites*}(multiprenote)(multipostnote)[prenote][postnote]{key}|...|[prenote][postnote]{key}
\end{ltxsyntax}

\noindent New users should get comfortable with \cmd{cite} and
\cmd{footcite} (for notes) and \cmd{parencite} (for parenthetical
citations). Their multicite forms---\cmd{cites}, \cmd{footcites}, and
\cmd{parencites}---take comma-separated lists of entry keys, while
allowing you to specify distinct \bibfield{prenote} and
\bibfield{postnote} fields for each work.

Use \cmd{nocite} to add works to bibliographies even if you haven't
cited them in the text with citations commands. Use it with an
asterisk to add every work in every \file{bib} file that your document
loaded with \cmd{addbibresource} or \cmd{bibliography}.

Windy City sets the \opt{autocite} preamble option to \opt{footnote},
which makes \cmd{autocite} equivalent to \cmd{smartcite}. The latter
runs different citation commands in different contexts. In Windy City,
\cmd{smartcite} is equivalent to \cmd{footcite} in a document's body
and \cmd{cite} in footnotes and endnotes.\footnote{In a \env{minipage}
environment, \cmd{smartcite} is equivalent to \cmd{cite}. To use
\cmd{smartcite} with endnotes, use the \sty{endnotes} package.} If you
set the \opt{autocite} preamble option to \opt{inline}, \cmd{autocite}
is equivalent to \cmd{parencite}. If you set it to \opt{plain},
\cmd{autocite} is equivalent to \cmd{cite}.

\begin{ltxsyntax}

\cmditem{cite*}[prenote][postnote]{key}

Use this command to cite a work without printing anything in the
author's position. It comes in handy when the context makes the
author's name clear. From \textit{CMOS} \ref{14.78}:

\begin{citebib}
\item \cite*[233]{franklin1868}
\item \cite*[234]{franklin1868}
\end{citebib}

\cmditem{parencite*}[prenote][postnote]{key}

Like \cmd{parencite} but suppresses the author's position---useful in
passages where the author receives explicit mention. Here's an example
from \textit{CMOS} \ref{15.25}:

\begin{quote} Fiorina et al. \parencite*{fiorina2005} and Fischer and
Hout \parencite*{fischer2006} reach more or less the same conclusions.
In contrast, Abramowitz and Saunders \parencite*{abramowitz2005}
suggest that the mass public is deeply divided between red states and
blue states and between churchgoers and secular voters. \end{quote}

The source for the passage above contains:

\begin{verbatim}
   \begin{quote} Fiorina et al. \parencite*{fiorina2005}... Fischer
   and Hout \parencite*{fischer2006}... Abramowitz and Saunders
   \parencite*{abramowitz2005}... \end{quote}
\end{verbatim}

\cmditem{textcite}[prenote][postnote]{key}

Use \cmd{textcite} for in-text citations. Here's an example from
\textit{CMOS} 13.65:

\begin{quote} ``If an astronaut falls into a black hole, its mass will
increase, but eventually the energy equivalent of that extra mass will
be returned to the universe in the form of radiation. Thus, in a
sense, the astronaut will be `recycled'''
\mkbibparens{\textcite[112]{hawking1988}}. \end{quote}

The source for the passage above contains:

\begin{verbatim}
   \begin{quote} ``If an astronaut falls into a black hole...''
   \mkbibparens{\textcite[112]{hawking1988}}. \end{quote}
\end{verbatim}

Note that \cmd{textcite} doesn't enclose citations in parentheses but
does drop the final period that standard citations usually end with.
The parentheses above come from \cmd{mkbibparens}, which switches the
note's internal parentheses to brackets.

\cmditem{textcite*}[prenote][postnote]{key}

Like \cmd{textcite} but suppresses the author's position. From
\textit{CMOS} 13.65:

\begin{quote} In their introduction to \textcite*{tocqueville1999},
translators Harvey Mansfield and Delba Winthrop write that Tocqueville
``shows that the people are sovereign, whether through the
Constitution or despite it, and he warns of the tyranny of the
majority'' \parencite[xvii]{tocqueville1999}. \end{quote}

The source for the passage above contains:

\begin{verbatim}
   \begin{quote} In their introduction to \textcite*{tocqueville1999}
   ...\parencite[xvii]{tocqueville1999}. \end{quote}
\end{verbatim}

\end{ltxsyntax}

\subsubsection{Other Citation Commands}
\label{cust.cmd}

Windy City's custom citation commands are convenient but not strictly
necessary. They make certain citations simpler than they would be with
standard commands.

\begin{ltxsyntax}
\cmditem{idemcite}[prenote][postnote]{key}
\cmditem{idemcites}(multiprenote)(multipostnote)[prenote][postnote]{key}|...|[prenote][postnote]{key}
\cmditem{footidemcite}[prenote][postnote]{key}
\cmditem{footidemcites}(multiprenote)(multipostnote)[prenote][postnote]{key}|...|[prenote][postnote]{key}
\end{ltxsyntax}

Perhaps the best use case for these commands is the example in
\textit{CMOS} \ref{14.57}. The sources mentioned in a passage receive
long citations with shortened names:

\begin{quote} Only when we gather the work of several
scholars---Walter Sutton's explications of some of Whitman's shorter
poems; Paul Fussell's careful study of structure in ``Cradle''; S. K.
Coffman's close readings of ``Crossing Brooklyn Ferry'' and ``Passage
to India''; and the attempts of Thomas I. Rountree and John Lovell,
dealing with ``Song of Myself'' and ``Passage to India,''
respectively, to elucidate the strategy in ``indirection''---do we
begin to get a sense of both the extent and the specificity of
Whitman's forms.\footnotemark[1] \end{quote}

\begin{citeonly}
\item \idemcites{sutton1959,fussell1962,coffman1954,coffman1955,rountree1958}[and][]{lovell1960}
\end{citeonly}

Without commands like \cmd{idemcite}, there's often no easy way to
tell Windy City to shorten names. The preamble option
\opt{idemtracker} helps with consecutive citations of the same author,
but that's no use when you want to shorten names because you've
mentioned them in the text.

Nevertheless, standard commands can reproduce the example in at least
three ways:

\begin{verbatim}
   \footnote{\cites*[Sutton,][]{sutton1959}[Fussell,][]{fussell1962}...
   \footcites*[Sutton,][]{sutton1959}[Fussell,][]{fussell1962}...
   \footnote{Sutton, \cite*{sutton1959}; Fussell, \cite*{fussell1962}...
\end{verbatim}

\noindent They're ugly workarounds, exploiting starred citations
commands to suppress the author's position while you add names
manually---but they do work.

With \cmd{idemcites} and \cmd{footidemcites}, on the other hand, you
have simpler alternatives that make your intentions clearer and your
file easier to read:

\begin{verbatim}
   \footnote{\idemcites{sutton1959,fussell1962,...}[and][]{lovell1960}}
   \footidemcites{sutton1959,fussell1962,...}[and][]{lovell1960}
   \footnote{\idemcite{sutton1959}; ... and \idemcite{lovell1960}}
\end{verbatim}

Notice that the only reason to use \cmd{footidemcites} rather than
\cmd{footidemcite} is to insert \textit{and} before the last citation.

\subsection{Entry Types}
\label{entrytypes}

Windy City recognizes every entry type from \BibTeX, as well as some
that are specific to \biblatex.

\begin{typelist}
\RaggedRight

\typeitem{article}

Alias: \bibtype{periodical}

\typeitem{book}

Aliases: \bibtype{booklet}, \bibtype{collection}, \bibtype{manual},
\bibtype{mvbook}, \bibtype{mvcollection}, \bibtype{proceedings},
\bibtype{report}, \bibtype{techreport}

\typeitem{incollection}

Aliases: \bibtype{bookinbook}, \bibtype{conference},
\bibtype{inproceedings}, \bibtype{inbook}, \bibtype{letter},
\bibtype{suppbook}, \bibtype{suppcollection}

\typeitem{letter} No aliases
\typeitem{misc} No aliases
\typeitem{online} No aliases
\typeitem{patent} No aliases

\typeitem{reference}

Alias: \bibtype{inreference}

\typeitem{review} No aliases

\typeitem{thesis}

Aliases: \bibtype{mastersthesis}, \bibtype{phdthesis},
\bibtype{unpublished}

\end{typelist}

\noindent For the most part, you may assign every work to the basic
entry types listed above. A PhD thesis, for example, may have a
\bibtype{thesis} or \bibtype{phdthesis} entry type. The output is the
same. If an entry has a type other than the ones listed above, Windy
City processes it as a book.

One comment about \bibtype{reference} and \bibtype{inreference}
entries: You may cross-reference \bibtype{inreference} entries to
\bibtype{reference} entries, as with articles in books, but you can
get the same output with one or the other alone. Take an example from
\textit{CMOS} \ref{14.232}:

\begin{citeonly}
\item \cite{salvation1980}
\end{citeonly}

A bibliography database could have an \bibtype{inreference} entry for
the article cross-ref\-er\-enc\-ed to a \bibtype{reference} entry for
\textit{Encyclopaedia Britannica}:

\begin{verbatim}
   @InReference{salvation1980,
     title = {salvation},
     crossref = {britannica1980}
   }
   @Reference{britannica1980,
     organization = {{\emph{Encyclopaedia Britannica}}},
     edition = {15},
     year = {1980}
   }
\end{verbatim}

\noindent This approach makes sense if you plan to cite more than one
article from the source.\footnote{Incidentally, reference works don't
always have titles in italics. As a result, you need to set italics in
your bibliography database.} But you could also have a single entry of
either type with the same data:

\begin{verbatim}
   @Reference{salvation1980,
     organization = {{\emph{Encyclopaedia Britannica}}},
     edition = {15},
     title = {salvation},
     year = {1980}
   }
\end{verbatim}

For unusually complicated citations, or those just not supported by
the style, consider using the \bibtype{misc} entry type. The style
handles these entries in way that makes it a fallback for almost
anything. The example below is from \textit{CMOS} 14.264:

\begin{citebib}
\item \cite{roosevelt1959}
\end{citebib}

\noindent This work's database entry contains most of its information
in \bibfield{usera} (for notes) and \bibfield{userb} (for
bibliographies). The \bibfield{title} field needs manual formatting,
since works of this type may have titles in italics or quotation
marks.

\begin{verbatim}
   @Misc{roosevelt1959,
     author = {Roosevelt, Eleanor},
     title = {\mkbibquote{Is America Facing World Leadership?}},
     usera = {convocation speech, Ball State Teacher's College, May 6,
              \thefield{year}, radio broadcast, reel-to-reel tape,
              MPEG copy, 1:12:49},
     userb = {Convocation Speech. Ball State Teacher's College. May 6,
              \thefield{year}. Radio broadcast. Reel-to-reel tape.
              MPEG copy. 1:12:49},
     url = {http://libx.bsu.edu/cdm4/singleitem/collection/ElRoos/id
            /1},
     year = {1959}
   }
\end{verbatim}

You may also use the \bibtype{misc} entry type to cross-reference
entries in a bibliography, as in \textit{CMOS} \ref{14.81} and
\ref{14.82}.

\begin{bibonly}
\nocite{ashe,creasey1976,creasey1978,creasey1966,morton,york}
\end{bibonly}

\noindent Here's the \bibtype{misc} entry for one of the
cross-references above:

\begin{verbatim}
   @Misc{ashe,
     options = {skipdate},
     author = {Ashe, Gordon},
     userb = {\emph{See} Creasey, John}
   }
\end{verbatim}

\noindent The option \opt{skipdate} is necessary only if you plan to
format your bibliography as a reference list. It prevents Windy City
from printing \textit{n.d.} (no date) after the name to indicate a
missing publication date. After adding the \bibtype{misc} entries, use
\cmd{nocite} with their entry keys to add them to your bibliography,
and cite the remaining entries as usual. For examples of automatic
cross-referencing in notes and bibliographies see section \ref{notes}
(\ref{14.108}), section \ref{paren} (\ref{15.42}), and section
\ref{collections}.

For more discussion of entry types, see section \ref{datafields}.

\subsection{Data Fields}
\label{datafields}

Windy City relies on a small number of data fields that aren't
recognized by \BibTeX\ or \biblatex. It also uses some standard ones
in perhaps unexpected ways.

\subsubsection{Standard Data Fields}
\label{stdfields}

Examples in this document don't always make it clear how Windy City
uses standard data fields. The section discusses some of them.

\begin{marglist}

\item[authtype] This field takes three values: \textit{anon} (to print
an anonymous author's name in brackets), \textit{anon?} (to add a
question mark inside the brackets), and \textit{pseudo} (to print
\textit{pseud.} in brackets after the name of a pseudonymous author).
From \textit{CMOS} \ref{14.79}:

\begin{citebib}
\item \cite{horsley1796}
\item \cite{hawkes1834}
\end{citebib}

With respect to pseudonymous authors, another solution is to use the
\bibfield{nameaddon} field. Unlike with \bibfield{authtype},
\bibfield{nameaddon} allows you to include an author's given name in
the brackets. For examples, see \textit{CMOS} \ref{14.81}.

\item[edition] To indicate a numbered edition of a work, put the
edition's number in this field---for example, \textit{2} for a second
edition. To indicate a revised edition, enter either \textit{revised}
or \textit{rev. ed.} Both options give the same output. For expanded
and updated editions, you may use \textit{expanded} or
\textit{updated}. Add additional data as necessary. For the following
example from \textit{CMOS} bibliography 2.4, \bibfield{edition}
contains \textit{updated edition by Arlene O'Sean and Antoinette
Schleyer}:

\begin{bibonly}
\nocite{swanson1999}
\end{bibonly}

\noindent You may also use the field to indicate a newspaper's
edition, such as \textit{Sunday Book Review} (see \textit{CMOS}
\ref{14.202}).

\item[issue] Windy City uses this field with the \bibtype{article},
\bibtype{online}, and \bibtype{review} entry types to record a work's
season of publication (spring, summer, autumn/fall, winter). The
\bibtype{date} field also records seasons---but the value \textit{23},
for the third season, returns \textit{Autumn}. To print \textit{Fall},
use the \bibfield{issue} field. In other words, for \textit{Fall
2013}, a work's bibliography database entry should contain:

\begin{verbatim}
   issue = {Fall},
   date = {2013},
\end{verbatim}

\noindent Whereas for \textit{Autumn 2013}, use either:

\begin{verbatim}
   date = {2013-23},
\end{verbatim}

\noindent Or:

\begin{verbatim}
   issue = {Autumn},
   date = {2013},
\end{verbatim}

\noindent Note that when \bibfield{date} contains just a year of
publication, you may use the classic \bibfield{year} field.

\item[pages] See comments on the \opt{nopages} bibliography option in
section \ref{preamble}.

\item[series] See comments on \bibfield{seriesaddon} in section
\ref{datafields}.

\item[titleaddon] Like \bibfield{nameaddon}, this field encloses its
content in brackets. This is especially useful for translated titles.
From \textit{CMOS} \ref{14.99}:

\begin{citebib}
\item \cite{wereszycki1977}; includes a summary in German.
\item \cite{pirumova1977b}
\end{citebib}

\item[type] With the \bibtype{thesis} entry type, Windy City uses
\bibfield{type} to distinguish a Master's thesis from a PhD
dissertation. For the latter, \bibtype{type} should have the value
\textit{phdthesis} or \textit{PhD diss.} For the former, use
\textit{mathesis} or \textit{master's thesis}. No \bibfield{type}
field is needed for the \bibtype{mathesis} and \bibtype{phdthesis}
entry types.

In rare cases, \bibfield{type} is necessary for the \bibtype{article}
and \bibtype{review} entry types when a work's bibliographic
information doesn't allow Windy City to distinguish a journal article
from a magazine or newspaper article. This happens when a magazine or
newspaper article has an issue number. Normally, Windy City would
respond by printing the issue number before the date and enclosing the
date in parentheses. But the issue number needs to go first, and the
date shouldn't be in parentheses. The workaround is to include a
\bibfield{type} field with the value \textit{newsmag}. For an example,
see \textit{CMOS} \ref{14.204} and compare it with the citation of
Beattie in \ref{14.171}.

\end{marglist}

\subsubsection{Other Data Fields}
\label{otherfields}

Some of Windy City's nonstandard data fields are for its internal
handling of cross-referencing. Those fields aren't listed below, as
they're not meant for use in a bibliography database. The rest store
information that's crucial for correct formatting. Without them, quite
a few citations in this document wouldn't match their counterparts in
\textit{CMOS}.

\begin{marglist}

\item[\smash{\tshortstack[l]{blogtitle\\blogsubtitle}}] The name of a
blog goes in \bibfield{blogtitle} and \bibfield{blogsubtitle}. Keep in
mind that blog posts take the usual \bibfield{title} and
\bibfield{subtitle} fields.

\item[editoraddon] Use this field to include additional editorial
information about a book. It's available for \bibtype{book},
\bibtype{incollection}, and \bibfield{review} entry types and their
aliases. (For information on aliases in Windy City, see section
\ref{entrytypes}.) When applicable, its content appears after a book's
editors and translators without intervening punctuation. For an
example, see section \ref{entry}, where the citation of
\textit{Chaucer Life-Records} prints: ``from materials compiled by
John M. Manly and Edith Richert, with the assistance of Lilian J.
Redstone et al.''

\item[seriesaddon] This field is for additional information about a
book's series. That includes information about the run of a series,
such as \textit{2nd ser.} and \textit{n.s.} For examples, see
\textit{CMOS} \ref{14.123} and \ref{14.126}. Keep in mind that, for
journals, which occasionally have a series but no series name,
information like \textit{2nd ser.} and \textit{n.s.} go in the
\bibfield{series} field.

\item[shorthand] Windy City doesn't automatically italicize a
\bibfield{shorthand}. Per \textit{CMOS} 14.60, a shorthand should be
italicized if the title that it abbreviates is italicized. Set it the
bibliography database with \cmd{emph} or \cmd{mkbibemph}. For examples
of a \bibfield{shorthand}, see section \ref{14.59} and the first
citation of \textit{CMOS} in this document's introduction.

\item[shorthandintro] You may override the default announcement of a
\bibfield{shorthand} by adding your preferred content to
\bibfield{shorthandintro}. For an example, see the first citation of
\textit{CMOS} in this document's introduction, where the announcement
is a separate sentence, rather than in parentheses.

\item[shortmaintitle] This field contains the short form of a
\bibfield{maintitle}. It should only be necessary for certain works in
collections. See the citation of \textit{The Complete Tales of Henry
James} in section \ref{multivolume}.

\end{marglist}

\section{Editors, Translators, and Friends}
\label{edtrans}

Windy City offers significant control over the handling of editors,
translators, and the like. Taking advantage of it, however, may not
seem intuitive at first. This section covers all the relevant features
and options.

\subsection{Types of Editors and Translators}
\label{edtransnames}

For the most part, Windy City associates the \bibfield{editor} and
\bibfield{translator} fields with the lowest level title within the
scope of an entry. In most cases, then, you can assign editors and
translators simply by adding the \bibfield{editor} and
\bibfield{translator} fields to a database entry. The entry below is
an example from \textit{CMOS} \ref{14.104}:

\begin{verbatim}
   @Collection{adorno1999,
     author = {Adorno, Theodor W. and Benjamin, Walter},
     title = {The Complete Correspondence, 1928–1940},
     editor = {Lonitz, Henri},
     translator = {Walker, Nicholas},
     address = {Cambridge, MA},
     publisher = {Harvard University Press},
     year = {1999}
   }
\end{verbatim}

\begin{citebib}
\item \cite{adorno1999}
\end{citebib}

\noindent When the style processes this entry, it identifies the
editor and translator of the work with the names in the
\bibfield{editor} and \bibfield{translator} fields. Since the entry
doesn't use the \opt{swaptrans} entry option (see section
\ref{edtranspos}), the resulting output lists the editor and
translator, in that order, after the authors and title.

Cross-referencing introduces a bit more complexity, but the principle
is the same: Within the scope of an entry, the style associates
\bibfield{editor} and \bibfield{translator} with the lowest level
title. In this example from \textit{CMOS} \ref{14.30}, an essay is
cross-referenced to a collection:

\begin{verbatim}
   @InCollection{kaiser1964,
     author = {Kaiser, Ernest},
     title = {The Literature of Harlem},
     shorttitle = {Literature of Harlem},
     crossref = {clarke1964}
   }
   @Collection{clarke1964,
     editor = {Clarke, J. H.},
     title = {Harlem},
     subtitle = {A Community in Transition},
     address = {New York},
     publisher = {Citadel Press},
     year = {1964}
   }
\end{verbatim}

\begin{citebib}
\item \cite{kaiser1964}
\end{citebib}

\noindent Since \bibfield{editor} appears within the
\bibtype{collection} entry, Windy City associates the editor's name
with \textit{Harlem}. If you moved \bibfield{editor} from
\bibtype{collection} to \bibtype{incollection}, the association would
change to \textit{The Literature of Harlem}.

Consider another example:

\begin{citebib}
\item \cite{petrarca1948}
\end{citebib}

\noindent Hans Nachod translated ``The Ascent of Mont Ventoux,'' among
other works in the collection, but not \textit{every} work in the
collection. Thus, the \bibfield{translator} field must fall within the
scope of the \bibtype{incollection} entry:

\begin{verbatim}
   @InCollection{petrarca1948,
     author = {Petrarca, Francesco},
     title = {The Ascent of Mont Ventoux},
     translator = {Nachod, Hans},
     pages = {36–46},
     crossref = {cassirer1948}
   }
   @Collection{cassirer1948,
     editor = {Cassirer, Ernst and Kristeller, Paul Oskar and Randall,
               Jr., John Herman},
     title = {The Renaissance Philosophy of Man},
     address = {Chicago},
     publisher = {University of Chicago Press},
     year = {1948}
   }
\end{verbatim}

\noindent By the same token, since \bibfield{editor} falls within the
scope of \bibtype{collection}, the style associates it with
\textit{The Renaissance Philosophy of Man}.

The style supports three more basic editorial roles: compiler,
reviser, and updater. To assign them, you need to use the
\bibfield{editortype} field. It may help to see the bibliography
database entries for examples in \textit{CMOS} \ref{14.103} and
\ref{14.142}:

\begin{verbatim}
   @Book{schechter2011,
     editor = {Schechter, Harold, and Kurt Brown},
     editortype = {compiler},
     title = {Killer Verse},
     subtitle = {Poems of Murder and Mayhem},
     address = {London},
     publisher = {Everyman Paperback Classics},
     year = {2011}
   }
   @Book{turabian2013,
     author = {Turabian, Kate L.},
     title = {A Manual for Writers of Term Papers, Theses, and
              Dissertations},
     edition = {8},
     editor = {Booth, Wayne C. and Colomb, Gregory G. and
               Williams, Joseph M. and {the University of Chicago
               Press Staff}},
     editortype = {reviser},
     address = {Chicago},
     publisher = {University of Chicago Press},
     year = {2013}
   }
\end{verbatim}

\begin{citebib}
\item \cite{schechter2011}
\item \cite{turabian2013}
\end{citebib}

In addition to the five basic roles (editor, translator, compiler,
reviser, and updater), you may assign names to any pair of them. A
work's editor, for example, can also be its translator, compiler,
reviser, or updater---but no more than one of these. And just as you
can swap the order of the editor and translator roles (again, see
section \ref{edtranspos}), you can swap all other combinations, say,
to identify a work as \textit{compiled and edited by} Jane Doe rather
than \textit{edited and compiled by} her. Simply change the value of
\bibfield{editortype}. A summary of these values appears in Table
\ref{table:ed}.

\begin{table}[H]
\begin{tabular}{@{}r l r l@{}}
\bibfield{editortype} & Assignment & \bibfield{editortype} & Assignment\\
\toprule
compiler & compiler & reviser & reviser\\
comped & compiler and editor & revcomp & reviser and compiler\\
comprev & compiler and reviser & reved & reviser and editor\\
comptrans & compiler and translator & revtrans & reviser and translator\\
compup & compiler and updater & revup & reviser and updater\\
edcomp & editor and compiler & edrev & editor and reviser\\
transcomp & translator and compiler & transrev & translator and reviser\\
\midrule
updater & updater & uptrans & updater and translator\\
upcomp & updater and compiler & edup & editor and updater\\
uped & updater and editor & transup & translator and updater\\
uprev & updater and reviser & &\\
\end{tabular}
\caption{}\label{table:ed}
\end{table}

For examples in \textit{CMOS}, we need to turn to the bibliography.
The first one below shows the output of \textit{reved} in the
\bibfield{editortype} field, the second of \textit{revup}.

\begin{citebib}
\item \cite{fowler1965}
\item \cite{gowers2015}
\end{citebib}

% Try swapping the editor type in the entries above for other values
% in Table \ref{table:ed}. Note that \textit{edrev} is the reverse of
% \textit{reved}---\textit{edited and revised by} instead of
% \textit{revised and edited by}. So, too, for \textit{comprev} and
% \textit{revcomp}, \textit{comptrans} and \textit{transcomp}, and so
% on.

Missing from Table \ref{table:ed} are \textit{editor},
\textit{translator}, \textit{edtrans}, and \textit{transed}. Windy
City has bibliography strings for them, too, but they're meant for the
style's internal use. It determines which of them applies, if any,
based on the contents of the \bibfield{editor} and
\bibfield{translator} fields and the option \opt{swaptrans}. You never
need to use them in a bibliography database.

Three more values of \bibfield{editortype} allow you to assign editors
to higher level titles: \textit{maintitle}, \textit{series}, and
\textit{issuetitle}. Here's an example from \textit{CMOS}
\ref{14.123}:

\begin{verbatim}
   @Book{allen2009,
     author = {Allen, Judith A.},
     title = {The Feminism of Charlotte Perkins Gilman},
     subtitle = {Sexualities, Histories, Progressivism},
     series = {Women in Culture and Society},
     editor = {Stimpson, Catharine R.},
     editortype = {series},
     address = {Chicago},
     publisher = {University of Chicago Press},
     year = {2009}
   }
\end{verbatim}

\begin{citebib}
\item \cite{allen2009}
\end{citebib}

\noindent If there were no \bibfield{editortype} assigning the editor
to the series, the output would list a book's editor. In short, for a
title's compilers, revisers, and updaters and for any title at a
higher level than \bibfield{title}, you need to use
\bibfield{editortype}. Combined with Windy City's support for citing
articles, chapters, books, and other works within books, this scheme
allows for several sets of editors per work, although it does limit
you to just one set of compilers, revisers, and updaters.

What about translators? Unless a work's bibliography database entry
has a \bibfield{translatortype} field with the value
\textit{maintitle}, Windy City associates the names in
\bibfield{translator} with the entry's lowest level title. (Recall
``The Ascent of Mont Ventoux'' above.) You should rarely, and perhaps
never, need more than one set of translators per work. But if you do,
Windy City lets you assign as many as three---as long as you
cross-reference a work with one set of translators to another with
two.

The style has three other name lists for editors: \bibfield{editora},
\bibfield{editorb}, and \bibfield{editorc}. If a work has more than
one set of editors, list editors of the lowest level title in
\bibfield{editor}. Usually, those are editors of a \bibfield{title}.
The next level up, as it were, is for \bibfield{editora}, followed by
\bibfield{editorb}, then \bibfield{editorc}. Remember to include the
appropriate \bibfield{type} field to indicate an editor's role. These
fields are \bibfield{editortype} (for \bibfield{editor}),
\bibfield{editoratype} (for \bibfield{editora}),
\bibfield{editorbtype} (for \bibfield{editorb}), and
\bibfield{editorctype} (for \bibfield{editorc}).

As it happens, Windy City puts a lot of effort into sorting out where
to print the names of editors and translators, so it's a bit more
clever than the previous paragraph suggests. In particular, if you use
\bibfield{editortype} to assign an \bibfield{editor} to a
\bibfield{maintitle} or some other higher level title, and have an
\bibfield{editora} with no corresponding \bibfield{editoratype}, Windy
City will assume that \bibfield{editora} is the editor of the
\bibfield{title}. Regardless, the best practice is to the follow the rule
of thumb described above, reserving \bibfield{editor} for
\bibfield{title} and working up from there.

\subsection{Switching Places and Roles}
\label{edtranspos}

Normally, Windy City lists editors first. But if translators are
listed first on a work's title page, you should reverse the order (see
\textit{CMOS} \ref{14.104}). For that, use the entry option
\opt{swaptrans}. Compare:

\begin{citebib}
\item \cite{doe2010a}
\end{citebib}
\begin{citebib}
\item \cite{doe2010b}
\end{citebib}

If a work has no author, but has an editor, the style will print the
editor's name in the author's position. The same happens if a work has
no author but does have a translator. The following are examples from
\textit{CMOS} \ref{14.103}:

\begin{citebib}
\item \cite[100]{egan2014}
\item \cite[34]{silverstein1974}
\end{citebib}

What if a work has no author but has editors and translators? Since
Windy City gives priority to editors, it defaults to listing editors
in the author's position:

\begin{citebib}
\item \cite{smith2002a}
\end{citebib}

\noindent Use \opt{swaptrans} to reverse them:

\begin{citebib}
\item \cite{smith2002b}
\end{citebib}

Similarly, if a work's editors and translators are the same, the style
will print the editors' role first, as in \textit{edited and
translated by}. Reverse them with \opt{swaptrans}. From \textit{CMOS}
\ref{14.104}:

\begin{citebib}
\item \cite{menchu1999}
\end{citebib}

To make switching roles easier in these cases, you may also use the
relevant \bibfield{editortype} field with a bibliography string from
Table \ref{table:ed}. After all, for compilers, revisers, and
updaters, you need to put a value in \bibfield{editortype} anyway. So,
if you pick one that begins with \textit{trans} (\textit{transcomp},
\textit{transed}, \textit{transrev}, or \textit{transup}), Windy City
will assume that the translator's role goes first.

On rare occasions, you may want to swap the position of an author and
an editor or translator. \textit{CMOS} \ref{14.104} gives an example:

\begin{citebib}
\item \cite{pound1953}
\end{citebib}

\noindent You can get this effect with the entry option
\opt{swapauth}, which works for \bibtype{book}, \bibtype{bookinbook},
\bibtype{collection}, \bibtype{inbook}, \bibtype{incollection},
\bibtype{mvbook}, and \bibtype{mvcollection} entry types. But beware:
Windy City doesn't look ahead to see if there's really an editor or
translator to take the author's place. If it doesn't find one, it will
still print the author's name after the title, leaving the author's
position empty. Also, if it finds both an editor and a translator, and
they're not the same person, it will print the editor's name in the
author's position---again, giving priority to editors. If you want the
translator's name in the author's position, use \opt{swapauth} with
\opt{swaptrans}.

For correct sorting in a bibliography, a work that uses \opt{swapauth}
needs a field like \bibfield{sortname} to sort it by the name of the
editor or translator whose name will occupy the author's position. It
would be nice if Windy City could do this for you, but at present no
feature of \biblatex seems to allow the kind of on-the-fly changes to
sorting that \opt{swapauth} requires. Below is the entry for the
previous example:

\begin{verbatim}
   @Book{pound1953,
     options = {swapauth},
     author = {Pound, Ezra},
     title = {Literary Essays},
     editor = {Eliot, T. S.},
     sortname = {Eliot, T. S.},
     address = {New York},
     publisher = {New Directions},
     year = {1953}
   }
\end{verbatim}

\section{Collections}
\label{collections}

Before you cite a collection or one of its volumes, you need to
consider how you want the citation to look (\textit{CMOS} gives you
options) and whether the collection should count as a single work or
as a composite. These issues have implications for how you structure
entries in your bibliography database and how you use citation
commands.

\subsection{Structuring Citations}
\label{collorder}

A work in a collection usually has a title and perhaps other
publication information that differs from the collection's. When
preparing your bibliography database, the most basic choice to make
about a work in a collection is which publication information has
priority, the collection's or the volume's. Consider an example from
\textit{CMOS} \ref{14.119}:

\begin{citebib}
\item \cite{armstrong2014}
\end{citebib}

\noindent The editor and title of the volume precede the editor and
title of the collection. In standard notes and bibliographies,
\textit{CMOS} gives you the option of reversing this order. (See
especially \textit{CMOS} \ref{14.119}, \ref{14.121}, and \ref{14.122},
and compare \ref{14.144} and \ref{15.41}.) Windy City does as well,
with the entry or preamble option \opt{swapvol}.

\begin{citebib}
\AtNextCitekey{\toggletrue{swapvol}}
\item \cite{armstrong2014}
\AtNextBibliography{\toggletrue{swapvol}}
\end{citebib}

In contexts where information for just one title appears (certain
short citations), the one with priority determines which one
identifies the work. By default, as you can see in the second note
below, the volume has priority:

\begin{citeonly}
\item \cite{armstrong2014}
\item \cite[45]{armstrong2014}
\end{citeonly}

\noindent Whereas with \opt{swapvol}:

\begin{citeonly}
\AtNextCitekey{\toggletrue{swapvol}}
\item \cite{armstrong2014}
\AtNextCitekey{\toggletrue{swapvol}}
\item \cite[45]{armstrong2014}
\end{citeonly}

Notice the volume number in the second note above, separated from the
page by a colon. When the collection has priority, the volume number
should appear in the citation. There are enough examples in
\textit{CMOS} to make that clear. But when the volume has priority,
the volume number is at best optional but probably incorrect. (It
could cause confusion, since the collection is what comes in volumes,
not the volumes themselves.) In any event, Windy City prints the
volume number only when the collection has priority, that is, when
\opt{swapvol} is \textit{true}.

If you want collections to have priority in every case, use the
\opt{swapvol} preamble option, which acts globally. More likely,
though, you'll want to give priority to the collection when citing
some works but not others. In those cases, use the \opt{swapvol} entry
option. A good candidate for the latter is this example from
\textit{CMOS} \ref{14.118}:

\begin{citebib}
\item \cite*{james1963.5}
\end{citebib}

\noindent The volume's title (\textit{1883--1884}) merely indicates a
portion of the collection, somewhat like the volume of an encyclopedia
with the title \textit{D–F}. There's nothing wrong with putting it
first. But putting it second seems more intuitive.

For other works, \opt{swapvol} may seem like a poor choice, even if
the output is formally correct:

\begin{citebib}
\AtNextCitekey{\toggletrue{swapvol}}
\item \cite{barrows1959}
\AtNextBibliography{\toggletrue{swapvol}}
\end{citebib}

\noindent Having the editor's name first and the author's name after
the volume's title looks odd. But it's not wrong. The volume's author
didn't write every volume of the collection. In giving priority to the
collection, then, the first position, where an author's name normally
goes, should go to the collection's editor, not to the volume's
author. By contrast, the default output looks more familiar:

\begin{citebib}
\item \cite{barrows1959}
\end{citebib}

One drawback of \opt{swapvol} is that some works changed by it won't
sort correctly in a bibliography without help. The affected works
include those without listed authors and those in which the volume has
an author but the collection as a whole doesn't---and possibly others.
You can fix them by adding a \bibfield{sortname} field to their
bibliography database entries.

This drawback shouldn't affect \bibfield{bookinbook},
\bibfield{inbook}, \bibfield{incollection}, or \bibfield{letter}
entries when they're cross-referenced to works that are changed by
\opt{swapvol}. They should have their own authors or titles that Windy
City can use for proper sorting. It seems that \textit{CMOS} gives
exactly one such example in \ref{14.120}:

\begin{citebib}
\item \cite[169--71]{king2014}
\item \cite[170]{king2014}
\end{citebib}

\noindent Here's the output with the preamble or entry option
\opt{swapvol}:

\begin{citebib}
\AtNextCitekey{\toggletrue{swapvol}}
\item \cite[169--71]{king2014}
\AtNextCitekey{\toggletrue{swapvol}}
\item \cite[170]{king2014}
\AtNextBibliography{\toggletrue{swapvol}}
\end{citebib}

In neither case does Windy City print the volume number before the
page reference. That's because, with or without \opt{swapvol}, the
citation is directly to the work's title, not to the collection. The
same goes for articles, chapters, books, and other works of this type.
Further, consistent with \textit{CMOS} \ref{14.108} and \ref{15.42},
Windy City cross-references previously cited volumes.

% Due to 'mincrossrefs=2', jones2018 would be added automatically to
% the bibliography at the end of the document because it's referenced
% twice below. However, it wouldn't appear in the refsegment without
% using \nocite.

\begin{citebib}
\item \cite[56]{doe2018}
\item \cite[128]{edwards2018}
\nocite{jones2018}
\end{citebib}

\noindent Now with \opt{swapvol}:

\begin{citebib}
\AtNextCitekey{\toggletrue{swapvol}}
\item \cite[56]{doe2018}
\AtNextCitekey{\toggletrue{swapvol}}
\item \cite[128]{edwards2018}
\AtNextBibliography{\toggletrue{swapvol}}
\nocite{jones2018}
\end{citebib}

How does Windy City determine which entries in a bibliography database
work with \opt{swapvol}? To keep things simple, let's focus on
individual volumes of a collection, rather than works collected in
them, like articles and chapters. Below are entries for an earlier
example:

\begin{verbatim}
   @InBook{armstrong2014,
     editor = {Armstrong, Tenisha},
     title = {To Save the Soul of America, January 1961–August
              1962},
     shorttitle = {To Save the Soul of America},
     volume = {7},
     year = {2014},
     crossref = {carson1992}
   }
   @Collection{carson1992,
     editor = {Carson, Clayborne},
     title = {The Papers of Martin Luther King, Jr.},
     volumes = {14},
     address = {Berkeley},
     publisher = {University of California Press},
     year = {1992–}
   }
\end{verbatim}

A \bibtype{bookinbook} or \bibtype{inbook} entry for a volume works
with \opt{swapvol} if it has a \bibfield{title}, a \bibfield{volume},
and another \bibfield{title} inherited from a cross-referenced entry,
and if it lacks certain fields that it shouldn't have, such as a
\bibfield{maintitle} field or a \bibfield{volume} field inherited from
the cross-referenced entry. You can find the exact details in
\file{windycity.bbx}. Just remember that entries like these should
always be cross-referenced to a \bibtype{collection} or one of its
aliases. Since Windy City treats them the same, it makes no difference
whether you designate entries for volumes as \bibtype{bookinbook} or
\bibtype{inbook}.

With one exception, you can get the same output with a \bibtype{book},
\bibtype{collection}, \bibtype{mvbook}, or \bibtype{mvcollection}
entry. As before, the choice of entry type is arbitrary. All four are
equivalent. Here's a counterpart to the previous example using
\bibtype{collection}:

\begin{verbatim}
   @Collection{carson2014,
     editor = {Armstrong, Tenisha},
     title = {To Save the Soul of America, January 1961–August
              1962},
     volume = {7},
     maintitle = {The Papers of Martin Luther King, Jr.},
     editora = {Carson, Clayborne},
     editoratype = {maintitle},
     volumes = {14},
     address = {Berkeley},
     publisher = {University of California Press},
     year = {2014}
   }
\end{verbatim}

These entries work with \opt{swapvol} if they have \bibfield{volume}
and \bibfield{maintitle} fields, among other conditions. You may
prefer them over \bibfield{bookinbook} or \bibfield{inbook} entries if
you only intend to cite one volume of the collection and have no need
to cite the collection as a whole. You risk trouble, though, if the
volumes of the collection have different publication dates, as they do
for \textit{The Papers of Martin Luther King, Jr}. If the publication
information for this volume goes last, the bibliography and long
citation should list the publication date as \textit{2014}. If the
publication information for the collection goes last, that date is
\textit{1992–}. A \bibtype{book} or \bibtype{collection} entry,
however, has just one field for a publication date, so using
\opt{swapvol} on \bibfield{carson2014} would print \textit{2014}, not
\textit{1992–}. Otherwise, the entry types are interchangeable. You
can find examples of both approaches in \file{windycity.bib}.

Certain entries cross-referenced to entries that work with
\opt{swapvol} also work with it. Those entry types are limited to
\bibtype{bookinbook}, \bibtype{inbook}, \bibtype{incollection}, or
\bibtype{letter}. For the letter cited above:

\begin{verbatim}
   @Letter{king2014,
     author = {King, Jr., Martin Luther},
     title = {Unpublished letter to the editor of the
              \emph{Afro-American} (Washington, DC)},
     crossref = {armstrong2014}
   }
\end{verbatim}

\noindent You may cross-reference this entry to
\bibfield{armstrong2014} or \bibfield{carson2014}. The only difference
is that, with \opt{swapvol}, cross-referencing to
\bibfield{carson2014} once more gives the wrong publication date for
the collection. With many collections, this wouldn't be an issue.
Convenience, then, is perhaps the only consideration in choosing how
you do it. If you plan to cite more than one volume of a collection,
and different works in those volumes or the collection as a whole,
cross-referencing to \bibtype{bookinbook} or \bibtype{inbook} entries
is probably simpler. The other way is simpler if you plan to cite just
one work in one volume.

As mentioned in \textit{CMOS} \ref{15.41}, reference lists should give
priority to the volume, not to the collection. Nevertheless, Windy
City doesn't prevent you from using \opt{swapvol} with \opt{reflist}.
Either way, remember that when a volume and a collection have
different publication dates, a reference list entry prints both dates,
which in turn requires you to use the \bibtype{bookinbook} or
\bibtype{inbook} approach.

\begin{refonly}
\nocite{armstrong2014}
\end{refonly}

A final issue concerns whether you want long citations to give a
work's total number of volumes. Whereas entries in a bibliography
almost always give that number (the exceptions are those that work
with \opt{swapvol}), doing so in notes is optional (per \textit{CMOS}
\ref{14.118}).

Here's what Windy City does: Apart from entries that work with
\opt{swapvol}, it prints the \bibfield{volumes} field in long
citations only if the \bibfield{postnote} field is empty. That's not a
requirement of \textit{CMOS}. But there are several reasons for
preferring this approach. If the \bibfield{postnote} is empty,
printing the total preserves the correspondence between the long
citation and its entry in the bibliography. This prevents the
appearance that something has gone wrong when an element in one is
missing from the other. If the \bibfield{postnote} isn't empty, it
likely contains information that, as with page references, should also
include a volume number (see section \ref{multivolume}). Including a
volume number makes it less important, arguably, to include the total
number of volumes. The main benefit, anyway, is that Windy City's
approach is consistent with citations of \textit{The Lisle Letters} in
\textit{CMOS} \ref{14.117} and \ref{14.118}. No simpler rule would be.

Confused? All you need to know is that, barring exceptions for
\opt{swapvol}, if you want to print a work's total number of volumes
without regard to the \bibfield{postnote}, use the entry option
\opt{listvols}. Windy City uses it to match some examples in
\textit{CMOS}, such as in \ref{14.59}:

\begin{citeonly}
\item \cite[1:126]{shurtleff1853}
\end{citeonly}

\subsection{Collections as Single Works}
\label{multivolume}

Although its discussion is a bit obscure, \textit{CMOS} treats some
multivolume collections as single works---but only, it seems, if every
volume of the collection has the same title and publication date. To
illustrate the distinction between a collection that counts as a
single work and one that doesn't, \textit{CMOS} gives the following
examples in \ref{14.118}:

\begin{citeonly}
\item \cite[4:243]{byrne1981}
\item \cite*[32--33]{james1963.5}
\item \cite[4:245]{byrne1981}
\item \cite*[34]{james1963.5}
\end{citeonly}

In citations of \textit{The Lisle Letters}, volume numbers and pages
are separated by a colon. With \textit{The Complete Tales of Henry
James}, only the second citation follows this pattern. In the first,
the volume number appears earlier, after the editor's name. Why?
Apparently, \textit{The Lisle Letters} count as a single, multivolume
work because every volume has the same title and publication date. Not
so \textit{The Complete Tales of Henry James}. Its volumes have
different titles and publication dates.

To get the right output, your bibliography database and citations need
to reflect this distinction. Think of it this way: If a multivolume
collection meets the criteria of a single work (all volumes have the
same title and publication date), your bibliography database should
have just one entry to which all citations of the collection refer,
regardless of whether they cite particular volumes or the collection
as a whole. Here's the entry for \textit{The Lisle Letters}:

\begin{verbatim}
   @Collection{byrne1981,
     editor = {Byrne, Muriel St. Clare},
     title = {The Lisle Letters},
     volumes = {6},
     address = {Chicago},
     publisher = {University of Chicago Press},
     year = {1981}
   }
\end{verbatim}

To cite a particular volume of the collection, include the volume
number in the citation's \bibfield{postnote}. For citations of pages,
remember the format from \textit{CMOS} \ref{14.118}: Volume numbers
and pages are separated with a colon. Here's the source for the first
citation of \textit{The Lisle Letters}:

\begin{verbatim}
   \cite[4:243]{byrne1981}
\end{verbatim}

\noindent To cite a volume by itself, without a page reference, or to
cite chapters, sections, and other parts of the work, remember to use
the appropriate abbreviations (for some examples, see \textit{CMOS}
\ref{14.120}, 15.23, and \ref{15.41}):

\begin{verbatim}
   \cite[vol. 3, chap. 9]{byrne1981}
\end{verbatim}

What if the \bibfield{postnote} is empty? Windy City assumes that you
mean to cite the collection as a whole. As such, the first, long
citation of the work prints the collection's total number of volumes.
Subsequent entries indicate the collection in whatever short form
corresponds to the preamble options. The following shows the default
output for two such citations of the collection:

\begin{citeonly}
\item \cite{byrne1981}
\item \cite{byrne1981}
\end{citeonly}

For collections like \textit{The Complete Tales of Henry James}, which
don't count as single works, every volume needs to have its own entry
in the bibliography database. Here's the entry for the volume cited in
\textit{CMOS} \ref{14.118}:

\begin{verbatim}
   @Collection{james1963.5,
     options = {swapvol},
     author = {James, Henry},
     title = {1883–1884},
     volume = {5},
     maintitle = {The Complete Tales of Henry James},
     shortmaintitle = {Complete Tales of Henry James},
     editor = {Edel, Leon},
     editortype = {maintitle},
     volumes = {12},
     address = {London},
     publisher = {Rupert Hart-Davis},
     year = {1963}
   }
\end{verbatim}

Since the volume number is part of the entry and needs to print in
different places depending on the context, don't include it in the
\bibfield{postnote}. Let Windy City handle it. Below is the source for
the first and second citations of \textit{The Complete Tales Henry
James}:

\begin{verbatim}
   \item \cite*[32--33]{james1963.5}
     ...
   \item \cite*[34]{james1963.5}
\end{verbatim}

\noindent The first citation prints the volume number after the
editor's name, long before the \bibfield{postnote}. The second prints
it just before the \bibfield{postnote}, separated from the page by a
colon:

\begin{citeonly}
\item[2.] \cite*[32--33]{james1963.5}
\item[\ldots]
\item[4.] \cite*[34]{james1963.5}
\end{citeonly}

Neither type of collection uses cross-referencing in the bibliography
database. For \textit{The Lisle Letters}, cross-referencing would
introduce needless complexity. A single work should have a single
entry, not multiple, cross-referenced entries. For \textit{The
Complete Tales Henry James}, cross-referencing would result in errors
because, with different titles and publication dates, not all
publication data for the collection is true of each volume. To cite
the collection as a whole, as in \textit{CMOS} \ref{14.117}, add a
separate entry:

\begin{verbatim}
   @Collection{james1962,
     author = {James, Henry},
     title = {The Complete Tales of Henry James},
     shorttitle = {Complete Tales of Henry James},
     editor = {Edel, Leon},
     volumes = {12},
     address = {London},
     publisher = {Rupert Hart-Davis},
     year = {1962–64}
   }
\end{verbatim}

\subsection{Works \textit{in} Volumes or \textit{as} Volumes}

Many examples so far show the use of the \bibfield{volume} field. What
they don't quite show is that, as with editors and translators, Windy
City associates \bibfield{volume} with an entry's lowest level title.
Usually, this doesn't require any thought when preparing a
bibliography database. The \bibfield{volume} field goes where you'd
expect it to. But if you're not careful, you could end up with errors,
especially when citing books in collections.

\begin{verbatim}
   @InBook{spinoza1900.1.1,
     title = {A Theological-Political Treatise},
     crossref = {spinoza1900.1}
   }
   @Collection{spinoza1900.1,
     author = {Spinoza, Benedict de},
     title = {The Chief Works of Benedict de Spinoza},
     edition = {revised},
     translator = {Elwes, R. H. M.},
     volume = {1},
     address = {London},
     publisher = {George Bell {and} Sons},
     year = {1900}
   }
\end{verbatim}

In the entries above, Windy City associates \bibfield{volume} with
\textit{The Chief Works of Benedict de Spinoza}, not with \textit{A
Theological-Political Treatise}. The latter, it assumes, isn't volume
one of the collection but a work contained \textit{in} volume one,
presumably with other works. Windy City formats the citation
accordingly:

\begin{citebib}
\item \cite{spinoza1900.1.1}
\end{citebib}

\noindent If \bibfield{volume} were within the scope of
\bibfield{spinoza1900.1.1}, Windy City would assume that \textit{A
Theological-Political Treatise} is volume one of \textit{The Chief
Works of Benedict de Spinoza}. The same error would occur if you put
all the data for the citation into a \bibfield{collection} entry.
Windy City would associate \bibfield{volume} with \bibfield{title},
not with \bibfield{maintitle}. The upshot is that correctly citing a
work like this requires cross-referencing an \bibtype{inbook} or
\bibtype{bookinbook} entry to a \bibtype{collection}, \bibtype{book},
\bibtype{mvbook}, or \bibtype{mvcollection} entry. (Again, the
alternatives give the same output.) Only then would Windy City
associate \bibfield{volume} with the right \bibfield{title}.

\section{Examples from \emph{CMOS} Chap. 14, ``Notes and
Bibliography''}
\label{notes}

Examples in this section reproduce those in \textit{CMOS} chapter 14.
To help with cross-checking, subsection numbers and headings are from
\textit{CMOS}.

\titleformat{\subsubsection}{\normalsize\it}{\arabic{subsection}.\arabic{subsubsection}}{2ex}{}
\subsection{Basic Format, with Examples and Variations}
\setcounter{subsection}{14}

\setcounter{subsubsection}{22}
\subsubsection{Notes and bibliography—examples and variations}
% 14.23 Notes and bibliography—examples and variations
\label{14.23}

\begin{citebib}
\item \cite[87-88]{strayed2012}
\item \cite[261, 265]{strayed2012}
\item \cite[32]{daum2015}
\item \cite[134--35]{daum2015}
\item \cite[188]{grazer2015}
\item \cite[190]{grazer2015}
\item \cite[242--55]{garcia1988}
\item \cite[33]{garcia1988}
\item \cite[310]{gould1984a}
\item \cite[309]{gould1984a}
\item \cite[484--85]{bagley2015}
\item \cite[501]{bagley2015}
\item \cite[311]{liu2015}
\item \cite[312]{liu2015}
\end{citebib}

\setcounter{subsection}{1}
\subsection{Notes}
\setcounter{subsection}{14}

\setcounter{subsubsection}{29}
\subsubsection{Basic structure of the short form}
% 14.30: Basic structure of the short form
\label{14.30}

\begin{citebib}
\item \cite[24--25]{morley1995}
\item \cite{schwartz1992}
\item \cite{kaiser1964}
\item \cite[43]{morley1995}
\item \cite[138]{schwartz1992}
\item \cite[189--90]{kaiser1964}
\end{citebib}

\setcounter{subsubsection}{33}
\subsubsection{Shortened citations versus ``ibid''}
% 14.34:
\label{14.34}

On short citations and the \opt{short} and \opt{ibid} preamble
options, see sections \ref{short} and \ref{preamble}.

\begin{citeonly}
\AtNextCitekey{\toggletrue{short}\toggletrue{firstshort}}
\item \cite[3]{morrison2004a}
\AtNextCitekey{\toggletrue{short}}
\item \cite[18]{morrison2004a}
\AtNextCitekey{\toggletrue{short}}
\item \cite[18]{morrison2004a}
\AtNextCitekey{\toggletrue{short}}
\item \cite[24--26]{morrison2004a}
\AtNextCitekey{\toggletrue{short}\toggletrue{firstshort}}
\item \cite[401-2]{morrison2004b}
\AtNextCitekey{\toggletrue{short}}
\item \cite[433]{morrison2004b}
\AtNextCitekey{\toggletrue{short}\toggletrue{firstshort}}
\item \cite[37--38]{diaz2008}
\AtNextCitekey{\toggletrue{short}}
\item \cite[403]{morrison2004b}
\AtNextCitekey{\toggletrue{short}}
\item \cite[152]{diaz2008}
\AtNextCitekey{\toggletrue{short}}
\item \cite[201-2]{diaz2008}
\AtNextMultiCite{\toggletrue{short}}
\item \cites[240]{morrison2004b}[32]{morrison2004a}
\AtNextCitekey{\toggletrue{short}} \item \cite[33]{morrison2004a}
\end{citeonly}

\noindent With options \opt{short} and \opt{ibid}:

\begin{citeonly}
\AtNextCitekey{\toggletrue{short}\toggletrue{firstshort}\toggletrue{ibid}}
\item \cite[3]{morrison2004a}
\AtNextCitekey{\toggletrue{short}\toggletrue{ibid}}
\item \cite[18]{morrison2004a}
\AtNextCitekey{\toggletrue{short}\toggletrue{ibid}}
\item \cite[18]{morrison2004a}
\AtNextCitekey{\toggletrue{short}\toggletrue{ibid}}
\item \cite[24--26]{morrison2004a}
\AtNextCitekey{\toggletrue{short}\toggletrue{firstshort}\toggletrue{ibid}}
\item \cite[401-2]{morrison2004b}
\AtNextCitekey{\toggletrue{short}\toggletrue{ibid}}
\item \cite[433]{morrison2004b}
\AtNextCitekey{\toggletrue{short}\toggletrue{firstshort}\toggletrue{ibid}}
\item \cite[37--38]{diaz2008}
\AtNextCitekey{\toggletrue{short}\toggletrue{ibid}}
\item \cite[403]{morrison2004b}
\AtNextCitekey{\toggletrue{short}\toggletrue{ibid}}
\item \cite[152]{diaz2008}
\AtNextCitekey{\toggletrue{short}\toggletrue{ibid}}
\item \cite[201-2]{diaz2008}
\AtNextMultiCite{\toggletrue{short}\toggletrue{firstshort}}
\item \cites[240]{morrison2004b}[32]{morrison2004a}
\AtNextCitekey{\toggletrue{short}\toggletrue{ibid}}
\item \cite[33]{morrison2004a}
\end{citeonly}

\setcounter{subsubsection}{53}
\subsubsection{Source notes for previously published material}
% 14.54 Source notes for previously published material

\begin{citebib}
\item \cite[Reprinted with permission from][15–64]{shapin1996}
\item \cite*[Originally published as][22–69, © 1992 by The University of
Chicago. All rights reserved. Reprinted by permission]{manet1992}
\end{citebib}

\setcounter{subsubsection}{56}
\subsubsection{Several citations in one note}
% 14.57 Several citations in one note
\label{14.57}

See \ref{cust.cmd} for information on how to reproduce this example.

\begin{quote} Only when we gather the work of several
scholars---Walter Sutton's explications of some of Whitman's shorter
poems; Paul Fussell's careful study of structure in ``Cradle''; S. K.
Coffman's close readings of ``Crossing Brooklyn Ferry'' and ``Passage
to India''; and the attempts of Thomas I. Rountree and John Lovell,
dealing with ``Song of Myself'' and ``Passage to India,''
respectively, to elucidate the strategy in ``indirection''---do we
begin to get a sense of both the extent and the specificity of
Whitman's forms.\footnotemark[1] \end{quote}

\begin{citeonly}
\item \idemcites{sutton1959,fussell1962,coffman1954,coffman1955,rountree1958}[and][]{lovell1960}
\end{citeonly}

\setcounter{subsubsection}{58}
\subsubsection{Abbreviations for frequently cited works}
% 14.59 Abbreviations for frequently cited works
\label{14.59}

For information on shorthands, see section \ref{otherfields}.

\begin{citebib}
% There's no way to suppress these fields in the bibliography:
%\AtNextCitekey{\clearfield{origtitle}%
%  \clearlist{origlocation}%
%  \clearlist{origpublisher}%
%  \clearfield{origyear}}
%\item \cite[368]{furet1999}
\item \cite[1:126]{shurtleff1853}
\item \cite[2:330]{shurtleff1853}
\end{citebib}

\setcounter{subsubsection}{67}
\subsubsection{The 3-em dash for one repeated name}
% 14.68 The 3-em dash for one repeated name

A 3-em dash replaces names in the author's position of a citation in
consecutive citations on the same page. Thus, whether the example
below exactly reproduces that in \textit{CMOS} 14.68 depends in part
on whether a page break falls somewhere within the list.

\begin{bibonly}
\nocite{judt1996,judt2008,judt1989,squire1983,squire1987}
\end{bibonly}

\setcounter{subsection}{2}
\subsection{Author's Name}
\setcounter{subsection}{14}

\setcounter{subsubsection}{74}
\subsubsection{One author}
% 14.75 One author

\begin{citebib}
\item \cite[33]{shields2013}
\item \cite[677]{chun2015}
\item \cite[5]{mccune2014}
\item \cite[100--101]{shields2013}
\item \cite[681]{chun2015}
\item \cite[105--11]{mccune2014}
\end{citebib}

\subsubsection{Two or more authors (or editors)}
% 14.76 Two or more authors (or editors)
\label{14.76}

\begin{citebib}
\item \cite[xvi]{sorrells2015}
\item \cite[20--21]{levitt2005}
\item \cite[422]{umbers2015}
\item \cite[xx-xxi]{sorrells2015}
\item \cite[158]{gmuca2015}
\item \cite[160]{gmuca2015}
\end{citebib}

\subsubsection{Two or more authors (or editors) with same family name}
% 14.77 Two or more authors (or editors) with same family name

\begin{citebib}
\item \cite[14]{kendris2010}
\item \cite[27--28]{kendris2010}
\end{citebib}

\subsubsection{Author's name in title}
% 14.78 Author's name in title
\label{14.78}

On the use of starred citation commands, see section \ref{std.cmd}.

\begin{citebib}
\item \cite*[233]{franklin1868}
\item \cite*[234]{franklin1868}
\end{citebib}

\subsubsection{No listed author (anonymous works)}
% 14.79 No listed author (anonymous works)
\label{14.79}

See section \ref{stdfields} on the use of the \bibfield{authtype}
field for anonymous works.

\begin{citebib}
\item \cite{anon1610}
\item \cite{anon1547}
\item \cite{horsley1796}
\item \cite{hawkes1834}
\end{citebib}

\subsubsection{Pseudonyms}
% 14.80 Pseudonyms

\begin{citebib}
\item \cite{akmuckraker2008}
\item \cite{carre1982}
\item \cite{stendhal1925}
\end{citebib}

\subsubsection{Cross-references for pseudonyms}
% 14.81 Cross-references for pseudonyms
\label{14.81}

For information on how to add entries that cross-reference others, see
section \ref{entrytypes}. For examples of automatic cross-referencing
in a bibliography, see \ref{14.108} and \ref{15.42}.

\begin{bibonly}
\nocite{ashe,creasey1976,creasey1978,creasey1966,morton,york}
\end{bibonly}

\subsubsection{Alternative real names}
% 14.82 Alternative real names
\label{14.82}

For information on how to get the last entry below, see section
\ref{entrytypes}.

\begin{bibonly}
\nocite{doniger2000,oflaherty}
\end{bibonly}

\subsubsection{Authors known by a given name}
% 14.83 Authors known by a given name

\begin{citebib}
\item \cite{elizabeth2000}
\end{citebib}

\subsubsection{Organization as author}
% 14.84 Organization as author
\label{14.84}

If an organization is the work's author, remember to add an extra pair
of brackets around the name of the organization in your bibliography
database.

\begin{citebib}
\item \cite{iso1997}
\AtNextCitekey{\clearfield{shorthand}}
\item \cite{chicago2017}
\end{citebib}

\setcounter{subsection}{3}
\subsection{Title of Work}
\setcounter{subsection}{14}

\setcounter{subsubsection}{88}
\subsubsection{Subtitles in cited works and the use of the colon}
% 14.89 Subtitles in cited works and the use of the colon

\begin{citebib}
\item \cite{gladwell2013}
\end{citebib}

\subsubsection{Two subtitles in a cited work}
% 14.90 Two subtitles in a cited work

\begin{citebib}
\item \cite{sereny1999}
\end{citebib}

\setcounter{subsubsection}{91}
\subsubsection{``And other stories'' and such}
% 14.92 ``And other stories'' and such

\begin{citebib}
\item \cite[104]{maclean1976}
\end{citebib}

\subsubsection{Dates in titles of cited works}
% 14.93 Dates in titles of cited works

\begin{citebib}
\item \cite{beiser2014}
\end{citebib}

\subsubsection{Quoted titles and other terms within cited titles of works}
% 14.94 Quoted titles and other terms within cited titles of works

\begin{citebib}
\item \cite{levitt2014}
\item \cite{mchugh1980}
\end{citebib}

\subsubsection{Italicized titles and other terms within cited titles of works}
% 14.95 Italicized titles and other terms within cited titles of works

\begin{citebib}
\item \cite{vanwagenen1973}
\end{citebib}

\subsubsection{Question marks or exclamation points in titles of cited works}
% 14.96 Question marks or exclamation points in titles of cited works

\begin{citebib}
\item \cite[63]{berra2002}
\item \cite[183]{oram2007}
\item \cite[778]{tessler2014}
\item \cite[336]{batson1990}
\item \cite[55--56]{berra2002}
\item \cite[184]{oram2007}
\item \cite[780]{tessler2014}
\item \cite[337]{batson1990}
\end{citebib}

\setcounter{subsubsection}{98}
\subsubsection{Translated titles of cited works}
% 14.99 Translated titles of cited works
\label{14.99}

\begin{citebib}
\item \cite{wereszycki1977}; includes a summary in German.
\item \cite[272]{kern1938}
\item \cite{pirumova1977b}
%\AtNextCitekey{\clearfield{shorthand}}
\item \cite{furet1999}
\end{citebib}

\setcounter{subsection}{4}
\subsection{Books}
\setcounter{subsection}{14}

\setcounter{subsubsection}{100}
\subsubsection{Form of author's name and title of book in source citations}
% 14.101 Form of author’s name and title of book in source citations

\begin{citebib}
\item \cite[79--80]{gawande2014}
\item \cite[191]{gawande2014}
\end{citebib}

\setcounter{subsubsection}{102}
\subsubsection{Editor in place of author}
% 14.103 Editor in place of author
\label{14.103}

\begin{citebib}
\item \cite[100]{egan2014}
\item \cite[33]{schechter2011}
\item \cite[34]{silverstein1974}
\item \cite[301--2]{egan2014}
\item \cite[54--56]{schechter2011}
\item \cite[38]{silverstein1974}
\end{citebib}

\subsubsection{Editor or translator in addition to author}
% 14.104 Editor or translator in addition to author
\label{14.104}

On how to make an author and editor swap places, see section
\ref{edtranspos}.

\begin{citebib}
\item \cite{bonnefoy1995}
\item \cite{menchu1999}
\item \cite{adorno1999}
\item \cite{pound1953}
\end{citebib}

\subsubsection{Other contributors listed on the title page}
% 14.105 Other contributors listed on the title page
\label{14.105}

\begin{citebib}
\item \cite{chaucer1966}
\item \cite{cullen1961}
\item \cite{hayek1994}
\item \cite{prather1998}
\item \cite{williams1990}
\end{citebib}

\subsubsection{Chapter in a single-author book}
% 14.106 Chapter in a single-author book

\begin{citebib}
\item \cite[211]{brower2015.8}
\item \cite{samples2006.7}
\item \cite[30-31]{samples2006.7}
\end{citebib}

\subsubsection{Contribution to a multiauthor book}
% 14.107 Contribution to a multiauthor book

\begin{citebib}
\item \cite[325]{miller2014}
\item \cite{ellet1968}
\end{citebib}

\subsubsection{Several contributions to the same multiauthor book}
% 14.108 Several contributions to the same multiauthor book
\label{14.108}

% Due to 'mincrossrefs=2', angle1968 and zukowsky1987 would be added
% automatically to the bibliography at the end of the document because
% each is referenced twice below. However, they wouldn't appear in the
% refsegment without using \nocite.

% Unlike article entries, incollection entries don't automatically
% print page ranges. That seems right, though CMOS isn't entirely
% clear on this point. See note for 'cite:pages' in windycity.cbx.

\begin{citebib}
\item \cite[84--87]{keating1968}
\item \cite[362--70]{lippincott1968}
\item \cite[107--19]{draper1987}
\item \cite[189--207]{harrington1987}
\nocite{angle1968,zukowsky1987}
\end{citebib}

\subsubsection{Book-length work within a book}
% 14.109 Book-length work within a book
\label{14.109}

\begin{citebib}
\item \cite{bernard1990a}
\item \cite{updike1995a}
\end{citebib}

\subsubsection{Introductions, prefaces, afterwords, and the like}
% 14.110 Introductions, prefaces, afterwords, and the like

% The bibliography database entry for Toni Morrison's foreword to
% \textit{Song of Solomon} doesn't list a page range. Unlike the
% second example, which shows an introduction by authors different
% than that of the main text, it doesn't need to.

\begin{citebib}
\item \cite{morrison2004b.f}
\item \cite{mansfield2000}
\end{citebib}

\subsubsection{Letters in published collections}
% 14.111 Letters in published collections

\begin{citebib}
\item \cite[133--34]{adams1867}
\item \cite{jackson1676}
\end{citebib}

\setcounter{subsubsection}{112}
\subsubsection{Editions other than the first}
% 14.113 Editions other than the first

\begin{citebib}
\item \cite[401--2]{einsohn2011}
\item \cite[101]{boudett2013}
\item \cite{strunk2000}
\end{citebib}

\subsubsection{Reprint editions and modern editions}
% 14.114 Reprint editions and modern editions
\label{14.114}

% You can have at most one \bibfield{origdate} per entry. So, if you
% cite a work in a collection, say, an article or book in an
% anthology, the style assumes that \bibfield{origdate} is for the
% collection, not for the individual work.

\begin{citebib}
\item \cite[152--53]{barzun1994}
\item \cite{bahadur2014}
% CMOS prints the original date in the wrong place:
%\item \cite{emerson1985}
\item \cite{schweitzer1966}
\end{citebib}

\subsubsection{Microform editions}
% 14.115 Microform editions

Farwell's citation fails to match \textit{CMOS} because the
\bibfield{howpublished} field, which seems like the best choice to
contain \textit{microfiche}, follows the \bibfield{postnote} field,
which contains \textit{p. 67, 3C12}. This order is necessary
elsewhere, such as in \textit{CMOS} \ref{14.163}.

\begin{citebib}
\item \cite[p. 67, 3C12]{farwell1997}
\item \cite{tauber1958}
\end{citebib}

\noindent Citing Farwell with the preamble or entry option
\opt{swapvol}:

\begin{citebib}
\AtNextCitekey{\toggletrue{swapvol}}
\item \cite[p. 67, 3C12]{farwell1997}
\AtNextBibliography{\toggletrue{swapvol}}
\end{citebib}

\setcounter{subsubsection}{116}
\subsubsection{Citing a multivolume work as a whole}
% 14.117 Citing a multivolume work as a whole
\label{14.117}

\begin{citebib}
\item \cite{aristotle1983}
\item \cite{byrne1981}
\item \cite{james1962}
\end{citebib}

\subsubsection{Citing a particular volume in a note}
% 14.118 Citing a particular volume in a note
\label{14.118}

For a discussion of how to handle these types of works, see section
\ref{multivolume}.

\begin{citebib}
\item \cite[4:243]{byrne1981}
\item \cite*[32--33]{james1963.5}
\item \cite[4:245]{byrne1981}
\item \cite*[34]{james1963.5}
\end{citebib}

\subsubsection{Citing a particular volume in a bibliography}
% 14.119 Citing a particular volume in a bibliography
\label{14.119}

\begin{citebib}
\item \cite{armstrong2014}
\end{citebib}

\noindent With preamble or entry option \opt{swapvol}:

\begin{citebib}
\AtNextCitekey{\toggletrue{swapvol}}
\item \cite{armstrong2014}
\AtNextBibliography{\toggletrue{swapvol}}
\end{citebib}

\subsubsection{Chapters and other parts of individual volumes}
% 14.120 Chapters and other parts of individual volumes
\label{14.120}

There are some peculiarities with the first example. In the printed
edition, but not online, \textit{CMOS} errs in having \textit{.ed}
rather than \textit{edited by} in the bibliography and neglects to
invert the author's name. More worrisome are the striking differences
between the note and bibliography. They may represent alternative ways
of formatting the data, as other examples do. But the note seems
inconsistent with \textit{CMOS} \ref{14.118}, and so doesn't make much
sense as an alternative. Windy City ignores it and in both cases
follows the example of the bibliography.

\begin{citebib}
\item \cite[180]{chen2010.3}
\item \cite[169--71]{king2014}
\end{citebib}

\noindent Citing King with the preamble or entry option
\opt{swapvol}:

\begin{citebib}
\AtNextCitekey{\toggletrue{swapvol}}
\item \cite[169--71]{king2014}
\AtNextBibliography{\toggletrue{swapvol}}
\end{citebib}

\subsubsection{One volume in two or more books}
% 14.121 One volume in two or more books
\label{14.121}

\begin{citebib}
\item \cite[351]{lach1977}
\item \cite{harley1994}
\end{citebib}

\noindent With preamble or entry option \opt{swapvol}:

\begin{citebib}
\AtNextCitekey{\toggletrue{swapvol}}
\item \cite[351]{lach1977}
\AtNextCitekey{\toggletrue{swapvol}}
\item \cite{harley1994}
\AtNextBibliography{\toggletrue{swapvol}}
\end{citebib}

\subsubsection{Authors and editors of multivolume works}
% 14.122 Authors and editors of multivolume works
\label{14.122}

For issues surrounding the first example below, see section
\ref{collorder}.

\begin{citebib}
\item \cite{barrows1959}
\item \cite{donne1995}
\end{citebib}

\noindent With preamble or entry option \opt{swapvol}:

% As explained in \ref{collorder}, works like barrows1959 sort
% correctly in the default format but incorrectly with the option
% 'swapvol'. The point of '\newrefcontext' below is to work around
% this problem.

\begin{citebib}
\AtNextCitekey{\toggletrue{swapvol}}
\item \cite{barrows1959}
\AtNextCitekey{\toggletrue{swapvol}}
\item \cite*{donne1995}
\newrefcontext[sorting=reverse]
\AtNextBibliography{\toggletrue{swapvol}}
\end{citebib}
\newrefcontext[sorting=nty]

\subsubsection{Series titles, numbers, and editors}
% 14.123 Series titles, numbers, and editors
\label{14.123}

\begin{citebib}
\item \cite{lei2014}
\item \cite{mazrim2011}
\item \cite{wauchope1950}
\item \cite{allen2009}
\end{citebib}

\subsubsection{Series or multivolume work?}
% 14.124 Series or multivolume work?

In the second example, \textit{vol. 6} refers to the book's series,
Readings in Western Civilization. Usually, the number of a series
isn't recorded as a volume, so Windy City doesn't use the
\bibfield{volume} field for them. Instead, it uses the
\bibfield{number} field, which it prints with no preceding
abbreviation. Add one to the field as necessary. The \bibfield{number}
field for the second example below contains \textit{vol. 6}.

\begin{citebib}
\item \cite{boyer1986}
\item \cite{cochrane1987}
\end{citebib}

\setcounter{subsubsection}{125}
\subsubsection{``Old series'' and ``new series''}
% 14.126 ``Old series'' and ``new series''
\label{14.126}

\begin{citebib}
\item \cite{boxer1953}
\item \cite{palmatary1950}
\end{citebib}

\subsubsection{Place, publisher, and date}
% 14.127 Place, publisher, and date

\begin{citebib}
\item \cite{woolf1927}
\end{citebib}

\subsubsection{Place and date only, for books published before 1900}
% 14.128 Place and date only, for books published before 1900

\begin{citebib}
\item \cite{goldsmith1766}
\item \cite{cervantes1605}
\end{citebib}

\setcounter{subsubsection}{131}
\subsubsection{No place of publication}
% 14.132 No place of publication

\begin{citeonly}
\item \cite{windsor1910}
\item \cite{vliet1890}
\end{citeonly}

\setcounter{subsubsection}{136}
\subsubsection{Self-published or privately published books}
% 14.137 Self-published or privately published books

\begin{citebib}
\item \cite{karavaev2015}
\item \cite{shumaker2014}
\end{citebib}

\setcounter{subsubsection}{139}
\subsubsection{Copublication}
% 14.140 Copublication

\begin{citebib}
\item \cite{strauss1962}
\end{citebib}

\subsubsection{Distributed books}
% 14.141 Distributed books

\begin{citebib}
\item \cite{willke2007}
\end{citebib}

\subsubsection{Publication Date---General}
% 14.142 Publication Date---General
\label{14.142}

\begin{citebib}
\AtNextCitekey{\clearfield{shorthand}}
%\item \cite[6.56; cf. 16th ed. (2010), 6.54]{chicago2017}
\item \cite*[6.56]{chicago2017}; cf. 16th ed. \parencite*{chicago2010}, 6.54.
\item \cite{turabian2013}
\end{citebib}

\setcounter{subsubsection}{143}
\subsubsection{Multivolume works published over more than one year}
% 14.144 Multivolume works published over more than one year
\label{14.144}

\begin{citebib}
\item \cite[329]{hayek2011}
\item \cite{tillich1951}
\end{citebib}

\noindent Citing Hayek with the preamble or entry option
\opt{swapvol}:

\begin{citebib}
\AtNextCitekey{\toggletrue{swapvol}}
\item \cite*[329]{hayek2011}
\AtNextBibliography{\toggletrue{swapvol}}
\end{citebib}

\subsubsection{No date of publication}
% 14.145 No date of publication

\begin{bibonly}
\nocite{boston,edinburgh1750,edinburgh}
\end{bibonly}

\subsubsection{Forthcoming publications}
% 14.146 Forthcoming publications

\begin{citebib}
\item \cite{author}
\item \cite[345--46]{writer}
\item \cite{contributor}
\end{citebib}

\setcounter{subsubsection}{158}
\subsubsection{Books requiring a specific application or device (e-books)}
% 14.159 Books requiring a specific application or device (e-books)

\begin{citebib}
\item \cite{borel2015}
\end{citebib}

\setcounter{subsubsection}{160}
\subsubsection{Books consulted online}
% 14.161 Books consulted online

The first and third notes below present a challenge: If a bibliography
database entry contains an address for a work, such as a DOI, Windy
City prints it in the work's first, long citation. Such is the case
with the second note below. To cite an address for just part of a
work, but print one for the whole work in the bibliography, you need
to override the style's default behavior. The first and third notes do
this with a command that temporarily clears the work's DOI from its
bibliography database entry. Here's an example from the source:

\begin{verbatim}
   \AtNextCitekey{\clearfield{doi}}
   \item \cite[chap. 3, \url{https://doi.org/10.1093/acprof:oso/
               9780199343638.003.0004}]{bonds2014}
\end{verbatim}

\begin{citebib}
\AtNextCitekey{\clearfield{doi}}
\item \cite[chap. 3, \url{https://doi.org/10.1093/acprof:oso/9780199343638.003.0004}]{bonds2014}
\item \cite[59]{lystra2004}
\AtNextCitekey{\clearfield{doi}}
\item \cite[chap. 11, \url{https://doi.org/10.1093/acprof:oso/9780199343638.003.0012}]{bonds2014}
\item \cite[60--61]{lystra2004}
\end{citebib}

\subsubsection{Freely available electronic editions of older works}
% 14.162 Freely available electronic editions of older works

The first example comes close to \textit{CMOS}, except that,
consistent with \ref{14.114} and \ref{15.40}, it lists the edition by
Project Gutenberg as a reprint.

\begin{citebib}
\item \cite[bk. 6, chap. 1]{james2008}
\item \cite[1:243]{james1909}
\end{citebib}

\subsubsection{Books on CD-ROM and other fixed media}
% 14.163 Books on CD-ROM and other fixed media
\label{14.163}

\begin{citebib}
\item \cite*[1.4]{chicago2003}
\end{citebib}

\setcounter{subsection}{5}
\subsection{Periodicals}
\setcounter{subsection}{14}

\setcounter{subsubsection}{170}
\subsubsection{Journal volume, issue, and date}
% 14.171 Journal volume, issue, and date
\label{14.171}

The note for Harper includes the month of publication. Windy City
includes it in the bibliography as well, even though \textit{CMOS}
omits it. As for Lock's entry, \textit{CMOS} clearly errs in printing
the surname twice. Also, for Wilder's article, Windy City prints
\textit{nos.} before \textit{1/2}, not \textit{no.}, as \textit{CMOS}
has it. To print the correct season, Wilder's entry in the
bibliography database includes \textit{Fall} in the \bibfield{issue}
field. Using the \bibfield{date} field with \textit{2013-23} would
print \textit{Autumn 2013}.

\begin{citebib}
\item \cite[155]{lock2015}
\item \cite[651]{wesoky2015}
\item \cite[645]{harper2014}
\item \cite[60]{wilder2013}
\item \cite[52]{beattie1974}
\end{citebib}

\subsubsection{Forthcoming journal articles}
% 14.172 Forthcoming journal articles

\begin{citebib}
\item \cite{authora}
%\item \cite{jubb2015}
\end{citebib}

%\subsubsection{Journal article preprints}
%% 14.173 Journal article preprints

%\begin{citebib}
%\item \cite{huang2015}
%\end{citebib}

\setcounter{subsubsection}{173}
\subsubsection{Journal page references}
% 14.174 Journal page references

\begin{citebib}
\item \cite{gold2015}
\item \cite[2--3]{paudyal2015}
\end{citebib}

\subsubsection{Journal articles consulted online}
% 14.175 Journal articles consulted online

\begin{citebib}
\item \cite[268]{whitney1929}
\item \cite[260--61]{schoenfield2016}
\end{citebib}

\subsubsection{Access dates for journal articles}
% 14.176 Access dates for journal articles

\begin{citebib}
\item \cite[81]{narr2015}
\item \cite[88--89]{narr2015}
\end{citebib}

\setcounter{subsubsection}{177}
\subsubsection{Journal special issues}
% 14.178 Journal special issues

\begin{citebib}
\item \cite{tezuka2013}
\end{citebib}

\setcounter{subsubsection}{179}
\subsubsection{Articles published in installments}
% 14.180 Articles published in installments

By default, Windy City prints each installment as a separate entry. To
get the format for the series, you'd need to use the \bibtype{misc}
entry type.

\begin{citebib}
\item \cite[312]{brown1978}
\end{citebib}

\setcounter{subsubsection}{181}
\subsubsection{Place where journal is published}
% 14.182 Place where journal is published

\begin{citebib}
\item \cite[65--70]{luu1999}
\item \cite{garrett1975}
\end{citebib}

\subsubsection{Translated or edited article}
% 14.183 Translated or edited article

\begin{citebib}
\item \cite{authorb}
\item \cite{authorc}
\end{citebib}

\subsubsection{New series for journal volumes}
% 14.184 New series for journal volumes
\label{14.184}

\begin{citebib}
\item \cite[414]{sewall1896}
\item \cite{moraes1950}
\end{citebib}

\subsubsection{Short titles for articles}
% 14.185 Short titles for articles

\begin{citebib}
\item \cite[223]{rosenblum2015}
\item \cite[225]{rosenblum2015}
\end{citebib}

\subsubsection{Abstracts}
% 14.186 Abstracts

\begin{citebib}
\item \cite{matute2015}
\end{citebib}

\setcounter{subsubsection}{187}
\subsubsection{Basic citation format for magazine articles}
% 14.188 Basic citation format for magazine articles

\begin{citebib}
\item \cite[48]{saulnier2008}
\item \cite[59]{lepore2015}
\end{citebib}

\subsubsection{Magazine articles consulted online}
% 14.189 Magazine articles consulted online

\begin{citebib}
\item \cite{vick2015}
\item \cite[5]{hanemann1926}
\end{citebib}

\subsubsection{Magazine departments}
% 14.190 Magazine departments

\begin{citebib}
\item \cite{marx2015}
\item \cite{wallraff2008}
\item \cite{gourmet2000}
\end{citebib}

\subsubsection{Basic citation format for newspaper articles}
% 14.191 Basic citation format for newspaper articles

\begin{citebib}
\item \cite{editorial1990}
\item \cite{royko1992}
\item \cite{forester2000}
\item \cite{samenow2016}
\end{citebib}

\setcounter{subsubsection}{194}
\subsubsection{Regular columns or features}
% 14.195 Regular columns or features

\begin{citebib}
\item \cite{jaffe2015}
\item \cite{editorial2015}
\end{citebib}

\setcounter{subsubsection}{196}
\subsubsection{Weekend supplements, magazines, and the like}
% 14.197 Weekend supplements, magazines, and the like

\begin{citebib}
\item \cite[48]{ghansah2015}
\end{citebib}

\setcounter{subsubsection}{198}
\subsubsection{Unsigned newspaper articles}
% 14.199 Unsigned newspaper articles

\begin{citebib}
\item \cite{nytimes2002}
\end{citebib}

\subsubsection{News services and news releases}
% 14.200 News services and news releases

\begin{citebib}
\item \cite{ap2015}
\end{citebib}

\setcounter{subsubsection}{201}
\subsubsection{Book reviews}
% 14.202 Book reviews
\label{14.202}

\begin{citebib}
\item \cite[B13--B14]{ratliff1999}
\item \cite{kamp2006}
\item \cite{brehm2015}
\end{citebib}

\setcounter{subsubsection}{203}
\subsubsection{Unsigned reviews}
% 14.204 Unsigned reviews
\label{14.204}

On the use of the \bibfield{type} field to format this example
correctly, see section \ref{stdfields}.

\begin{citebib}
\item \cite{zeitung1828}
\end{citebib}

\setcounter{subsection}{6}
\subsection{Websites, Blogs, and Social Media}
\setcounter{subsection}{14}

\setcounter{subsubsection}{207}
\subsubsection{Citing blog posts and blogs}
% 14.208 Citing blog posts and blogs
\label{14.208}

In \textit{CMOS}, one citation refers to \textit{The Chronicle of
Higher Education} and another to \textit{Chronicle of Higher
Education}. The latter also appears in \ref{15.51}, so it's probably
correct.

\begin{citebib}
\item \cite{amlen2015}
\item \cite{germano2017}
\item \cite{amlen}
\item \cite{linguafranca}
\item \cite{jim2017}
\end{citebib}

\subsubsection{Citing social media content}
% 14.209 Citing social media content

\begin{citebib}
\item \cite{diaz2016}
\item \cite{obrien2015}
\item \cite{chicago2015}
\item \cite{licis2016}
\end{citebib}

\setcounter{subsection}{7}
\subsection{Papers, Contracts, and Reports}
\setcounter{subsection}{14}

\setcounter{subsubsection}{214}
\subsubsection{Theses and dissertations}
% 14.215 Theses and dissertations

\begin{citebib}
\item \cite[59]{vedrashko2006}
\item \cite{choi2008}
\end{citebib}

\setcounter{subsubsection}{216}
\subsubsection{Lectures and papers or posters presented at meetings}
% 14.217 Lectures and papers or posters presented at meetings

\begin{citebib}
\item \cite{hong2015}
\end{citebib}

\subsubsection{Working papers and the like}
% 14.218 Working papers and the like

\begin{citebib}
\item \cite{lucki1980}
\item \cite{bronfenbrenner2011}
\end{citebib}

\setcounter{subsubsection}{219}
\subsubsection{Pamphlets, reports, and the like}
% 14.220 Pamphlets, reports, and the like

\begin{citebib}
\item \cite{lifestyles1996}
\item \cite{mcdonalds2014}
\item \cite[¶2,620]{standardtax1996}
\end{citebib}

\setcounter{subsection}{8}
\subsection{Special Types of References}
\setcounter{subsection}{14}

\setcounter{subsubsection}{231}
\subsubsection{Reference works consulted in physical formats}
% 14.232 Reference works consulted in physical formats
\label{14.232}

Some reference works show full publication information in the same way
as books. Use the \bibtype{book} entry type for them. The first three
citations below are different. They need the \bibtype{reference} or
\bibtype{inreference} entry type. See section \ref{entrytypes} for
more information. Following the suggestion in \textit{CMOS} 14.232,
\bibtype{reference} and \bibtype{inreference} works don't appear in
bibliographies or reference lists.

\begin{citebib}
\item \cite{salvation1980}
\item \cite{hootananny2009}
\item \cite{dab1937}
\item \cite[s.vv. \mkbibquote{police ranks}, \mkbibquote{postal addresses}]{timestyle2003}
\item \cite[6.8.2]{mla2008}
\end{citebib}

\subsubsection{Reference works consulted online}
% 14.233 Reference works consulted online
\label{14.233}

Like some of the reference works in the previous section, the ones
below need the \bibtype{reference} or \bibtype{inreference} entry
type. As odd as it may seem, but consistent with \textit{CMOS}, they,
too, aren't included in bibliographies and reference lists. See
section \ref{entrytypes} for more information.

\begin{citeonly}
\item \cite{toscanini2016}
\item \cite{cairns2016}
\item \cite{wikipedia2016}
\item \cite{merriam2016}
\end{citeonly}

\subsubsection{Citing individual reference entries by author}
% 14.234 Citing individual reference entries by author

\begin{citebib}
\item \cite{isaacson2005}
\end{citebib}

\setcounter{subsubsection}{245}
\subsubsection{Citing specific editions of classical references}
% 14.246 Citing specific editions of classical references

\begin{citebib}
\item \cite{epictetus1916}
\end{citebib}

\setcounter{subsubsection}{250}
\subsubsection{Modern editions of the classics}
% 14.251 Modern editions of the classics

\begin{citebib}
\item \cite{aristotle1983}
\item \cite{maimonides1965}
\end{citebib}

\setcounter{subsubsection}{257}
\subsubsection{Patents}
% 14.258 Patents

\begin{citebib}
\item \cite{iizuka1986}
\end{citebib}

\setcounter{subsubsection}{259}
\subsubsection{Citations taken from secondary sources}
% 14.260 Citations taken from secondary sources

\begin{citebib}
\item \cite[269]{zukofsky1931}, quoted in \cite[78]{costello1981}
\end{citebib}

\section{Examples from \emph{CMOS} Chap. 15, ``Author-Date
References''}
\label{paren}

Examples in this section reproduce those in \textit{CMOS} chapter 15.
To help with cross-checking, subsection numbers and headings are from
\textit{CMOS}. Since parenthetical citations are relatively simple,
and since the format of references lists is derivative of the default,
the examples below are more selective than those in the previous
section.

\subsection{Basic Format, with Examples and Variations}
\setcounter{subsection}{15}

\setcounter{subsubsection}{8}
\subsubsection{Author-date references—examples and variations}
% 15.9 Author-date references—examples and variations
\label{15.9}

\begin{citeref}
\item \parencite[87--88]{strayed2012}
\item \parencite[32]{daum2015}
\item \parencite[188]{grazer2015}
\item \parencite[242--55]{garcia1988}
\item \parencite[310]{gould1984a}
\item \parencite[484--85]{bagley2015}
\item \parencite[312]{liu2015}
\end{citeref}

\setcounter{subsection}{1}
\subsection{Reference Lists and Text Citations}
\setcounter{subsection}{15}

\setcounter{subsubsection}{13}
\subsubsection{Placement of dates in reference list entries}
% 15.14 Placement of dates in reference list entries
\label{15.14}

\begin{citeref}
\item \parencite{pager2015}
\item \parencite{unger2014}
\end{citeref}

\setcounter{subsubsection}{19}
\subsubsection{Reference list entries with same author(s), same year}
% 15.20 Reference list entries with same author(s), same year
\label{15.20}

\begin{citeref}
\item \parencite[218]{fogel2004b}
\item \parencite[45--46]{fogel2004a}
\end{citeref}

\setcounter{subsubsection}{21}
\subsubsection{Text citations---basic form}
% 15.22 Text citations—basic form
\label{15.22}

Ignore the error in \textit{CMOS}: In reference lists, a title goes
after the year, not before.

\begin{citeref}
\item \parencite{hetherington2015}
\item \parencite{grove2015}
\item \parencite{hetherington2015,grove2015}
\end{citeref}

\begin{citeref}
\item \parencite{doershuk2017}
\item \parencite{doershuk2016}
\end{citeref}

\setcounter{subsubsection}{23}
\subsubsection{Additional material in text citations}
% 15.24 Additional material in text citations

\begin{citeref}
\item \parencite[; t-tests are used here]{mandolan2017}
\end{citeref}

\subsubsection{Text citations in relation to surrounding text and punctuation}
% 15.25 Text citations in relation to surrounding text and punctuation
\label{15.25}

\begin{quote} Fiorina et al. \parencite*{fiorina2005} and Fischer and
Hout \parencite*{fischer2006} reach more or less the same conclusions.
In contrast, Abramowitz and Saunders \parencite*{abramowitz2005}
suggest that the mass public is deeply divided between red states and
blue states and between churchgoers and secular voters. \end{quote}

\setcounter{subsubsection}{26}
\subsubsection{Several references to the same source}
% 15.27 Several references to the same source

\begin{citeref}% 'even\-tanned' prevents even--[break]tanned
\item Complexion figures prominently in Morgan's descriptions. When
Jasper compliments his mother's choice of car (a twelve-cylinder
Mediterranean roadster with leather and wood-grained interior), ``his
cheeks blotch indignantly, painted by jealousy and rage''
\parencite[47]{chaston2000}. On the other hand, his mother's mask
never changes, her ``even\-tanned good looks''
\parencite[56]{chaston2000}, ``burnished visage''
\parencite[101]{chaston2000}, and ``air-brushed confidence''
\parencite[211]{chaston2000} providing the foil to the drama in her
midst. \end{citeref}

\setcounter{subsubsection}{28}
\subsubsection{Text citations of works with more than three authors}
% 15.29 Text citations of works with more than three authors

\begin{citeref}
\item \parencite{schonen2017a}
\item \parencite{schonen2017b}
\end{citeref}

\subsubsection{Multiple text references}
% 15.30 Multiple text references

\begin{citeref}
\item \parencite{armstrong1989,beigl1989,pickett1985}
\item \parencites{whittaker1967,whittaker1975,wiens1989a,wiens1989b}
\item \parencites[328]{wong1999}[475]{wong2000}[67]{garcia1998}
\item \parencites{guest2006}[see also][]{stalle2008}{rahn2009}
\end{citeref}

\setcounter{subsection}{2}
\subsection{Author-Date References: Special Cases}
\setcounter{subsection}{15}

\setcounter{subsubsection}{33}
\subsubsection{Author-date format for anonymous works (no listed author)}
% 15.34 Author-date format for anonymous works (no listed author)

See section \ref{stdfields} on the use of the \bibfield{authtype}
field for anonymous works.

\begin{citeref}
\item \parencite{anon1610}
\item \parencite{anon1547}
\item \parencite{horsley1796}
\item \parencite{hawkes1834}
\end{citeref}

\subsubsection{Pseudonyms in author-date references}
% 15.35 Pseudonyms in author-date references

\begin{citeref}
\item \parencite{stendhal1925}
\end{citeref}

\subsubsection{Editor in place of author in text citations}
% 15.36 Editor in place of author in text citations

\begin{citeref}
\item \parencite{silverstein1974}
\item \parencite{soltes1999}
\end{citeref}

\subsubsection{Organization as author in author-date references}
% 15.37 Organization as author in author-date references

In the reference list, \textit{CMOS} errs in printing \textit{:1997}
after \textit{ISO 4}. Compare it with the nearly identical example in
\ref{14.84}.

\begin{citeref}
\item \parencite{iso1997.ref}
\end{citeref}

\setcounter{subsubsection}{39}
\subsubsection{Reprint editions and modern editions—more than one date}
% 15.40 Reprint editions and modern editions—more than one date
\label{15.40}

\begin{citeref}
\item \parencite{austen2003}
\item \parencite{maitland1998}
\end{citeref}

\subsubsection{Multivolume works published over more than one year}
% 15.41 Multivolume works published over more than one year
\label{15.41}

\begin{citeref}
\item \parencite[1:133]{tillich1951}
%\item \parencite[vol. 2]{tillich1951}
\item \parencite[329]{hayek2011}
\end{citeref}

\subsubsection{Cross-references to multiauthor books in reference lists}
% 15.42 Cross-references to multiauthor books in reference lists
\label{15.42}

\begin{citeref}
\item \parencite{draper1987}
\item \parencite{harrington1987}
\item \parencite{zukowsky1987}
\end{citeref}

\setcounter{subsubsection}{43}
\subsubsection{No date of publication in author-date references}
% 15.44 No date of publication in author-date references

\begin{citeref}
\item \parencite{nano1750}
\item \parencite{nano}
\end{citeref}

\subsubsection{``Forthcoming'' in author-date references}
% 15.45 ``Forthcoming'' in author-date references

\begin{citeref}
\item \parencite{faraday}
\end{citeref}

\setcounter{subsubsection}{46}
\subsubsection{Parentheses or comma with issue number}
% 15.47 Parentheses or comma with issue number

In the second reference list entry below, Windy City prints a colon
after the journal number. \textit{CMOS} prints a comma there---likely
an error.

\begin{citeref}
\item \parencite{glass2014}
\item \parencite{meyerovitch1959}
\end{citeref}

\subsubsection{Colon with volume number}
% 15.48 Colon with volume number

The example below shows the output when an article's publication month
isn't included in the bibliography database and so doesn't come
between a volume number and a page reference.

\begin{citeref}
\item \parencite{gunderson2015}
\end{citeref}

\subsubsection{Newspapers and magazines in reference lists}
% 15.49 Newspapers and magazines in reference lists

\begin{citeref}
\item \parencite{nytimes2002}
\end{citeref}

\setcounter{subsubsection}{50}
\subsubsection{Citing blogs in author-date format}
% 15.51 Citing blogs in author-date format
\label{15.51}

\textit{CMOS} seems mistaken in printing a period after
\textit{Chronicle of Higher Education} instead of a comma. Compare
with \ref{14.208} and \ref{15.42}.

\begin{citeref}
\item \parencite{germano2017}
\end{citeref}

\subsubsection{Citing social media content in author-date format}
% 15.52 Citing social media content in author-date format
\label{15.52}

\begin{citeref}
\item \parencite{diaz2016}
\item \parencite{obrien2015}
\item \parencite{chicago2015}
\end{citeref}

\setcounter{subsubsection}{54}
\subsubsection{Patents or other documents cited by more than one date}
% 15.55 Patents or other documents cited by more than one date

\begin{citeref}
\item \parencite{iizuka1986}
\end{citeref}

\subsubsection{``Quoted in'' in author-date references}
% 15.56 “Quoted in” in author-date references

\begin{citeref}
\item In Louis Zukofsky's ``Sincerity and Objectification,''
from the February 1931 issue of \textit{Poetry} magazine
\parencite[quoted in][]{costello1981}\ldots
\end{citeref}

\defbibnote{sh}{This section shows the output of \cmd{printbiblist}
with the argument \opt{shorthand}. Running \cmd{printshorthands}
produces the same output. By default, works in this list also appear
in bibliographies. To exclude them, use the preamble option
\opt{nolos}. See section \ref{preamble} for more information.\\}%

\defbibnote{bib}{This section shows the default output of
\cmd{printbibliography}. The next section shows the author-date
format.\\}%

\defbibnote{ref}{This section shows the output of
\cmd{printbibliography} in the author-date format. For information on
how to produce this output, see section \ref{preamble}. Issues with
sorting in this section are mentioned there. Creating a reference list
in the preferred way, with the \opt{reflist} preamble option, should
prevent those issues.\\}%

\printbiblist[prenote=sh]{shorthand}
\refstepcounter{sh}\label{sh}
\printbibliography[notkeyword=notinbib,prenote=bib]
\refstepcounter{bib}\label{bib}
\printbibliography[%
  env=reflist,
  heading=references,
  notkeyword=notinref,
  prenote=ref]
\refstepcounter{ref}\label{ref}
\end{document}
